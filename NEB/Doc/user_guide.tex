\documentclass[12pt,a4paper]{article}
\def\version{4.3.2}
\def\qe{{\sc Quantum ESPRESSO}}
\def\neb{{\sc NEB}}
\usepackage{html}

% BEWARE: don't revert from graphicx for epsfig, because latex2html
% doesn't handle epsfig commands !!!
\usepackage{graphicx}

\textwidth = 17cm
\textheight = 24cm
\topmargin =-1 cm
\oddsidemargin = 0 cm

\def\pwx{\texttt{pw.x}}
\def\cpx{\texttt{cp.x}}
\def\phx{\texttt{ph.x}}
\def\nebx{\texttt{neb.x}}
\def\configure{\texttt{configure}}
\def\PWscf{\texttt{PWscf}}
\def\PHonon{\texttt{PHonon}}
\def\CP{\texttt{CP}}
\def\PostProc{\texttt{PostProc}}
\def\make{\texttt{make}}

\begin{document} 
\author{}
\date{}

\def\qeImage{../../Doc/quantum_espresso.pdf}
\def\democritosImage{../../Doc/democritos.pdf}

\begin{htmlonly}
\def\qeImage{../../Doc/quantum_espresso.png}
\def\democritosImage{../../Doc/democritos.png}
\end{htmlonly}

\title{
  \includegraphics[width=5cm]{\qeImage} \hskip 2cm
  \includegraphics[width=6cm]{\democritosImage}\\
  \vskip 1cm
  % title
  \Huge User's Guide for \neb \ \smallskip
  \Large (version \version)
}
%\endhtmlonly

%\latexonly
%\title{
% \epsfig{figure=quantum_espresso.png,width=5cm}\hskip 2cm
% \epsfig{figure=democritos.png,width=6cm}\vskip 1cm
%  % title
%  \Huge User's Guide for \qe \smallskip
%  \Large (version \version)
%}
%\endlatexonly

\maketitle

\tableofcontents

\section{Introduction}

This guide covers the installation and usage of \neb \ (opEn-Source 
Package for energy barriers and reaction pathway calculations through the
Nudged Elastic Band method
, version \version.

\neb \ is part of the \qe \ distribution and can not be compiled nor used independently.

\neb \ calculates reaction pathways and energy barriers 
using the Nudged Elastic Band (NEB) and Fourier String Method Dynamics 
(SMD) methods. Note that these calculations are no longer performed
by the \pwx\ executable of \PWscf. Also note that NEB with Car-Parrinello
Molecular Dynamics is currently not implemented.

The original \qe \ implementation of the Nudged Elastic Band and 
String Method Dynamics was developed by Carlo Sbraccia (Princeton).

\neb\ is free software, released under the 
GNU General Public License. \\ See
\texttt{http://www.gnu.org/licenses/old-licenses/gpl-2.0.txt}, 
or the file License in the distribution).
    
We shall greatly appreciate if scientific work done using this code will 
contain an explicit acknowledgment and the following reference:
\begin{quote}
P. Giannozzi, S. Baroni, N. Bonini, M. Calandra, R. Car, C. Cavazzoni,
D. Ceresoli, G. L. Chiarotti, M. Cococcioni, I. Dabo, A. Dal Corso,
S. Fabris, G. Fratesi, S. de Gironcoli, R. Gebauer, U. Gerstmann,
C. Gougoussis, A. Kokalj, M. Lazzeri, L. Martin-Samos, N. Marzari,
F. Mauri, R. Mazzarello, S. Paolini, A. Pasquarello, L. Paulatto,
C. Sbraccia, S. Scandolo, G. Sclauzero, A. P. Seitsonen, A. Smogunov,
P. Umari, R. M. Wentzcovitch, J.Phys.:Condens.Matter 21, 395502 (2009),
http://arxiv.org/abs/0906.2569
\end{quote}

All trademarks mentioned in this guide belong to their respective owners.

\section{Installation}

\subsection{Download}
 
is automatically downloaded from \texttt{http://qe-forge.org/frs/?group\_id=37}


\subsection{Prerequisites}
\label{Sec:Installation}

To install \neb\ from source, you need first of all a minimal Unix 
environment and the \qe \ distribution.

\subsection{\configure}

For instruction on how to install the \qe\ source package please refer to
the \qe \ general documentation (user\_guide) that is available at
\begin{verbatim}
http://www.quantum-espresso.org/wiki/index.php/Main_Page. 
\end{verbatim}

After the correct installation of \qe \, the installation and compilation of \neb \
can be done just typing \texttt{make neb} from the main espresso directory.

\subsection{Compilation}

Typing \texttt{make neb} will produce the following codes in \texttt{NEB/src}:
\begin{itemize}
\item \nebx: calculates reaction barriers and pathways using NEB.
\item \texttt{path\_int.x}: used by utility \texttt{path\_int.sh}
  that generates, starting from a path (a set of images), a new one with a 
  different number of images. The initial and final points of the new
 path can differ from those in the original one. 
\end{itemize}

Symlinks to executable programs will be placed in the
\texttt{bin/} subdirectory.

\subsection{Running examples}
\label{SubSec:Examples}
As a final check that compilation was successful, you may want to run some or
all of the examples. You should first of all ensure that you have downloaded 

To run the examples, you should follow this procedure:
\begin{enumerate}   
\item Go to the \texttt{examples/} directory from the main \qe \ directory and edit the 
  \texttt{environment\_variables} file, setting the following variables as needed: 
\begin{quote}
   BIN\_DIR: directory where executables reside\\
   PSEUDO\_DIR: directory where pseudopotential files reside\\
   TMP\_DIR: directory to be used as temporary storage area
\end{quote}
\end{enumerate}
The default values of BIN\_DIR and PSEUDO\_DIR should be fine, 
unless you have installed things in nonstandard places. TMP\_DIR 
must be a directory where you have read and write access to, with 
enough available space to host the temporary files produced by the 
example runs, and possibly offering high I/O performance (i.e., don't 
use an NFS-mounted directory). NOTA BENE: do not use a
directory containing other data, the examples will clean it!
If you have compiled the parallel version of \qe\ (this
is the default if parallel libraries are detected), you will usually
have to specify a driver program (such as \texttt{mpirun} or \texttt{mpiexec}) 
and the number of processors: see Sec.\ref{SubSec:para} for
details. In order to do that, edit again the \texttt{environment\_variables} 
file
and set the PARA\_PREFIX and PARA\_POSTFIX variables as needed. 
Parallel executables will be run by a command like this: 
\begin{verbatim}
      $PARA_PREFIX neb.x $PARA_POSTFIX -inp file.in > file.out
\end{verbatim}
For example, if the command line is like this (as for an IBM SP):
\begin{verbatim}
      poe neb.x -procs 4 -inp file.in > file.out
\end{verbatim}
you should set PARA\_PREFIX="poe", PARA\_POSTFIX="-procs
4". Furthermore, if your machine does not support interactive use, you
must run the commands specified below through the batch queuing
system installed on that machine. Ask your system administrator for
instructions. 

Go to NEB/examples and execute:
\begin{verbatim}
      ./run_example
\end{verbatim}
This will create a subdirectory results, containing the input and
output files generated by the calculation.

The \texttt{reference/} subdirectory contains
verified output files, that you can check your results against. They
were generated on a Linux PC using the Intel compiler. On different
architectures the precise numbers could be slightly different, in
particular if different FFT dimensions are automatically selected. For
this reason, a plain diff of your results against the reference data
doesn't work, or at least, it requires human inspection of the
results. 

\section{Parallelism}
\label{Sec:para}

\subsection{Understanding Parallelism}

Two different parallelization paradigms are currently implemented 
in \qe:
\begin{enumerate}
\item {\em Message-Passing (MPI)}. A copy of the executable runs 
on each CPU; each copy lives in a different world, with its own
private set of data, and communicates with other executables only
via calls to MPI libraries. MPI parallelization requires compilation 
for parallel execution, linking with MPI libraries, execution using 
a launcher program (depending upon the specific machine). The number of CPUs used
is specified at run-time either as an option to the launcher or
by the batch queue system. 
\item {\em OpenMP}.  A single executable spawn subprocesses
(threads) that perform in parallel specific tasks. 
OpenMP can be implemented via compiler directives ({\em explicit} 
OpenMP) or via {\em multithreading} libraries  ({\em library} OpenMP).
Explicit OpenMP require compilation for OpenMP execution;
library OpenMP requires only linking to a multithreading
version of mathematical libraries, e.g.:
ESSLSMP, ACML\_MP, MKL (the latter is natively multithreading).
The number of threads is specified at run-time in the environment 
variable OMP\_NUM\_THREADS. 
\end{enumerate}

MPI is the well-established, general-purpose parallelization.
In \qe\ several parallelization levels, specified at run-time
via command-line options to the executable, are implemented
with MPI. This is your first choice for execution on a parallel 
machine.

Library OpenMP is a low-effort parallelization suitable for
multicore CPUs. Its effectiveness relies upon the quality of 
the multithreading libraries and the availability of 
multithreading FFTs. If you are using MKL,\footnote{Beware: 
MKL v.10.2.2 has a buggy \texttt{dsyev} yielding wrong results 
with more than one thread; fixed in v.10.2.4}
you may want to select FFTW3 (set \texttt{CPPFLAGS=-D\_\_FFTW3...}
in \texttt{make.sys}) and to link with the MKL interface to FFTW3. 
You will get a decent speedup ($\sim 25$\%) on two cores.

Explicit OpenMP is a very recent addition, still at an 
experimental stage, devised to increase scalability on
large multicore parallel machines. Explicit OpenMP is 
devised to be run together with MPI and also together 
with multithreaded libraries. BEWARE: you have to be VERY 
careful to prevent conflicts between the various kinds of
parallelization. If you don't know how to run MPI processes
and OpenMP threads in a controlled manner, forget about mixed 
OpenMP-MPI parallelization.

\subsection{Running on parallel machines}
\label{SubSec:para}

Parallel execution is strongly system- and installation-dependent. 
Typically one has to specify:
\begin{enumerate}
\item a launcher program (not always needed), 
such as \texttt{poe}, \texttt{mpirun}, \texttt{mpiexec},
  with the  appropriate options (if any);
\item the number of processors, typically as an option to the launcher
  program, but in some cases to be specified after the name of the
  program to be
  executed; 
\item the program to be executed, with the proper path if needed: for
  instance, \pwx, or \texttt{./pw.x}, or \texttt{\$HOME/bin/pw.x}, or
  whatever applies; 
\item other \qe-specific parallelization options, to be
  read and interpreted by the running code: 
\begin{itemize}
\item the number of ``images'' used by NEB or phonon calculations;
\item the number of ``pools'' into which processors are to be grouped
  (\pwx\ only);
\item the number of ``task groups'' into which processors are to be
  grouped;
\item the number of processors performing iterative diagonalization
  (for \pwx) or orthonormalization (for \cpx).
\end{itemize}
\end{enumerate}
Items 1) and 2) are machine- and installation-dependent, and may be 
different for interactive and batch execution. Note that large
parallel machines are  often configured so as to disallow interactive
execution: if in doubt, ask your system administrator.
Item 3) also depend on your specific configuration (shell, execution
path, etc). 
Item 4) is optional but may be important: see the following section
for the meaning of the various options.

For illustration, here is how to run \pwx\ on 16 processors partitioned into
8 pools (2 processors each), for several typical cases. 

IBM SP machines, batch:
\begin{verbatim}
   pw.x -npool 8 < input
\end{verbatim}
This should also work interactively, with environment variables NPROC
set to 16, MP\_HOSTFILE set to the file containing a list of processors.

\subsection{Parallelization levels}

Data structures are distributed across processors.
Processors are organized in a hierarchy of groups, 
which are identified by different MPI communicators level.
The groups hierarchy is as follow:
\begin{verbatim}
                 /  pools _ task   groups
  world _ images
                 \ linear-algebra  groups
\end{verbatim}

{\bf world}: is the group of all processors (MPI\_COMM\_WORLD).

{\bf images}: Processors can then be divided into different "images",
corresponding to a point in configuration space (i.e. to
a different set of atomic positions) for NEB calculations;
to one (or more than one) "irrep" or wave-vector in phonon
calculations.

{\bf Communications}:
Images and pools are loosely coupled and processors communicate
between different images and pools only once in a while, whereas
processors within each pool are tightly coupled and communications
are significant. This means that Gigabit ethernet (typical for
cheap PC clusters) is ok up to 4-8 processors per pool, but {\em fast}
communication hardware (e.g. Mirynet or comparable) is absolutely 
needed beyond 8 processors per pool.

{\bf Choosing parameters}:
To control the number of processors in each group,
command line switches: \texttt{-nimage}, \texttt{-npools},
\texttt{-ntg}, \texttt{northo} (for \cpx) or \texttt{-ndiag}
(for \pwx) are used.
As an example consider the following command line:
\begin{verbatim}
mpirun -np 4096 ./pw.x -nimage 8 -npool 2 -ntg 8 -ndiag 144 -input my.input
\end{verbatim}
This executes \PWscf\ on 4096 processors, to simulate a system
with 8 images, each of which is distributed across 512 processors.
k-points are distributed across 2 pools of 256 processors each, 
3D FFT is performed using 8 task groups (64 processors each, so
the 3D real-space grid is cut into 64 slices), and the diagonalization
of the subspace Hamiltonian is distributed to a square grid of 144
processors (12x12).

Default values are: \texttt{-nimage 1 -npool 1 -ntg 1} ; 
\texttt{ndiag} is set to 1 if ScaLAPACK is not compiled,
it is set to the square integer smaller than or equal to  half the number 
of processors of each pool.

\newpage

\section{Using \neb}

{\bf Reminder 1:} NEB calculations are no longer performed by \pwx.
In order to perform a NEB calculation, you should compile
\texttt{NEB/neb.x} (command \texttt{make neb}). 

\noindent {\bf  Reminder 2:} \nebx\ does not read from standard input.
You cannot use input redirection, as in \texttt{neb.x < neb.in ...}.

A NEB calculation can be run in two different ways:
\begin{enumerate}
\item by reading a single input file, to be specified using command-line 
options \texttt{neb.x -inp} or \texttt{neb.x -input};
\item by specifying the number $N$ of images (i.e. points in the 
configuration space), using command-line option 
\texttt{neb.x -input\_images N}, and providing input data in 
a \texttt{neb.dat} file and in files \texttt{pw\_X.in}, $X=1,...,N$.
\end{enumerate}

For case 1) the input file contains KEYWORDS (see below for format 
specifications). 
These KEYWORDS enable the code to distinguish which part of the file 
contains NEB-specific data and which parts contains input data for the
energy and force computational engine (currently only PW).
After the parsing, different files are generated: \texttt{neb.dat}, with
NEB-specific input data, and a set of PWscf input files 
\texttt{pw\_1.in},..,\texttt{pw\_N.in} PWscf input files, one for each
set of atomic position. All options for a single SCF calculation apply.

The general structure of the file to be parsed is:

\begin{verbatim}
BEGIN
BEGIN_PATH_INPUT
~... neb specific namelists and cards
END_PATH_INPUT
BEGIN_ENGINE_INPUT
BEGIN_ENGINE_INPUT
~...pw specific namelists and cards
BEGIN_POSITIONS
FIRST_IMAGE
~...pw ATOMIC_POSITIONS card
INTERMEDIATE_IMAGE
~...pw ATOMIC_POSITIONS card
LAST_IMAGE
~...pw ATOMIC_POSITIONS card
END_POSITIONS
~... other pw specific cards
END_ENGINE_INPUT
END
\end{verbatim}

For case 2), the \texttt{neb.dat} file and all \texttt{pw\_X.in} 
should be already present. 

A detailed description of all NEB-specific input variables 
is contained in files \texttt{Doc/INPUT\_NEB.*}. See Example 17. 

A NEB calculation will produce a number of files in the current directory
(i.e. in the directory were the code is run) containing additional information
on the minimum-energy path. The files are organized as following
(where \texttt{prefix} is specified in the input file):
\begin{description}
\item[\texttt{prefix.dat}]
is a three-column file containing the position of each image on the reaction
coordinate (arb. units), its energy in eV relative to the energy of the first image
and the residual error for the image in eV/$a_0$.
\item[\texttt{prefix.int}]
contains an interpolation of the path energy profile that pass exactly through each
image; it is computed using both the image energies and their derivatives
\item[\texttt{prefix.path}]
information used by \qe\ 
to restart a path calculation, its format depends on the input
details and is undocumented
\item[\texttt{prefix.axsf}]
atomic positions of all path images in the XCrySDen animation format:
to visualize it, use \texttt{xcrysden -\--axsf prefix.axsf}
\item[\texttt{prefix.xyz}]
atomic positions of all path images in the generic xyz format, used by
many quantum-chemistry softwares
\item[\texttt{prefix.crd}]
path information in the input format used by \pwx, suitable for a manual
restart of the calculation
\end{description}

"NEB calculation are a bit tricky in general and require extreme care to be
setup correctly. NEB also takes easily hundreds of iteration to converge,
of course depending on the number of atoms and of images. Here is some
free advice:
\begin{enumerate}
\item 
Don't use Climbing Image (CI) from the beginning. It makes convergence slower, 
especially if the special image changes during the convergence process (this 
may happen if \texttt{CI\_scheme='auto'} and if it does it may mess up everything).
Converge your calculation, then restart from the last configuration with
CI option enabled (note that this will {\em increase} the barrier).
\item
Carefully choose the initial path. Remember that \qe\ assumes continuity
between the first and the last image at the initial condition. In other 
words, periodic images are NOT used; you may have to manually translate
an atom by one or more unit cell base vectors in order to have a meaningful
initial path. You can visualize NEB input files with XCrySDen as animations,
take some time to check if any atoms overlap or get very close in the initial
path (you will have to add intermediate images, in this case).
\item
Try to start the NEB process with most atomic positions fixed, 
in order to converge the more "problematic" ones, before leaving
all atoms move.
\item
Especially for larger systems, you can start NEB with lower accuracy 
(less k-points, lower cutoff) and then increase it when it has
converged to refine your calculation.
\item
Use the Broyden algorithm instead of the default one: it is a bit more
fragile, but it removes the problem of "oscillations" in the calculated
activation energies. If these oscillations persist, and you cannot afford 
more images, focus to a smaller problem, decompose it into pieces.
\item
A gross estimate of the required number of iterations is
(number of images) * (number of atoms) * 3. Atoms that do not
move should not be counted. It may take half that many iterations, 
or twice as many, but more or less that's the order of magnitude, 
unless one starts from a very good or very bad initial guess.
\end{enumerate}
(Courtesy of Lorenzo Paulatto)


\section{Performances}

\subsection{Execution time}

Since v.4.2 \qe\ prints real (wall) time instead of CPU time.

The following is a rough estimate of the complexity of a plain
scf calculation with \pwx, for NCPP. USPP and PAW 

\subsection{File space requirements}

A typical \pwx\ run will require an amount of temporary disk space in the
order of O double precision complex numbers:
$$O = N_k M N + q N_{r1} N_{r2}N_{r3}$$

\section{Troubleshooting}

Almost all problems in \qe\ arise from incorrect input data 
and result in
error stops. Error messages should be self-explanatory, but unfortunately
this is not always true. If the code issues a warning messages and continues,
pay attention to it but do not assume that something is necessarily wrong in
your calculation: most warning messages signal harmless problems.

\paragraph{errors in examples with parallel execution}

If you get error messages in the example scripts -- i.e. not errors in
the codes -- on a parallel machine, such as e.g.: 
{\em run example: -n: command not found}
you may have forgotten 
the " " in the definitions of PARA\_PREFIX and PARA\_POSTFIX.

\end{document}
