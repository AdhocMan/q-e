\documentclass[12pt,a4paper]{article}
\def\version{6.6}
\def\qe{{\sc Quantum ESPRESSO}}

\usepackage{html}

% BEWARE: don't revert from graphicx for epsfig, because latex2html
% doesn't handle epsfig commands !!!
\usepackage{graphicx}

\textwidth = 17cm
\textheight = 24cm
\topmargin =-1 cm
\oddsidemargin = 0 cm

\def\pwx{\texttt{pw.x}}
\def\cpx{\texttt{cp.x}}
\def\phx{\texttt{ph.x}}
\def\nebx{\texttt{neb.x}}
\def\configure{\texttt{configure}}
\def\PWscf{\texttt{PWscf}}
\def\PHonon{\texttt{PHonon}}
\def\CP{\texttt{CP}}
\def\PostProc{\texttt{PostProc}}
\def\make{\texttt{make}}

\begin{document} 
\author{}
\date{}

\def\qeImage{../../Doc/quantum_espresso.pdf}

\title{
  \includegraphics[width=5cm]{\qeImage} \\
  % title
  \Huge \CP\ User's Guide (v. \version)
  \\ \Large (only partially updated)
}

\maketitle

\tableofcontents

\section{Introduction}

This guide covers the  usage of the
\CP\ package, version \version, a core component 
of the \qe\ distribution.
Further documentation, beyond what is provided 
in this guide, can be found in the directory
\texttt{CPV/Doc/}, containing a copy of this guide.

{\em Important notice: due to the lack of time and of manpower, this
manual is only partially updated and may contain outdated information.}

This guide assumes that you know the physics 
that \CP\ describes and the methods it implements.
It also assumes  that you have already installed,
or know how to install, \qe. If not, please read
the general User's Guide for \qe, found in 
directory \texttt{Doc/} two levels above the 
one containing this guide; or consult the web site:\\
\texttt{http://www.quantum-espresso.org}.

People who want to modify or contribute to 
\CP\ should read the Developer Manual: \\
\texttt{Doc/developer\_man.pdf}.

\CP\ can perform Car-Parrinello molecular dynamics, including
variable-cell dynamics, and free-energy surface calculation at
fixed cell through meta-dynamics, if patched with PLUMED.

The \CP\ package is based on the original code written by Roberto Car
and Michele Parrinello. \CP\ was developed by Alfredo Pasquarello
(EPF Lausanne), Kari Laasonen (Oulu), Andrea Trave, Roberto
Car (Princeton), Nicola Marzari (EPF Lausanne), Paolo Giannozzi, and others.
FPMD, later merged with \CP, was developed by Carlo Cavazzoni, 
Gerardo Ballabio (CINECA), Sandro Scandolo (ICTP), 
Guido Chiarotti, Paolo Focher, and others.
We quote in particular:
\begin{itemize}
  \item Federico Grasselli and Riccardo Bertossa (SISSA) for bug fixes,
        extensions to Autopilot;
  \item Biswajit Santra, Hsin-Yu Ko, Marcus Calegari Andrade (Princeton)
	for SCAN functional;
  \item Robert DiStasio (Cornell)), Biswajit Santra, and Hsin-Yu Ko 
	for hybrid functionals with MLWF;
        (maximally localized Wannier functions);
  \item Manu Sharma (Princeton) and Yudong Wu (Princeton) for dynamics
    with MLWF;
  \item Paolo Umari (Univ. Padua) for finite electric fields and conjugate
   gradients;
  \item Paolo Umari and Ismaila Dabo for ensemble-DFT;
  \item Xiaofei Wang (Princeton) for META-GGA;
  \item The Autopilot feature was implemented by Targacept, Inc.
\end{itemize}
This guide has been mostly writen by Gerardo Ballabio and Carlo Cavazzoni.

\CP\ is free software, released under the
GNU General Public License. \\ See
\texttt{http://www.gnu.org/licenses/old-licenses/gpl-2.0.txt},
or the file License in the distribution).

%%%%%%%%%%%%%%%%%%%%%%%%%%%%%%%%%%%%%%%%%%%%%%%%%%%%%%%%%%%%%%%%%
\input ../../Doc/quote.tex
%%%%%%%%%%%%%%%%%%%%%%%%%%%%%%%%%%%%%%%%%%%%%%%%%%%%%%%%%%%%%%%%%

\section{Compilation}

\CP\ is included in the core \qe\ distribution.
Instruction on how to install it can be found in the
general documentation (User's Guide) for \qe.

Typing \texttt{make cp} from the main \qe\ directory
or \make\ from the \texttt{CPV/} subdirectory produces the following codes in \texttt{CPV/src}:
\begin{itemize}
\item \cpx: Car-Parrinello Molecular Dynamics
code
\item \texttt{cppp.x}: postprocessing code for \cpx. 
	See \texttt{Doc/INPUT\_CPPP.*} for input variables.
\item \texttt{wfdd.x}: utility code for finding maximally
localized Wannier functions using damped dynamics.
\end{itemize}
Symlinks to executable programs will be placed in the \texttt{bin/} subdirectory. 

As a final check that compilation was successful,
you may want to run some or all of the tests
and examples. 
Automated tests for \cpx\ are in directory 
\texttt{test-suite/} and can be run via the 
\texttt{Makefile} found there.
Please see the general User's Guide for their setup. 

You may take the tests and examples distributed 
with \CP\ as templates for writing your own input
files. Input files for tests are contained in
subdirectories \texttt{test-suite/cp\_*} with file type 
\texttt{*.in1}, \texttt{*.in2}, ... . Input files for examples
are produced, if you run the examples, in the 
\texttt{results/} subdirectories, with names ending
with \texttt{.in}.

For general information on parallelism and how 
to run in parallel execution, please see the general User's Guide. \CP\  currently can take advantage
of both MPI and OpenMP parallelization. The
``plane-wave'', ``linear-algebra'' and ``task-group''
parallelization levels are implemented.
  
\section{Input data}

Input data for \cpx\ is organized into several namelists, followed by other 
fields (``cards'') introduced by keywords. The namelists are

\begin{tabular}{ll}
      \&CONTROL:& general variables controlling the run\\
      \&SYSTEM: &structural information on the system under investigation\\
      \&ELECTRONS: &electronic variables, electron dynamics\\
      \&IONS : &ionic variables, ionic dynamics\\
      \&CELL (optional): &variable-cell  dynamics\\
\end{tabular}
 \\
The \texttt{\&CELL} namelist may be omitted for
fixed-cell calculations. This depends on the value of variable \texttt{calculation}
in namelist \&CONTROL. Most variables in namelists have default values. Only
the following variables in \&SYSTEM must always be specified:

\begin{tabular}{lll}
      \texttt{ibrav} & (integer)& Bravais-lattice index\\
      \texttt{celldm} &(real, dimension 6)& crystallographic constants\\
      \texttt{nat} &(integer)& number of atoms in the unit cell\\
      \texttt{ntyp} &(integer)& number of types of atoms in the unit cell\\
      \texttt{ecutwfc} &(real)& kinetic energy cutoff (Ry) for wavefunctions.
\end{tabular}    \\).
    
Explanations for the meaning of variables \texttt{ibrav} and \texttt{celldm},
as well as on alternative ways to input structural data,
are contained in files \texttt{Doc/INPUT\_CP.*}. These files are the reference for input data and describe 
a large number of other variables as well. Almost all variables have default 
values, which may or may not fit your needs.

Comment lines in namelists can be introduced by a "!", exactly as in 
fortran code. 

After the namelists, you have several fields (``cards'')
introduced by keywords with self-explanatory names:
\begin{quote}
       ATOMIC\_SPECIES\\
       ATOMIC\_POSITIONS\\
       CELL\_PARAMETERS (optional)\\
       OCCUPATIONS (optional)\\
\end{quote}
The keywords may be followed on the same line by an option. Unknown
fields are ignored. 
See the files mentioned above for details on the available ``cards''.

Comments lines in ``cards'' can be introduced by either a ``!'' or a ``\#''
character in the first position of a line.
 
\subsection{Data files}

The output data files are written in the directory specified by variable
\texttt{outdir}, with names specified by variable \texttt{prefix} (a string that is prepended
to all file names, whose default value is: \texttt{prefix='pwscf'}). The \texttt{iotk}
toolkit is used to write the file in a XML format, whose definition can
be found in the Developer Manual. In order to use the data directory
on a different machine, you need to convert the binary files to formatted
and back, using the \texttt{bin/iotk} script.

The execution stops if you create a file \texttt{prefix.EXIT} either 
in the working directory (i.e. where the program is executed), or in 
the  \texttt{outdir} directory. Note that with some versions of MPI, 
the working directory  is the directory where the executable is! 
The advantage of this procedure is that all files are properly closed, 
whereas  just killing the process may leave data and output files in 
an unusable state.

\subsection{Format of arrays containing charge density, potential, etc.}

The index of arrays used to store functions defined on 3D meshes is
actually a shorthand for three indices, following the FORTRAN convention 
("leftmost index runs faster"). An example will explain this better. 
Suppose you have a 3D array \texttt{psi(nr1x,nr2x,nr3x)}. FORTRAN 
compilers store this array sequentially  in the computer RAM in the following way:
\begin{verbatim}
        psi(   1,   1,   1)
        psi(   2,   1,   1)
        ...
        psi(nr1x,   1,   1)
        psi(   1,   2,   1)
        psi(   2,   2,   1)
        ...
        psi(nr1x,   2,   1)
        ...
        ...
        psi(nr1x,nr2x,   1)
        ...
        psi(nr1x,nr2x,nr3x)
etc
\end{verbatim}
Let \texttt{ind} be the position of the \texttt{(i,j,k)} element in the above list: 
the following relation
\begin{verbatim}
        ind = i + (j - 1) * nr1x + (k - 1) *  nr2x * nr1x
\end{verbatim}
holds. This should clarify the relation between 1D and 3D indexing. In real
space, the \texttt{(i,j,k)} point of the FFT grid with dimensions 
\texttt{nr1} ($\le$\texttt{nr1x}), 
\texttt{nr2}  ($\le$\texttt{nr2x}), , \texttt{nr3} ($\le$\texttt{nr3x}), is
$$
r_{ijk}=\frac{i-1}{nr1} \tau_1  +  \frac{j-1}{nr2} \tau_2 +
\frac{k-1}{nr3} \tau_3 
$$
where the $\tau_i$ are the basis vectors of the Bravais lattice. 
The latter are stored row-wise in the \texttt{at} array:
$\tau_1 = $ \texttt{at(:, 1)}, 
$\tau_2 = $ \texttt{at(:, 2)}, 
$\tau_3 = $ \texttt{at(:, 3)}.

The distinction between the dimensions of the FFT grid,
\texttt{(nr1,nr2,nr3)} and the physical dimensions of the array,
\texttt{(nr1x,nr2x,nr3x)} is done only because it is computationally
convenient in some cases that the two sets are not the same.
In particular, it is often convenient to have \texttt{nrx1}=\texttt{nr1}+1
to reduce memory conflicts.

\section{Using \CP}

It is important to understand that a CP simulation is a sequence of different 
runs, some of them used to "prepare" the initial state of the system, and 
other performed to collect statistics, or to modify the state of the system
itself, i.e. modify the temperature or the pressure.
    
To prepare and run a CP simulation you should first of all
define the system:
  \begin{quote}
    atomic positions\\
    system cell\\
    pseudopotentials\\
    cut-offs\\
    number of electrons and bands (optional)\\
    FFT grids (optional)
  \end{quote}
An example of input file (Benzene Molecule):
\begin{verbatim}
         &control
            title = 'Benzene Molecule',
            calculation = 'cp',
            restart_mode = 'from_scratch',
            ndr = 51,
            ndw = 51,
            nstep = 100,
            iprint = 10,
            isave = 100,
            tstress = .TRUE.,
            tprnfor = .TRUE.,
            dt    = 5.0d0,
            etot_conv_thr = 1.d-9,
            ekin_conv_thr = 1.d-4,
            prefix = 'c6h6',
            pseudo_dir='/scratch/benzene/',
            outdir='/scratch/benzene/Out/'
         /
         &system
            ibrav = 14,
            celldm(1) = 16.0,
            celldm(2) = 1.0,
            celldm(3) = 0.5,
            celldm(4) = 0.0,
            celldm(5) = 0.0,
            celldm(6) = 0.0,
            nat = 12,
            ntyp = 2,
            nbnd = 15,
            ecutwfc = 40.0,
            nr1b= 10, nr2b = 10, nr3b = 10,
            input_dft = 'BLYP'
         /
         &electrons
            emass = 400.d0,
            emass_cutoff = 2.5d0,
            electron_dynamics = 'sd'
         /
         &ions
            ion_dynamics = 'none'
         /
         &cell
            cell_dynamics = 'none',
            press = 0.0d0,
          /
          ATOMIC_SPECIES
          C 12.0d0 c_blyp_gia.pp
          H 1.00d0 h.ps
          ATOMIC_POSITIONS (bohr)
          C     2.6 0.0 0.0
          C     1.3 -1.3 0.0
          C    -1.3 -1.3 0.0
          C    -2.6 0.0 0.0
          C    -1.3 1.3 0.0
          C     1.3 1.3 0.0
          H     4.4 0.0 0.0
          H     2.2 -2.2 0.0
          H    -2.2 -2.2 0.0
          H    -4.4 0.0 0.0
          H    -2.2 2.2 0.0
          H     2.2 2.2 0.0
\end{verbatim} 
You can find the description of the input variables in file 
\texttt{Doc/INPUT\_CP.*}.
 
\subsection{Reaching the electronic ground state}

The first run, when starting from scratch, is always an electronic 
minimization, with fixed ions and cell, to bring the electronic system on the ground state (GS) relative to the starting atomic configuration. This step is conceptually very similar to
self-consistency in a \pwx\ run.

Sometimes a single run is not enough to reach the GS. In this case,
you need to re-run the electronic minimization stage. Use the input 
of the first run, changing \texttt{restart\_mode = 'from\_scratch'}
to \texttt{restart\_mode = 'restart'}.
   
NOTA BENE: Unless you are already experienced with the system 
you are studying or with the internals of the code, you will usually need 
to tune some input parameters, like \texttt{emass}, \texttt{dt}, and cut-offs. For this 
purpose, a few trial runs could be useful: you can perform short
minimizations (say, 10 steps) changing and adjusting these parameters 
to fit your needs. You can specify the degree of convergence with these
two thresholds:
\begin{quote}
\texttt{etot\_conv\_thr}: total energy difference between two consecutive steps\\
\texttt{ekin\_conv\_thr}: value of the fictitious kinetic energy of the electrons.
\end{quote}
   
Usually we consider the system on the GS when 
\texttt{ekin\_conv\_thr} $ < 10^{-5}$.
You could check the value of the fictitious kinetic energy on the standard 
output (column EKINC).

Different strategies are available to minimize electrons, but the most used 
ones are:
\begin{itemize}
\item steepest descent: \texttt{electron\_dynamics = 'sd'}
\item damped dynamics: \texttt{electron\_dynamics = 'damp'},
\texttt{electron\_damping} = a number typically ranging from 0.1 and 0.5 
\end{itemize}
See the input description to compute the optimal damping factor.

\subsection{Relax the system}

Once your system is in the GS, depending on how you have prepared the starting
atomic configuration:
\begin{enumerate}
\item
if you have set the atomic positions "by hand" and/or from a classical code, 
check the forces on atoms, and if they are large ($\sim 0.1 \div 1.0$
atomic units), you should perform an ionic minimization, otherwise the
system could break up during the dynamics.
\item
if you have taken the positions from a previous run or a previous ab-initio 
simulation, check the forces, and if they are too small ($\sim 10^{-4}$ 
atomic units), this means that atoms are already in equilibrium positions 
and, even if left free, they will not move. Then you need to randomize 
positions a little bit (see below).
\end{enumerate}

Let us consider case 1). There are 
different strategies to relax the system, but the most used 
are again steepest-descent or damped-dynamics for ions and electrons. 
You could also mix electronic and ionic minimization scheme freely, 
i.e. ions in steepest-descent and electron in with damped-dynamics or vice versa.
\begin{itemize}    
\item[(a)] suppose we want to perform steepest-descent for ions. Then we should specify 
the following section for ions:
\begin{verbatim} 
         &ions
           ion_dynamics = 'sd'
         /
\end{verbatim} 
Change also the ionic masses to accelerate the minimization:
\begin{verbatim} 
         ATOMIC_SPECIES
          C 2.0d0 c_blyp_gia.pp
          H 2.00d0 h.ps
\end{verbatim} 
while leaving other input parameters unchanged.
{\em Note} that if the forces are really high ($> 1.0$ atomic units), you
should always use steepest descent for the first ($\sim 100$
relaxation steps. 
\item[(b)] As the system approaches the equilibrium positions, the steepest 
descent scheme slows down, so is better to switch to damped dynamics:
\begin{verbatim} 
         &ions
           ion_dynamics = 'damp',
           ion_damping = 0.2,
           ion_velocities = 'zero'
         /
\end{verbatim}
A  value of \texttt{ion\_damping} around 0.05 is good for many systems. 
It is also better to specify to restart with zero ionic and electronic 
velocities, since we have changed the masses.
    
Change further the ionic masses to accelerate the minimization:
\begin{verbatim} 
           ATOMIC_SPECIES
           C 0.1d0 c_blyp_gia.pp
           H 0.1d0 h.ps
\end{verbatim}
\item[(c)] when the system is really close to the equilibrium, the damped dynamics 
slow down too, especially because, since we are moving electron and ions 
together, the ionic forces are not properly correct, then it is often better 
to perform a ionic step every N electronic steps, or to move ions only when
electron are in their GS (within the chosen threshold).
    
This can be specified by adding, in the ionic section, the 
\texttt{ion\_nstepe}
parameter, then the \&IONS namelist become as follows:
\begin{verbatim} 
         &ions
           ion_dynamics = 'damp',
           ion_damping = 0.2,
           ion_velocities = 'zero',
           ion_nstepe = 10
         /
\end{verbatim}
Then we specify in the \&CONTROL namelist:
\begin{verbatim} 
           etot_conv_thr = 1.d-6,
           ekin_conv_thr = 1.d-5,
           forc_conv_thr = 1.d-3
\end{verbatim}
As a result, the code checks every 10 electronic steps whether
the electronic system satisfies the two thresholds 
\texttt{etot\_conv\_thr}, \texttt{ekin\_conv\_thr}: if it does, 
the ions are advanced by one step.
The process thus continues until the forces become smaller than
\texttt{forc\_conv\_thr}.

{\em Note} that to fully relax the system you need many runs, and different 
strategies, that you should mix and change in order to speed-up the convergence.
The process is not automatic, but is strongly based on experience, and trial 
and error.

Remember also that the convergence to the equilibrium positions depends on 
the energy threshold for the electronic GS, in fact correct forces (required
to move ions toward the minimum) are obtained only when electrons are in their 
GS. Then a small threshold on forces could not be satisfied, if you do not 
require an even smaller threshold on total energy.
\end{itemize}

Let us now move to case 2: randomization of positions.
   
If you have relaxed the system or if the starting system is already in
the equilibrium positions, then you need to displace ions from the equilibrium 
positions, otherwise they will not move in a dynamics simulation.
After the randomization you should bring electrons on the GS again,
in order to start a dynamic with the correct forces and with electrons 
in the GS. Then you should switch off the ionic dynamics and activate 
the randomization for each species, specifying the amplitude of the 
randomization itself. This could be done with the following 
\&IONS namelist:
\begin{verbatim}
          &ions
            ion_dynamics = 'none',
            tranp(1) = .TRUE.,
            tranp(2) = .TRUE.,
            amprp(1) = 0.01
            amprp(2) = 0.01
          /
\end{verbatim}
In this way a random displacement (of max 0.01 a.u.) is added to atoms of 
species 1 and 2. All other input parameters could remain the same.
Note that the difference in the total energy (etot) between relaxed and
randomized positions can be used to estimate the temperature that will
be reached by the system. In fact, starting with zero ionic velocities,
all the difference is potential energy, but in a dynamics simulation, the
energy will be equipartitioned between kinetic and potential, then to
estimate the temperature take the difference in energy (de), convert it
in Kelvin, divide for the number of atoms and multiply by 2/3.
Randomization could be useful also while we are relaxing the system,
especially when we suspect that the ions are in a local minimum or in
an energy plateau.

\subsection{CP dynamics}

At this point after having minimized the electrons, and with ions displaced from their equilibrium positions, we are ready to start a CP
dynamics. We need to specify \texttt{'verlet'} both in ionic and electronic
dynamics. The threshold in control input section will be ignored, like
any parameter related to minimization strategy. The first time we perform 
a CP run after a minimization, it is always better to put velocities equal
to zero, unless we have velocities, from a previous simulation, to
specify in the input file. Restore the proper masses for the ions. In this
way we will sample the microcanonical ensemble. The input section
changes as follow:
\begin{verbatim}
           &electrons
              emass = 400.d0,
              emass_cutoff = 2.5d0,
              electron_dynamics = 'verlet',
              electron_velocities = 'zero'
           /
           &ions
              ion_dynamics = 'verlet',
              ion_velocities = 'zero'
           /
           ATOMIC_SPECIES
           C 12.0d0 c_blyp_gia.pp
           H 1.00d0 h.ps
\end{verbatim}

If you want to specify the initial velocities for ions, you have to set
\texttt{ion\_velocities ='from\_input'}, and add the IONIC\_VELOCITIES
card, after the ATOMIC\_POSITION card, with the list of velocities in 
atomic units.

NOTA BENE: in restarting the dynamics after the first CP run,
remember to remove or comment the velocities parameters:
\begin{verbatim}
           &electrons
              emass = 400.d0,
              emass_cutoff = 2.5d0,
              electron_dynamics = 'verlet'
              ! electron_velocities = 'zero'
           /
           &ions
              ion_dynamics = 'verlet'
              ! ion_velocities = 'zero'
           /
\end{verbatim}
otherwise you will quench the system interrupting the sampling of the
microcanonical ensemble.

\paragraph{ Varying the temperature }
   
It is possible to change the temperature of the system or to sample the 
canonical ensemble fixing the average temperature, this is done using 
the Nos\'e thermostat. To activate this thermostat for ions you have 
to specify in namelist \&IONS:
\begin{verbatim}
           &ions
              ion_dynamics = 'verlet',
              ion_temperature = 'nose',
              fnosep = 60.0,
              tempw = 300.0
           /  
\end{verbatim}
where \texttt{fnosep} is the frequency of the thermostat in THz, that should be
chosen to be comparable with the center of the vibrational spectrum of
the system, in order to excite as many vibrational modes as possible.
\texttt{tempw} is the desired average temperature in Kelvin.
   
{\em Note:} to avoid a strong coupling between the Nos\'e thermostat 
and the system, proceed step by step. Don't switch on the thermostat 
from a completely relaxed configuration: adding a random displacement
is strongly recommended. Check which is the average temperature via a
few steps of a microcanonical simulation. Don't increase the temperature
too much. Finally switch on the thermostat. In the case of molecular system,
different modes have to be thermalized: it is better to use a chain of 
thermostat or equivalently running different simulations with different 
frequencies. 
 
\paragraph{ No\'se thermostat for electrons }

It is possible to specify also the thermostat for the electrons. This is
usually activated in metals or in systems where we have a transfer of
energy between ionic and electronic degrees of freedom. Beware: the
usage of electronic thermostats is quite delicate. The following information 
comes from K. Kudin: 

''The main issue is that there is usually some "natural" fictitious kinetic 
energy that electrons gain from the ionic motion ("drag"). One could easily 
quantify how much of the fictitious energy comes from this drag by doing a CP 
run, then a couple of CG (same as BO) steps, and then going back to CP.
The fictitious electronic energy at the last CP restart will be purely 
due to the drag effect.''

''The thermostat on electrons will either try to overexcite the otherwise 
"cold" electrons, or it will try to take them down to an unnaturally cold 
state where their fictitious kinetic energy is even below what would be 
just due pure drag. Neither of this is good.''

''I think the only workable regime with an electronic thermostat is a 
mild overexcitation of the electrons, however, to do this one will need 
to know rather precisely what is the fictitious kinetic energy due to the
drag.''


\subsection{Advanced usage}

\subsubsection{ Self-interaction Correction }

The self-interaction correction (SIC) included in the \CP\
package is based
on the Constrained Local-Spin-Density approach proposed my F. Mauri and 
coworkers (M. D'Avezac et al. PRB 71, 205210 (2005)). It was used for
the first time in \qe\ by F. Baletto, C. Cavazzoni 
and S.Scandolo (PRL 95, 176801 (2005)).

This approach is a simple and nice way to treat ONE, and only one, 
excess charge. It is moreover necessary to check a priori that 
the spin-up and spin-down eigenvalues are not too different, for the 
corresponding neutral system, working in the Local-Spin-Density 
Approximation (setting \texttt{nspin = 2}). If these two conditions are satisfied
and you are interest in charged systems, you can apply the SIC.
This approach is a on-the-fly method to correct the self-interaction 
with the excess charge with itself.

Briefly, both the Hartree and the XC part have been 
corrected to avoid the interaction of the excess charge with itself.

For example, for the Boron atoms, where we have an even number of 
electrons (valence electrons = 3), the parameters for working with
the SIC are:
\begin{verbatim}
           &system
           nbnd= 2,
           tot_magnetization=1,
           sic_alpha = 1.d0,
           sic_epsilon = 1.0d0,
           sic = 'sic_mac',
           force_pairing = .true.,
\end{verbatim}
The two main parameters are:
\begin{quote}
\texttt{force\_pairing = .true.}, which forces the paired electrons to be the same;\\ 
\texttt{sic='sic\_mac'}, which instructs the code to use Mauri's correction.
\end{quote}

{\bf Warning}: 
This approach has known problems for dissociation mechanism
driven by excess electrons.

Comment 1:
Two parameters, \texttt{sic\_alpha} and \texttt{sic\_epsilon'}, have been introduced 
following the suggestion of M. Sprik (ICR(05)) to treat the radical
(OH)-H$_2$O. In any case, a complete ab-initio approach is followed 
using \texttt{sic\_alpha=1}, \texttt{sic\_epsilon=1}.

Comment 2:
When you apply this SIC scheme to a molecule or to an atom, which are neutral,
remember to add the correction to the energy level as proposed by Landau: 
in a neutral system, subtracting the self-interaction, the unpaired electron
feels a charged system, even if using a compensating positive background. 
For a cubic box, the correction term due to the Madelung energy is approx. 
given by $1.4186/L_{box} - 1.047/(L_{box})^3$, where $L_{box}$ is the 
linear dimension of your box (=celldm(1)). The Madelung coefficient is 
taken from I. Dabo et al. PRB 77, 115139 (2007).
(info by F. Baletto, francesca.baletto@kcl.ac.uk)

% \subsubsection{ Variable-cell MD }

%The variable-cell MD is when the Car-Parrinello technique is also applied 
%to the cell. This technique is useful to study system at very high pressure.

\subsubsection{ ensemble-DFT }

The ensemble-DFT (eDFT) is a robust method to simulate the metals in the 
framework of ''ab-initio'' molecular dynamics. It was introduced in 1997 
by Marzari et al.

The specific subroutines for the eDFT are in 
\texttt{CPV/src/ensemble\_dft.f90} where you 
define all the quantities of interest. The subroutine 
\texttt{CPV/src/inner\_loop\_cold.f90}
called by \texttt{cg\_sub.f90}, control the inner loop, and so the minimization of 
the free energy $A$ with respect to the occupation matrix.

To select a eDFT calculations, the user has to set:
\begin{verbatim}
            calculation = 'cp'
            occupations= 'ensemble' 
            tcg = .true.
            passop= 0.3
            maxiter = 250
\end{verbatim}
to use the CG procedure. In the eDFT it is also the outer loop, where the
energy is minimized with respect to the wavefunction keeping fixed the 
occupation matrix. While the specific parameters for the inner loop.
Since eDFT was born to treat metals, keep in mind that we want to describe 
the broadening of the occupations around the Fermi energy.
Below the new parameters in the electrons list, are listed.
\begin{itemize}
\item \texttt{smearing}: used to select the occupation distribution;
there are two options: Fermi-Dirac smearing='fd', cold-smearing
smearing='cs' (recommended) 
\item \texttt{degauss}: is the electronic temperature; it controls the broadening
of the occupation numbers around the Fermi energy. 
\item \texttt{ninner}: is the number of iterative cycles in the inner loop, 
done to minimize the free energy $A$ with respect the occupation numbers.
The typical range is 2-8.
\item \texttt{conv\_thr}: is the threshold value to stop the search of the 'minimum' 
free energy.
\item \texttt{niter\_cold\_restart}: controls the frequency at which a full iterative
inner cycle is done. It is in the range $1\div$\texttt{ninner}. It is a trick to speed up 
the calculation.
\item \texttt{lambda\_cold}: is the length step along the search line for the best 
value for $A$, when the iterative cycle is not performed. The value is close 
to 0.03, smaller for large and complicated metallic systems.
\end{itemize}
{\em NOTE:} \texttt{degauss} is in Hartree, while in \PWscf is in Ry (!!!). 
The typical range is 0.01-0.02 Ha.

The input for an Al surface is:
\begin{verbatim}
            &CONTROL
             calculation = 'cp',
             restart_mode = 'from_scratch',
             nstep  = 10,
             iprint = 5,
             isave  = 5,
             dt    = 125.0d0,
             prefix = 'Aluminum_surface',
             pseudo_dir = '~/UPF/',
             outdir = '/scratch/'
             ndr=50
             ndw=51
            /
            &SYSTEM
             ibrav=  14,
             celldm(1)= 21.694d0, celldm(2)= 1.00D0, celldm(3)= 2.121D0,
             celldm(4)= 0.0d0,   celldm(5)= 0.0d0, celldm(6)= 0.0d0,
             nat= 96,
             ntyp= 1,
             nspin=1,
             ecutwfc= 15,
             nbnd=160,
             input_dft = 'pbe'
             occupations= 'ensemble',
             smearing='cs',
             degauss=0.018,
            /
            &ELECTRONS
             orthogonalization = 'Gram-Schmidt',
             startingwfc = 'random',
             ampre = 0.02,
             tcg = .true.,
             passop= 0.3,
             maxiter = 250,
             emass_cutoff = 3.00,
             conv_thr=1.d-6
             n_inner = 2,
             lambda_cold = 0.03,
             niter_cold_restart = 2,
            /
            &IONS
             ion_dynamics  = 'verlet',
             ion_temperature = 'nose'
             fnosep = 4.0d0,
             tempw = 500.d0
            /
            ATOMIC_SPECIES
             Al 26.89 Al.pbe.UPF
\end{verbatim}
{\em NOTA1}  remember that the time step is to integrate the ionic dynamics,
so you can choose something in the range of 1-5 fs. \\
{\em NOTA2} with eDFT you are simulating metals or systems for which the 
occupation number is also fractional, so the number of band, \texttt{nbnd}, has to 
be chosen such as to have some empty states. As a rule of thumb, start
with an initial occupation number of about 1.6-1.8 (the more bands you 
consider, the more the calculation is accurate, but it also takes longer.
The CPU time scales almost linearly with the number of bands.) \\
{\em NOTA3} the parameter \texttt{emass\_cutoff} is used in the preconditioning 
and it has a completely different meaning with respect to plain CP. 
It ranges between 4 and 7.

All the other parameters have the same meaning in the usual \CP\ input, 
and they are discussed above.

\subsubsection{Free-energy surface calculations}
Once \texttt{CP} is patched with \texttt{PLUMED} plug-in, it becomes possible to turn-on most of the PLUMED functionalities
running \texttt{CP} as: \texttt{./cp.x -plumed} plus the other usual \texttt{CP} arguments. The PLUMED input file has to be located in the specified \texttt{outdir} with
the fixed name \texttt{plumed.dat}.


\subsubsection{Treatment of USPPs}

The cutoff \texttt{ecutrho} defines the resolution on the real space FFT mesh (as expressed 
by \texttt{nr1}, \texttt{nr2} and \texttt{nr3}, that the code left on its own sets automatically).
In the USPP case we refer to this mesh as the "hard" mesh, since it 
is denser than the smooth mesh that is needed to represent the square 
of the non-norm-conserving wavefunctions.
  
On this "hard", fine-spaced mesh, you need to determine the size of the
cube that will encompass the largest of the augmentation charges - this
is what \texttt{nr1b}, \texttt{nr2b}, \texttt{nr3b} are. hey are independent 
of the system size, but dependent on the size
of the augmentation charge (an atomic property that doesn't vary 
that much for different systems) and on the
real-space resolution needed by augmentation charges (rule of thumb:
\texttt{ecutrho} is between 6 and 12 times \texttt{ecutwfc}).

The small boxes should be set as small as possible, but large enough
to contain the core of the largest element in your system.
The formula for estimating the box size is quite simple: 
\begin{quote}
   \texttt{nr1b} = $2 R_c / L_x \times$ \texttt{nr1}
\end{quote}
and the like, where $R_{cut}$ is largest cut-off radius among the various atom
types present in the system, $L_x$ is the
physical length of your box along the $x$ axis. You have to round your
result to the nearest larger integer.
In practice, \texttt{nr1b} etc. are often in the region of 20-24-28; testing seems
again a necessity.

The core charge is in principle finite only at the core region (as defined
by some $R_{rcut}$ ) and vanishes out side the core. Numerically the charge is
represented in a Fourier series which may give rise to small charge
oscillations outside the core and even to negative charge density, but
only if the cut-off is too low. Having these small boxes removes the
charge oscillations problem (at least outside the box) and also offers
some numerical advantages in going to higher cut-offs." (info by Nicola Marzari)

\subsubsection{Hybrid functional calculations using maximally localized Wannier functions}
In this section, we illustrate some guidelines to perform exact exchange (EXX) calculations using Wannier functions efficiently. 

The references for this algorithm are:
\begin{itemize}
  \item[(i)] Theory: X. Wu , A. Selloni, and R. Car, Phys. Rev. B 79, 085102 (2009).
  \item[(ii)] Implementation: H.-Y. Ko, B. Santra, R. A. DiStasio, L. Kong, Z. Li, X. Wu, and R. Car, arxiv.
\end{itemize}

The parallelization scheme in this algorithm is based upon the number of electronic states. 
In the current implementation, there are certain restrictions on the choice of the number of MPI tasks.
Also slightly different algorithms are employed depending on whether the number of MPI tasks used in the calculation are greater or less than the number of electronic states.
We highly recommend users to follow the notes below.
This algorithm can be used most efficiently if the numbers of electronic states are uniformly distributed over the number of MPI tasks. 
For a system having N electronic states the optimum numbers of MPI tasks (nproc) are the following:

\begin{itemize}
  \item[(a)] In case of nproc $\leq$ N, the optimum choices are N/m, where m is any positive integer.
    \begin{itemize}
      \item Robustness:  Can be used for odd and even number of electronic states.
      \item OpenMP threads:  Can be used.
      \item Taskgroup:  Only the default value of the task group (-ntg 1) is allowed.
    \end{itemize}
  \item[(b)] In case of nproc  $>$ N, the optimum choices are N*m, where m is any positive integer.
    \begin{itemize}
      \item Robustness:  Can be used for even number of electronic states.
      \item Largest value of m:  As long as nj\_max (see output) is greater than 1, 
        however beyond m=8 the scaling may become poor. The scaling should be tested by users.
      \item OpenMP threads: Can be used and highly recommended. We have tested number of threads
        starting from 2 up to 64. More threads are also allowed. 
        For very large calculations (nproc $>$ 1000 ) efficiency can largely depend on the computer 
        architecture and the balance between the MPI tasks and the OpenMP threads. 
        User should test for an optimal balance. Reasonably good scaling can be achieved by using 
        m=6-8 and OpenMP threads=2-16.
      \item Taskgroup:  Can be greater than 1 and users should choose the largest possible value
        for ntg. To estimate ntg, find the value of nr3x in the output and compute nproc/nr3x and
        take the integer value. We have tested the value of ntg as $2^m$, where m is any positive integer. 
        Other values of ntg should be used with caution.
      \item Ndiag:  Use -ndiag X option in the execution of cp.x. Without this option jobs
        may crash on certain architectures. Set X to any perfect square number which is equal to or less than N.
    \end{itemize}
  \item DEBUG:  The EXX calculations always work when number of MPI tasks = number of electronic states.
    In case of any uncertainty, the EXX energy computed using different numbers of MPI tasks can be
    checked by performing test calculations using number of MPI tasks = number of electronic states.
\end{itemize}

An example input is listed as following:
\begin{verbatim}
&CONTROL
  calculation       = 'cp-wf',
  title             = "(H2O)32 Molecule: electron minimization PBE0",
  restart_mode      = "from_scratch",
  pseudo_dir        = './',
  outdir            = './',
  prefix            = "water",
  nstep             = 220,
  iprint            = 100,
  isave             = 100,
  dt                = 4.D0,
  ekin_conv_thr     = 1.D-5,
  etot_conv_thr     = 1.D-5,
/
&SYSTEM
  ibrav             = 1,
  celldm(1)         = 18.6655, 
  nat               = 96,
  ntyp              = 2,
  ecutwfc           = 85.D0,
  input_dft         = 'pbe0',
/
&ELECTRONS
  emass             = 400.D0,
  emass_cutoff      = 3.D0,
  ortho_eps         = 1.D-8,
  ortho_max         = 300,
  electron_dynamics = "damp",
  electron_damping  = 0.1D0,
/
&IONS
  ion_dynamics      = "none", 
/
&WANNIER
  nit               = 60,
  calwf             = 3,
  tolw              = 1.D-6,
  nsteps            = 20,
  adapt             = .FALSE.
  wfdt              = 4.D0,
  wf_q              = 500,
  wf_friction       = 0.3D0,
  exx_neigh         = 60,     ! exx related optional
  exx_dis_cutoff    = 8.0D0,  ! exx related optional
  exx_ps_rcut_self  = 6.0D0,  ! exx related optional
  exx_ps_rcut_pair  = 5.0D0,  ! exx related optional
  exx_me_rcut_self  = 9.3D0,  ! exx related optional
  exx_me_rcut_pair  = 7.0D0,  ! exx related optional
  exx_poisson_eps   = 1.D-6,  ! exx related optional
/
ATOMIC_SPECIES
O 16.0D0 O_HSCV_PBE-1.0.UPF
H  2.0D0 H_HSCV_PBE-1.0.UPF 
\end{verbatim}


\section{Performances}

% \subsection{Execution time}

% \subsection{Memory requirements}

% \subsection{File space requirements}

% \subsection{Parallelization issues}
% \label{SubSec:badpara}

\cpx\ can run in principle on any number of processors.
The effectiveness of parallelization is ultimately judged by the 
''scaling'', i.e. how the time needed to perform a job scales
 with the number of processors, and depends upon:
\begin{itemize}
\item the size and type of the system under study;
\item the judicious choice of the various levels of parallelization 
(detailed in Sec.\ref{SubSec:para});
\item the availability of fast interprocess communications (or lack of it).
\end{itemize}
Ideally one would like to have linear scaling, i.e. $T \sim T_0/N_p$ for 
$N_p$ processors, where $T_0$ is the estimated time for serial execution.
 In addition, one would like to have linear scaling of
the RAM per processor: $O_N \sim O_0/N_p$, so that large-memory systems
fit into the RAM of each processor.

As a general rule, image parallelization:
\begin{itemize}
\item  may give good scaling, but the slowest image will determine
the overall performances (''load balancing'' may be a problem);
\item requires very little communications (suitable for ethernet 
communications);
\item does not reduce the required memory per processor (unsuitable for 
large-memory jobs).
\end{itemize}
Parallelization on k-points:
\begin{itemize}
\item guarantees (almost) linear scaling if the number of k-points
is a multiple of the number of pools;
\item requires little communications (suitable for ethernet communications);
\item does not reduce the required memory per processor (unsuitable for 
large-memory jobs).
\end{itemize}
Parallelization on PWs:
\begin{itemize}
\item yields good to very good scaling, especially if the number of processors
in a pool is a divisor of $N_3$ and $N_{r3}$ (the dimensions along the z-axis 
of the FFT grids, \texttt{nr3} and \texttt{nr3s}, which coincide for NCPPs);
\item requires heavy communications (suitable for Gigabit ethernet up to 
4, 8 CPUs at most, specialized communication hardware needed for 8 or more
processors );
\item yields almost linear reduction of memory per processor with the number
of processors in the pool.
\end{itemize}

A note on scaling: optimal serial performances are achieved when the data are
as much as possible kept into the cache. As a side effect, PW
parallelization may yield superlinear (better than linear) scaling,
thanks to the increase in serial speed coming from the reduction of data size 
(making it easier for the machine to keep data in the cache).

VERY IMPORTANT: For each system there is an optimal range of number of processors on which to 
run the job.  A too large number of processors will yield performance 
degradation. If the size of pools is especially delicate: $N_p$ should not 
exceed $N_3$ and $N_{r3}$, and should ideally be no larger than
$1/2\div1/4 N_3$ and/or $N_{r3}$. In order to increase scalability,
it is often convenient to 
further subdivide a pool of processors into ''task groups''.
When the number of processors exceeds the number of FFT planes, 
data can be redistributed to "task groups" so that each group 
can process several wavefunctions at the same time.

The optimal number of processors for "linear-algebra"
parallelization, taking care of multiplication and diagonalization 
of $M\times M$ matrices, should be determined by observing the
performances of \texttt{cdiagh/rdiagh} (\pwx) or \texttt{ortho} (\cpx)
for different numbers of processors in the linear-algebra group
(must be a square integer).

Actual parallel performances will also depend on the available software 
(MPI libraries) and on the available communication hardware. For
PC clusters, OpenMPI (\texttt{http://www.openmpi.org/}) seems to yield better 
performances than other implementations (info by Kostantin Kudin). 
Note however that you need a decent communication hardware (at least 
Gigabit ethernet) in order to have acceptable performances with 
PW parallelization. Do not expect good scaling with cheap hardware: 
PW calculations are by no means an "embarrassing parallel" problem.
   
Also note that multiprocessor motherboards for Intel Pentium CPUs typically 
have just one memory bus for all processors. This dramatically
slows down any code doing massive access to memory (as most codes 
in the \qe\ distribution do) that runs on processors of the same
motherboard.

\end{document}
