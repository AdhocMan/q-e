\documentstyle[12pt]{article}
\pagestyle{empty}
\textwidth = 15.5 cm
\textheight = 23.5 cm
\topmargin =-1.0 cm
\oddsidemargin = 0.5 cm
\listparindent=0pt
\itemsep=5pt
\def\r{{\bf r}}
\begin{document} 
\title{Notes on pseudopotential generation}
\author{\em Paolo Giannozzi, DEMOCRITOS and Scuola Normale Superiore di Pisa\\
e-mail: {\tt giannozz@nest.sns.it}\\
URL: {\tt http://www.sns.it/\~{}giannozz}}
\maketitle

\section{Introduction} 

When I started to do my first first-principle calculation
(that is, my first$^2$-principle calculation) with S. Baroni
on CsI under pressure (1985), it became quickly evident that
available pseudopotentials (PP's) couldn't do the job. So we 
generated our own PP's. Since that first experience I have 
generated a large number of PP's and people keep asking me 
new PP's from time to time. I am happy that "my" PP's are 
appreciated and used by other people. I don't think however 
that the generation of PP's is such a hard task that it requires 
an official (or unofficial) PP wizard to do this. For this reason 
I want to share here my (little) experience.

These notes were originally written having in mind my version
of the PP generation code and of the various related utilities
(still available on the web from my home page, but no longer 
maintained). I am in the process of adapting them to cover the
extended capabilities of the improved version {\tt atomic}, 
packaged with PWscf ({\tt http://www.pwscf.org}). If you remark 
any inconsistencies, please let me know.

The {\tt atomic} code, written in large part by A. Dal Corso
(Democritos and Sissa Trieste) can generate both 
Norm-Conserving (NC) \cite{NC} and Ultrasoft (US) \cite{van} PP's.
It allows for multiple projectors, full relativistic calculations,
spin-split PP's for spin-orbit calculations.

\subsection{Who needs to generate a pseudopotential?}

There are at least three well-known published sets of NC-PP's:
those of Bachelet, Hamann, and Schl\"uter \cite{BHS},
those of Gonze, Stumpf, and Scheffler \cite{Gonze}, and
those of Goedecker, Teter, and Hutter \cite{Goedecker}. 
Moreover, all major packages for electronic-structure calculations
include a downloadable table of PP's. One could then wonder 
what a PP generation code is useful for. The problem is that 
sometimes published PP's will not suit your needs. For instance,
you may need:
\begin{enumerate}
 \item[--] a better accuracy than what is provided 
     by PP's in the published PP tables;
 \item[--] PP's generated with an exchange-correlation
     functional that is not available in the tables;
 \item[--] a distinction between "valence" and "core"
     electrons that is different from the one chosen in the tables.
 \item[--] ``softer'' PP's than those in the tables.
 \item[--] PP's with a core hole for calculations of
     XPS or NEXAFS
 \item[--] all-electron and pseudo atomic orbitats for
     all-electron wavefunction reconstruction
 \item[--] ...
\end{enumerate}
...or you may simply need to know how to produce a PP!

\subsection{About similar work}

There are other PP generation packages available on-line.
Those I am aware of include:
\begin{itemize}
\item the code by Jos\'e-Lu{\'\i}s Martins {\em et al.}\cite{TM}:\\
{\tt http://bohr.inesc-mn.pt/\~{}jlm/pseudo.html}
\item the {\tt fhi98PP} package\cite{fhi98PP}:\\
{\tt http://www.fhi-berlin.mpg.de/th/fhi98md/fhi98PP}
\item the OPIUM code by Andrew Rappe {\em et al.}\cite{RRKJ}:\\
{\tt http://opium.sourceforge.net/}
\item David Vanderbilt's US-PP package \cite{van}:\\
{\tt http://www.physics.rutgers.edu/\~{}dhv/uspp/index.html}.
\end{itemize}
Other codes may be available upon request from the authors 
(I have heard of codes by Hamann and by Bachelet).

Years ago, it occurred to me that a web-based PP generation
tool would have been nice. Being too lazy and too ignorant 
in web-based applications, I did nothing. 
I recently discovered that Miguel Marques {\em et al.} have
implemented something like this: see
{\tt http://www.tddft.org/programs/octopus/pseudo.php}.

\section{Pseudopotential generation, in general} 

In the following I am assuming that the basic PP theory 
is known to the reader. Otherwise, see 
Refs.\cite{NC,BHS,TM,fhi98PP,RRKJ} and references quoted 
therein for NC-PP's; Refs.\cite{van,PAW} for US-PP's. 
I am also assuming that the generated PP's are to be used
with a plane-wave (PW) basis set.

The PP generation is a three-step process. First, one generates
atomic levels and wavefunctions with Density-functional theory (DFT). 
Second, from atomic results one generates the PP. Third, one checks 
whether what he got is actually working. If not, one tries again in 
a different way.

The first step is invariably done assuming a spherically symmetric
self-consistent Hamiltonian, so that all elementary quantum mechanics 
results for the atom apply. The atomic state is defined by the
"electronic configuration", one-electron states are defined by a
principal quantum number and by the angular momentum and are obtained
by solving a self-consistent radial Schr\"odinger-like (Kohn-Sham)
equation.

The second step exists in many variants. One can generate ``traditional'' 
single-projector NC-PP's; multiple-projector NC-PP's; or US-PP's.
In the following we will consider mostly the case of ``traditional'' 
NC-PP's. The crucial step is the generation of smooth, nodeless
``pseudo-orbitals'' from atomic all-electron orbitals. Two popular 
methods are presently implemented: Troullier-Martins \cite{TM}
and Rappe-Rabe-Kaxiras-Joannopoulos \cite{RRKJ} (RRKJ).

The third step is closer to cooking than to science. There is a
large arbitrariness in the preceding step that one would like to 
exploit in order to get the "best" PP, but there is no well-defined
way to do this. Moreover one is often forced to strike a compromise
between accuracy and computer time. This step is the main focus of
these notes.

\section{Step-by-step Pseudopotential generation} 

If you want to generate a PP for a given atom, the checklist is the
following:

\begin{enumerate}
\item choose the exchange-correlation functional
\item choose the electronic reference configuration
\item choose the valence-core partition
\item choose which reference states to pseudize, and at which energies
\item choose the type of pseudization
\item choose the matching radii
\item choose the parameters for the ``nonlinear core correction''
\item choose the local potential
\item generate the pseudopotential
\item check its transferability
\item check the required cutoff
\end{enumerate}

\subsection{Choosing the exchange-correlation functional}

A large number of exchange-correlation functionals, both 
in the Local-Density Approximation (LDA) or in the Generalized
Gradient Approximation (GGA), are implemented.
Most of them have been extensively tested, but beware: 
some exotic or seldom-used functionals might contain bugs.

PP's must be generated with the SAME functional that will
be later used in calculations. The use of, for instance,
GGA functionals with LDA PP's is inconsistent. This is why
the PP contain information on the DFT level used in their
generation: if you or your code ignore it, you do it at your
own risk.

NOTE THAT GGA functionals may present numerical problems
when the charge density goes to zero. For instance, the Becke
gradient correction to the exchange may diverge for 
$\rho \rightarrow 0$. This does not happen in a free atom
if the charge density behaves as it should, that is, as
$\rho(r)\rightarrow exp(-\alpha r)$ for $r \rightarrow \infty$.
However in a pseudoatom a weird behavior may arise 
around the core region, $r\rightarrow 0$, because the 
pseudocharge in that region is very small or sometimes 
vanishing (if there are no filled $s$ states). As a consequence,
nasty-looking ``spikes'' appear in the unscreened pseudopotential
very close to the nucleus. This is not nice at all but it is
usually harmless, because the interested region is really 
very small. However in some unfortunate cases there can be 
convergence problems. If you do not want to see those horrible 
spikes, or if you experience problems, you have the following
choices:
\begin{enumerate}
\item[--] Use a better-behaved GGA, such as PBE
\item[--] Use the ``nonlinear core correction'' (see below; 
it ensures the presence of some charge close to the nucleus).
% \item[--] cut out the gradient correction for small $r$
% (set variable {\tt rcut} to $\sim 0.001$ or so).
\end{enumerate}


\subsection{Choosing the electronic reference configuration}

This may be any reasonable configuration not too far away from
the expected configuration in solids or molecules. As a first
choice, use the atomic ground state. Use the ground state if you 
do not have a good reason to do otherwise, such as for instance:
\begin{enumerate}
\item[--]
   You do not want to deal with unbound states.
   Very often states with highest angular momentum $l$ are not bound
   in the atom (an example: the $3d$ state in Si is not bound on the
   ground state $3s^23p^2$, at least with LDA or GGA). In such a case 
   one has the choice between 
   \begin{enumerate} 
      \item[--] using different configurations for higher and lower $l$, 
                as in Refs.\cite{BHS,Gonze};
      \item[--] choosing a single, more ionic configuration for which 
                all desired states are bound;
      \item[--] generate PP's on unbound states: requires to choose
                a suitable reference energy.
   \end{enumerate}
\item[--]
   The results of your PP are very sensitive to the chosen configuration.
   This is something that in principle should not happen, but
   I have encountered in my career at least one remarkable case
   in which it does. In III-V zincblende semiconductors, the 
   equilibrium lattice parameter is rather sensitive to the form
   of the $d$ potential of the cation (due to the presence of $p-d$ 
   coupling between anion $p$ states and cation $d$ states 
   \cite{Zunger}). By varying the reference configuration, one 
   can change the equilibrium lattice parameter by as much as $1-2\%$. 
   The problem arises if you want to calculate accurate dynamical
   properties of GaAs/AlAs alloys and superlattices: you need to
   get a good theoretical lattice matching between GaAs and AlAs,
   or otherwise unpleasant spurious effects may arise. When I was 
   confronted with this problem, I didn't find any better solution
   than to tweak the $4d$ reference configuration for Ga until I got
   the observed lattice-matching.
\item[--]
   You know that the atom will be in a given configuration in the
   system you are interested in and you try to stay close to it.
   This is not very elegant but sometimes it is useful: in transition
   metals, with semicore states in the core, it is probably better to
   chose a reasonable configuration for $d$ states and not to use it
   for systems with very different $d$ configuration. The problem is that
   the $(n+1)s$ and $(n+1)p$ PP have a hard time in reproducing the true
   potential if the $nd$ state changes a lot with respect to the
   starting configuration (consider this no more than a hand-waiving 
   argument). In Rare-Earth compounds, leaving the $4f$ electrons in the 
   core with the correct occupancy (if known) may be a quick and dirty way 
   to avoid the well-known problems of DFT yielding the wrong occupancy 
   in highly correlated materials.
\item[--]
   You don't manage to build a decent PP with the ground state configuration, 
   for whatever reason.
\end{enumerate}

NOTE 1: you can calculate PP for a $l$ as high as you want, but you
are not obliged to use all of them in PW calculations. The general
rule is that if your atom has states up to $l=l_c$ in the core, you
need a PP with angular momenta up to $l=l_c+1$. Angular momenta
$l>l_c+1$ will feel the same potential as $l=l_c+1$, because
for all of them there is no orthogonalization to core states.
As a consequence a PP should have projectors on angular momenta up to
$l_c$ and $l=l_c+1$ should be the local reference state for PW
calculations. This rule is not very strict and may be relaxed: high
angular momenta are seldom important (but be careful if they are). 
Moreover separable PP pose serious constraints on local reference $l$
(see below) and the choice is sometimes obliged. Note also that the
highest the $l$ in the PP, the more expensive the PW calculation will 
be.

NOTE 2: a completely empty configuration ($s^0p^0d^0$) or
a configuration with fractional occupation numbers are both
perfectly acceptable. Even if fractional occupation numbers do
not correspond to a physical state, they correspond to a
perfectly defined mathematical object.

NOTE 3: if you generate a single-projector PP using a configuration
with semicore states in the valence, remember that for each $l$
only the state with lowest $n$can be used to generate the PP, 
and that the state with same $l$ and higher $n$ MUST BE EMPTY.

NOTE 4: PP could in principle be generated on a spin-polarized
configuration, but a spin-unpolarized one is typically used.
Since PP are constructed to be transferrable, they can describe
spin-polarized configurations as well. The "nonlinear core correction" 
is typically needed if you plan to use PP in spin-polarized (magnetic)
systems.

\subsubsection{Generating all-electron results}

You may now generate all-electron (AE) wavefunctions and one-electron
levels for the reference configuration. This is done by using program
{\tt ld1.x}. You must specify in the input data: atomic symbol, what
you choose as exchange-correlation functional (not needed if you stick 
to LDA), electronic reference configuration.
A complete description of the input is contained in file
{\tt INPUT\_LD1}. If you want accurate AE results for heavy atoms,
you may want to specify a denser grid in $r$-space than the default
one. The defaults one should be good enough for PP generation, though.

Before you proceed, it is a good idea to verify that the atomic data
you just produced actually make sense. Some kind souls have posted on
the web a complete set of reference atomic data :

{\tt http://physics.nist.gov/PhysRefData/DFTdata/ }
\par\noindent
These data have been obtained with the Vosko-Wilk-Nusair functional,
that for the unpolarized case is very similar to the Perdew-Zunger 
functional.

\subsection{Choosing the valence-core partition}

This seems a trivial step, and often it is: valence states are those
that contribute to bonding, core states are those that do not
contribute. Things may sometimes be more complicated than this.
For instance:
\begin{enumerate}
\item[--] in transition metals, whose typical outer electronic
configuration is $nd^i(n+1)s^j(n+1)p^k$ ($n=$main quantum number), 
it is not
always evident that the $ns$ and $np$ states can be safely put into
the core. The problem is that $nd$ states are localized in the same
spatial region as $ns$ and $np$ states, deeper than $(n+1)s$ and
$(n+1)p$ states. This may lead to intolerable loss of transferability.
\item[--] Heavy alkali metals (Rb, Cs, maybe also K) have a large
polarizable core. PP's with just one electron may not work properly
(even with ``nonlinear core correction'', see below) 
\item[--] In some II-VI and III-V semiconductors, such as ZnSe and
GaN, the contribution of the $d$ states of the cation to the bonding 
is not negligible and may require explicit inclusion of those $d$ 
states into the valence.
\end{enumerate}
In all these cases, promoting the highest core states $ns$ and $np$,
or $nd$ (the "semicore" states) into valence may be a computationally
expensive but obliged way to improve poor transferability. It may
happen that the same atom that works great with the "natural" valence
in a solid with weak or metallic bonding does not work well at all in
another compound with a different (stronger) type of bonding. This
happens for many transition and noble metals.
Note that including semicore states into valence could make your
PP harder, will increase the number of electrons, and may require
more than one projector per angular momentum, or lead to slightly
worse results for those cases in which such inclusion is not needed. 
Include semicore states into valence only if it is really needed.

\subsection{Choosing reference states to pseudize, reference energies}

With single-projector PP's (one potential per angular momentum $l$, 
i.e. one projector per $l$ in the separable form), the choice of the 
electronic configuration automatically determines the reference states
to pseudize: for each $l$, the bound valence eigenstate is pseudized
at the corresponding eigenvalue.
It is however possible to generate PP's by pseudizing atomic waves,
i.e. regular solutions of the radial Kohn-Sham equation, at any
energy. More than one such atomic waves at the same $l$ can be 
pseudized, resulting in a PP with more than one projector per $l$. 
This possibility considerably extends the number of ``degrees of 
freedom'' in the generation of a PP. As a rule of thumb: start first 
with one projector per $l$, at the energy of the bound state. For
atoms having semicore states in the valence, an obvious choice is 
to include two projectors, using both bound states. If you are not 
happy, experiment a bit with more projectors, different pseudization 
energies, etc.

\subsection{Choosing the type of pseudization}

Two possible types of pseudization are implemented, both claiming
to yield optimally smooth PP's:
\begin{itemize}
\item Troullier-Martins \cite{TM} (TM) 
\item Rappe-Rabe-Kaxiras-Joannopoulos \cite{RRKJ} (RRKJ).
\end{itemize}
Both pseudizations replace atomic orbitals in the core region 
with smooth nodeless pseudo-orbitals. The TM method uses an
exponential of a polynomial; the RRKJ method uses three or four
Bessel functions. The former is very robust. The latter may 
occasionally fail to produce the required nodeless pseudo-orbital.
If this happens, there is an option to set a small nonzero value of 
the charge density at the origin: this forces the use of four Bessel 
functions.

\subsection{Choosing the matching radii}

At the matching radius $r_c$ the AE and PP wavefunction
of angular momentum $l$ match, with at least continuous first
derivative. The choice of the $r_c$ is very important and must be
guided by the following criteria: 
\begin{enumerate}
\item[--] the $r_c$ must be larger than the outermost node (if any)
of the wavefunction for any given $l$
\item[--] a typical $r_c$ is the outermost peak, beyond if needed
(see following point)
\item[--] the larger the $r_c$, the softer the potential (less PW
needed), but also the less transferable
\item[--] usually there is one $l$ that is harder than the others
(in transition metals, the $d$ state, in second-row elements N, O, F, 
the $p$ state). One should concentrate on this one and push outwards 
its $r_c$ as much as possible.
\item[--] this is not very important, and usually impossible to
achieve: one should try to have not too different $r_c$'s for
different angular momenta
\end{enumerate}
To make a long story short: the difficult question is ``how much
should I push $r_c$ outwards in order to have reasonable results for
unreasonable hard atoms''. There is no well-defined answer. A typical
wavefunction maximum for hard atoms is 0.7-0.8 a.u (in general, the
outermost peak, but hard atoms are those with $2p$, $3d$, $4f$ valence
states, with no orthogonalization to core states of the same $l$ and
no nodes) (this is why they are hard, anyway). $r_c=0.8$ a.u. will
yield unacceptably hard PP's. With a little bit of experience one can
say that for second-row ($2p$) elements $r_c=1.1-1.2$ will yield
reasonably good results for 50-70 Ry PW kinetic energy cutoff; for
$3d$ transition metals, the same $r_c$ will require $> 80$ Ry cutoff
(highest $l$ have slower convergence for the same $r_c$).

Note that it is the hardest atom that determines the PW cutoff in a
solid or molecule. Do not waste time trying to find optimally soft 
PP's if you have harder atoms around.

\subsection{Choosing the parameters for the ``nonlinear core correction''}

The core correction accounts at least partially for the nonlinearity
in the exchange-correlation potential. In the generation of a PP one
first produces a potential with the desired pseudowavefunctions and
pseudoenergies. In order to separate a ``bare'' PP from the screening
part, one removes the screening potential generated by the valence
charge only. This introduces an error because the XC potential is not
linear in the charge density. With the core correction one keeps a  
smoothed core charge to be added to the valence charge both at the
unscreening step and when using the PP.

The core correction is a must for alkali atoms (especially in
ionic compounds) and for PP's to be used in spin-polarized 
(magnetic) systems. It is recommended whenever there is a large 
overlap between valence and core charge: for instance, in transition 
metals if the semicore states are kept into the core. It is never 
harmful but sometimes it may be of little help.

The smoothing works by replacing the true core charge with a fake,
smoother, core charge for $r<r_{cc}$. The parameter $r_{cc}$ is
provided on input. If not, it is chosen as the point at which
the core charge $\rho_c(r_{cc})$ is twice as big as the valence
charge $\rho_v(r_{cc})$. In fact the effect of nonlinearity is 
important only in regions where $\rho_c(r)\sim\rho_v(r)$. Note 
that the smaller $r_{cc}$, the more accurate the core correction, 
but also the harder the smoothened core charge, and vice versa.

\subsection{Choosing the local potential}


\subsection{Generating the pseudopotential}

The generation step is also done by program {\tt ld1.x}.
One has to supply, in addition to AE data: a list of 
orbitals to be used in the pseudization (in increasing
order of angular momentum), the pseudization energies, 
the matching radii, the filename where the newly generated
PP, is written, plus a number of other optional parameters, 
fully described in file {\tt INPUT\_LD1}.

\subsection{Checking for transferability}

A simple way to check for correctness and to get a feeling for 
the transferability of a PP, with little effort, is to test the 
results of PP and AE atomic calculations on atomic configurations 
differing from the starting one. The error on total energy 
differences between PP and AE results gives a feeling on how 
good the PP is. Just to give an idea: an error $\sim 0.001$ Ry 
is very good, $\sim 0.01$ Ry may still be acceptable.
The code {\tt ld1.x} has a ``testing'' mode in which it does
exactly the above operation. You provide the input PP file and
a number of test configurations.

% It is very important to check that the pseudowavefunction match the
% atomic wavefunction as accurately as possible beyond $r_c$.
% Very primitive plotting programs, producing postscript output,
% are available for plotting and comparing pseudo and atomic
% wavefunctions ({\tt wfcgraf}) and for plotting the pseudopotential in
% real space ({\tt psrgraf}). 

% NOTE: in order to use the plotting programs, the labeling of
% states of the PP calculation should be the same as for the AE
% calculations without core states. The appropriate principal
% quantum numbers for valence electrons (i.e. the lowest $s$ state
% must have $n=1$, the lowest $p$ and second $s$ states must have 
% $n=2$ and so on) are automagically generated from wavefunctions
% labels (please check!). Labels are used to identify the AE-PP
% wavefunction correspondence.

Another way to check for transferability is to compare AE and PS 
logarithmic derivatives, also calculated by {\tt ld1.x}. Typically 
this comparison is done on the reference configuration,
but not necessarily. You should supply on input:
\begin{enumerate}
\item[--] the radius $r_d$ at which logarithmic derivatives are 
          calculated ($r_d$ should be of the order of the
          ionic or covalent radius, and larger than any of the $r_c$'s)
\item[--] the energy range $E_{min}, E_{max}$ and the number 
          of points for the plot. The energy range
          should cover the typical valence one-electron energy 
          range expected in the targeted application of the PP. 
\item[--] output file names (one fro AE, one for PP) where results 
          are written.
\end{enumerate}
The file containing logarithmic derivatives can be easily read and 
plotted using for instance the plotting program {\tt xmgrace}. 
Sizable discrepancies between AE and PS logarithmic derivatives 
are a sign of trouble (unless your energy range is too large or 
not centered around the range of pseudization energies, of course).

Note that 
\subsection{Checking for required cutoff}

One can have a feeling of the hardness of the potential from the
following indicators:
\begin{enumerate}
\item[--] the value of $r_c$ (see above)
\item[--] the behavior of the Fourier transform $V_l(q)$. A very
          primitive plotting program {\tt psqgraf} can be used to plot
          $V_l(q)$ on a postscript file. Note that $q^2$ is the energy
          in Ry.
\item[--] an atomic calculation with a basis set of spherical 
          Bessel functions $j_l(qr)$ (equivalent to projecting PW's
          on states with given $l$).
\end{enumerate}
The latter is performed by program {\tt ldb}. It requires on input
a PP file, the energy cutoff in Ry, and the electronic configuration.
The first set of
numbers, under ``Semilocal orbital energies:'', shows the one-electron
levels. At convergence they should be, within numerical accuracy,
equal to those of program {\tt ld1} with the same PP and electronic
configuration. The behavior of the energy levels as a function of the
cutoff will be very close to what one will get in a PW calculation.

NOTE: the above does not replace the usual cutoff checks in
solid-state or molecular calculations. Direct checks on the quantity
of interest in your system are still needed.

\subsection{Checking for separable form}

The separable (KB) form of PP's is exceedingly convenient in
electronic structure calculations, unless you are doing exceedingly
simple systems (crystalline silicon...). In the KB formalism, one
rewrites the PP's as projectors. An arbitrary function can be
added to the local ($l$-independent) part of the PP and subtracted to
all $l$ components. Generally one exploits this arbitrariness to
remove one $l$ component using it as local part.

Unfortunately the KB projection can lead to loss of transferability
(often negligible, sometimes not) or even to the appearance of
``ghost states'' -- states with the wrong number of nodes that are
absent in the all-electron atom -- that make the PP completely
useless.  For practical use, one has to check
carefully whether KB PP are useful or not. In particular one must
exploit the freedom in choosing the ``local part'' in order avoid
ghosts.

For PW calculations it is convenient to choose as local part the
highest $l$, because this removes more projectors ($2l+1$ per atom)
than for low $l$. This is the default choice done by program {\tt trou}.
According to Murphy's law, this is also the choice that
more often gives raise to problems.

% There are three methods to check for ghosts:
% \begin{itemize}
% \item[--] use known criteria for  the appearance of
% ``ghosts''\cite{Gonze} (not implemented).
% \item[--] compare the all-electron (or PP) logarithmic derivative with
% the KB logarithmic derivative (see above). The former are written by
% program {\tt ld1} to a file with suffix {\tt .pp}. The latter (calculated 
% using the KB form for PP but keeping the same SCF potential of semilocal 
% PP) are written to a file with suffix {\tt .kb}. Plot the two, for
% instance using {\tt xmgr}. Any sizable discrepancy between the two 
% is a sign of trouble. Note that for {\tt l=lloc} you must obtain exactly 
% the same values. Use the value of {\tt lloc} that seems to yield better 
% results: edit the PP file to change it (see above).
% \item[--] use program {\tt ldb}. Atomic codes using the integration of
% radial wavefunction (like {\tt ld1}) are unable to find spurious states,
% because their algorithm discards states with wrong number of
% nodes. However if a basis set is used the spurious states will show
% up. The self-consistent calculation is performed first on the PP in the
% normal form. Then a KB Hamiltonian is constructed, using the atomic
% pseudowavefunction and the screening potential just calculated,
% and diagonalized on the spherical Bessel basis set (without
% redoing the self-consistency, it is not needed for our purposes). 
% The results are written in the right-hand part of the output, after 
% ``KB orbital energies:'', and should match as exactly as possible the
% results of the left-hand part. This must be tried at least on the
% reference configuration, for a few occupied and unoccupied 
% (pseudo)states, for all possible local parts.
% \end{itemize}

% A very unfortunate case may sometimes occur: you generate
% a marvelous PP but, no matter what you choose as local reference
% state, ghosts appear. You have to retry with a different reference
% configuration  or matching radii. Good luck.
\appendix

\section{Atomic Calculations}

\subsection{Nonrelativistic case}

Let us assume that the charge density $n(r)$ and the potential $V(r)$
are spherically symmetric. The Kohn-Sham (KS) equation:
\begin{equation}
\left(-{\hbar^2\over 2m}\nabla^2 + V(r)-\epsilon\right) \psi(\r)=0
\end{equation}
can be written in spherical coordinates. We write the wavefunctions as
\begin{equation}
\psi(\r)=\left({R_{nl}(r)\over r}\right)Y_{lm}(\hat\r),
\end{equation}
where $n$ is the main quantum 
number $l=n-1,n-2,\dots,0$ is angular momentum, $m=l,l-1,\dots,-l+1,-l$
is the projection of the angular momentum on some axis. 
The radial KS equation becomes:
\begin{eqnarray}
\left(-{\hbar^2\over 2m} {1\over r} {d^2 R_{nl}(r)\over dr^2}
      +(V(r)-\epsilon) {1\over r} R_{nl}(r)
\right) Y_{lm}(\hat\r) \hskip 4cm \nonumber \\ \mbox{} - 
{\hbar^2\over 2m} \left({1\over\mbox{sin}\theta}{\partial\over\partial\theta}
                        \left(\mbox{sin}\theta{\partial Y_{lm}(\hat\r)
                                               \over\partial\theta}\right)
                      + {1\over\mbox{sin}^2\theta}
                           {\partial^2Y_{lm}(\hat\r)\over\partial\phi^2}
                  \right) {1\over r^3} R_{nl}(r) = 0.
\end{eqnarray}
This yields an angular equation for the spherical harmonics $Y_{lm}(\hat\r)$:
\begin{equation}
-\left({1\over\mbox{sin}\theta}{\partial\over\partial\theta}
       \left(\mbox{sin}\theta{\partial Y_{lm}(\hat\r)\over\partial\theta}
       \right)
     + {1\over\mbox{sin}^2\theta}
                           {\partial^2Y_{lm}(\hat\r)\over\partial\phi^2}
\right) = l(l+1)Y_{lm}(\hat\r)
\end{equation}
and a radial equation for the radial part $R_{nl}(r)$:
\begin{equation}
-{\hbar^2\over 2m} {d^2 R_{nl}(r)\over dr^2}
+\left( {\hbar^2\over 2m} {l(l+1)\over r^2} + V(r)-\epsilon
\right) R_{nl}(r) = 0.
\end{equation}
The charge density is given by
\begin{equation}
n(r) = \sum_{nlm} \Theta_{nl}\left |{R_{nl}(r)\over r}Y_{lm}(\hat r)\right|^2 
     = \sum_{nl} \Theta_{nl}{R^2_{nl}(r)\over 4\pi r^2}
\end{equation}
where $\Theta_{nl}$ are the occupancies ($\Theta_{nl}\le 2l+1$)
and it is assumed that the occupancies of $m$ are such as to yield
a spherically symmetric charge density (which is true only for closed
shell atoms).
\subsubsection{Useful formulae} 

Gradient in spherical coordinates $(r,\theta,\phi)$:
\begin{equation}
\nabla\psi = \left({\partial\psi\over\partial r},
                   {1\over r}{\partial\psi\over\partial\theta},
                   {1\over r \mbox{sin}\theta}
                              {\partial\psi\over\partial\phi}
             \right)
\end{equation}
Laplacian in spherical coordinates:
\begin{equation}
\nabla^2\psi = {1\over r} {\partial^2\over\partial r^2}(r\psi)
             + {1\over r^2\mbox{sin}\theta} {\partial\over\partial\theta}
               \left(\mbox{sin}\theta{\partial\psi\over\partial\theta}\right)
             + {1\over r^2\mbox{sin}^2\theta}
                {\partial^2\psi\over\partial\phi^2}
\end{equation}

\subsection{Fully relativistic case} 

The relativistic KS equations are
Dirac-like equations for a spinor with a ``large'' $R_{nlj}(r)$ and
a ``small'' $S_{nlj}(r)$ component:
\begin{eqnarray}
c\left({d \over dr} + {\kappa\over r}\right)R_{nlj}(r) & = & 
       \left(2mc^2 - V(r) + \epsilon \right)S_{nlj}(r)\\
c\left({d \over dr} - {\kappa\over r}\right)S_{nlj}(r) & = & 
       \left( V(r) + \epsilon \right)       R_{nlj}(r)
\end{eqnarray}
where $j$ is the total angular momentum ($j=1/2$ if $l=0$, 
$j=l+1/2,l-1/2$ otherwise); $\kappa=-2(j-l)(j+1/2)$ is the Dirac 
quantum number ($\kappa=-1$ is $l=0$, $\kappa=-l-1,l$ otherwise);
and the charge density is given by
\begin{equation}
  n(r) = \sum_{nlj} \Theta_{nlj}{R^2_{nlj}(r)+S^2_{nlj}(r)\over 4\pi r^2}.
\end{equation}


\subsection{Scalar-relativistic case} 

The full relativistic KS equations
is be transformed into an equation for the large component only
and averaged over spin-orbit components. In atomic units
(Rydberg: $\hbar=1, m=1/2, e^2=2$):
\begin{eqnarray}
-{d^2 R_{nl}(r)\over dr^2}
+\left( {l(l+1)\over r^2} + M(r)\left(V(r)-\epsilon\right)
\right) R_{nl}(r) \qquad \nonumber \\ \mbox{} -
 {\alpha^2\over 4M(r)} {dV(r)\over dr} 
                    \left({dR_{nl}(r)\over dr} +
                          \langle\kappa\rangle {R_{nl}(r)\over r}\right)= 0,
\end{eqnarray}
where $\alpha=1/137.036$ is the fine-structure constant,
$\langle\kappa\rangle=-1$ is the degeneracy-weighted average value 
of the Dirac's $\kappa$ for the two spin-orbit-split levels, $M(r)$ is
defined as
\begin{equation}
M(r)= 1 - {\alpha^2\over 4} \left(V(r)-\epsilon\right).
\end{equation}
The charge density is defined as in the nonrelativistic case:
\begin{equation}
n(r) = \sum_{nl} \Theta_{nl}{R^2_{nl}(r)\over 4\pi r^2}.
\end{equation}

\subsection{Numerical solution} 

The radial (scalar-relativistic) KS equation is integrated 
on a radial grid. It is convenient to
have a denser grid close to the nucleus and a coarser one far
away. Traditionally a logarithmic grid is used: 
$r_i=r_0\mbox{exp}(i\Delta x)$. With this grid, one has
\begin{equation}
\int_0^\infty f(r) d r = \int_0^\infty f(x) r(x) dx
\end{equation}
and
\begin{equation}
{d f(r)\over d r}={1\over r} {d f(x)\over d x},\qquad
{d^2 f(r)\over d r^2}=-{1\over r^2} {d f(x)\over d x}
+ {1\over r^2} {d^2 f(x)\over d x^2}.
\end{equation}
We start with a given self-consistent potential $V$ and
a trial eigenvalue $\epsilon$. The equation is integrated
from $r=0$ outwards to $r_t$, the outermost classical 
(nonrelativistic for simplicity) turning point, defined
by $ {l(l+1) /r_t^2} + \left(V(r_t)-\epsilon\right)=0$.
In a logarithmic grid (see above) the equation to solve becomes:
\begin{eqnarray}
{1\over r^2} {d^2 R_{nl}(x)\over d x^2} & = & 
  {1\over r^2} {d R_{nl}(x)\over d x} + 
  \left( {l(l+1)\over r^2} + 
  M(r)\left(V(r)-\epsilon\right) \right) R_{nl}(r) \nonumber \\
 & & \mbox{} - {\alpha^2\over 4M(r)} {dV(r)\over dr} 
      \left({1\over r} {dR_{nl}(x)\over dx} +
            \langle\kappa\rangle {R_{nl}(r)\over r}\right).
\end{eqnarray}
This determines ${d^2 R_{nl}(x)/d x^2}$ which is used to
determine ${d R_{nl}(x)/ dx}$ which in turn is used to
determine $R_{nl}(r)$, using predictor-corrector or whatever
classical integration method. ${dV(r)/dr}$ is evaluated
numerically from any finite difference method. The series
is started using the known (?) asymptotic behavior of $R_{nl}(r)$ 
close to the nucleus (with ionic charge $Z$)
\begin{equation}
R_{nl}(r)\simeq r^\gamma,\qquad \gamma={l\sqrt{l^2-\alpha^2 Z^2}+
(l+1)\sqrt{(l+1)^2-\alpha^2 Z^2}\over 2l+1}.
\end{equation}
The number of nodes is counted. If there are too few (many)
nodes, the trial eigenvalue is increased (decreased) and
the procedure is restarted until the correct number $n-l-1$
of nodes is reached. Then a second integration is started 
inward, starting from a suitably large $r\sim 10r_t$ down 
to $r_t$, using as a starting point the asymptotic behavior 
of $R_{nl}(r)$ at large $r$: 
\begin{equation}
R_{nl}(r)\simeq e^{-k(r)r},\qquad 
k(r)=\sqrt{{l(l+1)\over r^2} + \left(V(r)-\epsilon\right)}.
\end{equation}
The two pieces are continuously joined at $r_t$ and a correction to the trial 
eigenvalue is estimated using perturbation theory (see below). The procedure 
is iterated to self-consistency.

The perturbative estimate of correction to trial eigenvalues is described in
the following for the nonrelativistic case (it is not worth to make relativistic
corrections on top of a correction). The trial eigenvector $R_{nl}(r)$ will have 
a cusp at $r_t$ if the trial eigenvalue is not a true eigenvalue:
\begin{equation}
A = {d R_{nl}(r_t^+)\over dr} - {d R_{nl}(r_t^-)\over dr} \ne 0.
\end{equation}
Such discontinuity in the first derivative translates into a
$\delta(r_t)$ in the second derivative:
\begin{equation}
{d^2  R_{nl}(r)\over dr^2} = {d^2 \widetilde R_{nl}(r)\over dr^2}
  + A \delta(r-r_t)
\end{equation}
where the tilde denotes the function obtained by matching the
second derivatives in the $r< r_t$ and $r> r_t$ regions.
This means that we are actually solving a different problem in which 
$V(r)$ is replaced by $V(r)+\Delta V(r)$, 
given by
\begin{equation}
 \Delta V(r) = -{\hbar^2 \over 2m} {A\over R_{nl}(r_t)}\delta(r-r_t).
\end{equation}
The energy difference between the solution to such fictitious potential 
and the solution to the real potential can be estimated from
perturbation theory:
\begin{equation}
 \Delta\epsilon_{nl} = - \langle\psi|\Delta V |\psi\rangle
                     =   {\hbar^2 \over 2m} R_{nl}(r_t) A.
\end{equation}

\section{Equations for the Troullier-Martins method}

We assume a pseudowavefunction $R^{ps}$ having the following form:
\begin{eqnarray}
R^{ps}(r)&=&r^{l+1}e^{p(r)} \quad r\le r_c \\
R^{ps}(r)&=&R(r)         \quad r\ge r_c
\end{eqnarray}
where
\begin{equation}
p(r)=c_0+c_2r^2+c_4r^4+c_6r^6+c_8r^8+c_{10}r^{10}+c_{12}r^{12}.
\end{equation}
On this pseudowavefunction we impose the norm conservation
condition:
\begin{equation}
\int_{r<r_c} (R^{ps}(r))^2 dr=\int_{r<r_c} (R(r))^2 dr
\end{equation}
and continuity conditions on the wavefunction and its derivatives up
to order four at the matching point:
\begin{equation}
{d^nR^{ps}(r_c)\over dr^n}={d^nR(r_c)\over dr^n}, \quad n=0,...,4
\end{equation}

\noindent$\bullet$ Continuity of the wavefunction:
\begin{equation}
R^{ps}(r_c)=r_c^{l+1}e^{p(r_c)}=R(r_c)
\end{equation}
\begin{equation}
p(r_c) = \mbox{log}{R(r_c)\over r_c^{l+1}} 
\end{equation}

\noindent$\bullet$ Continuity of the first derivative of the wavefunction:
\begin{equation}
{dR^{ps}(r)\over dr} = (l+1)r^le^{p(r)} + r^{l+1}e^{p(r)}p'(r)
                         ={l+1\over r}R^{ps}(r) + p'(r)R^{ps}(r)
\end{equation}
that is
\begin{equation}
 p'(r_c) = {dR(r_c)\over dr} {1\over R^{ps}(r_c)} -
           {l+1\over r_c}.
\end{equation}

\noindent$\bullet$ Continuity of the second  derivative of the wavefunction:
\begin{eqnarray}
{d^2R^{ps}(r)\over d^2r} 
   & = &
     {d\over dr}\left((l+1)r^le^{p(r)} + r^{l+1}e^{p(r)}p'(r)\right) 
\nonumber \\ & = &
     l(l+1)r^{l-1}e^{p(r)} + 2(l+1)r^le^{p(r)}p'(r) +
     r^{l+1}e^{p(r)}\left[p'(r)\right]^2 + r^{l+1}e^{p(r)}p''(r)
\nonumber\\ & = &
     \left( {l(l+1)\over r^2}+ {2(l+1)\over r}p'(r) +
            \left[p'(r)\right]^2 + p''(r) \right) r^{l+1}e^{p(r)}.
\end{eqnarray}
From the radial Schr\"odinger equation:
\begin{equation}
 {d^2R^{ps}(r)\over dr^2} = 
    \left ( {l(l+1)\over r^2} +{2m\over\hbar^2}(V(r)-\epsilon)\right)R^{ps}(r)
\end{equation}
that is
\begin{equation}
p''(r_c) = {2m\over\hbar^2}(V(r_c)-\epsilon) - 2 {l+1\over r_c}p'(r_c)
         -\left[p'(r_c)\right]^2 
\end{equation}

\noindent$\bullet$ Continuity of the third and fourth derivatives of the
wavefunction. This is assured if the third and fourth derivatives of
$p(r)$ are continuous. By direct derivation of the expression of
$p''(r)$:
\begin{equation}
p'''(r_c) = {2m\over\hbar^2}V'(r_c) + 2 {l+1\over r^2_c}p'(r_c)
          - 2{l+1\over r_c}p''(r_c) -2 p'(r_c)p''(r_c)
\end{equation}

\begin{eqnarray}
p''''(r_c) & = & {2m\over\hbar^2}V''(r_c) - 4 {l+1\over r^3_c}p'(r_c)
                 + 4{l+1\over r^2_c}p''(r) \nonumber \\ & & \mbox{} 
            - 2{l+1\over r_c}p'''(r_c)
           - 2 \left[p''(r_c)p''(r_c)\right]^2 -2 p'(r_c)p'''(r_c)
\end{eqnarray}

The additional condition: $V''(0)=0$ is imposed. 
The screened potential is
\begin{eqnarray}
V(r)&=&{\hbar^2\over2m}\left({1\over R^{ps}(r)}{d^2R^{ps}(r)\over dr^2} 
     - {l(l+1)\over r^2}\right)+\epsilon \\
    &=&{\hbar^2\over2m}\left( 2{l+1\over r} p'(r)+\left[p(r)\right]^2+p''(r)
       \right)+\epsilon
\end{eqnarray}

Keeping only lower-order terms in $r$:
\begin{eqnarray}
V(r) &\simeq&{\hbar^2\over2m}\left( 2{l+1\over r} (2c_2r + 4c_4r^3) 
               + 4c_2^2r^2 + 2c_2 + 12c_4r^2\right) + \epsilon \\
     &   =  & {\hbar^2\over2m}\left( 2c_2(2l+3) + 
              \left((2l+5)c_4+c_2^2\right) r^2\right)+ \epsilon .
\end{eqnarray}
The additional constraint is:
\begin{equation}
(2l+5)c_4+c_2^2=0.
\end{equation}



\begin{thebibliography}{10}

\bibitem{NC} 
D.R. Hamann, M. Schl\"uter, and C. Chiang, 
Phys. Rev. Lett. {\bf 43}, 1494 (1979).

\bibitem{van} D. Vanderbilt, Phys. Rev. B  {\bf 47}, 10142 (1993).

\bibitem{BHS} 
G.B. Bachelet, D.R. Hamann and M. Schl\"uter, Phys. Rev. B {\bf 26},
4199 (1982). 

\bibitem{Gonze} 
X. Gonze, R. Stumpf, and M. Scheffler, Phys. Rev. B {\bf 44}, 8503 (1991).

\bibitem{Goedecker} 
S. Goedecker, M. Teter, and J. Hutter, Phys. Rev. B {\bf 54}, 1703 (1996).

\bibitem{TM} N. Troullier and J.L. Martins, Phys. Rev. B {\bf 43},
1993 (1991).

\bibitem{fhi98PP} 
M.Fuchs and M. Scheffler, Comput. Phys. Commun. {\bf 119}, 67 (1999).

\bibitem{RRKJ} A. M. Rappe, K. M. Rabe, E. Kaxiras, and J. D. Joannopoulos, 
Phys. Rev. B {\bf 41}, 1227 (1990) 
(erratum: Phys. Rev. B {\bf 44}, 13175 (1991)).

\bibitem{PAW} P.~E. Bl\"ochl, Phys. Rev. B {\bf 50}, 17953 (1994).

\bibitem{KB} L. Kleinman and D.M. Bylander, Phys. Rev. Lett. 48, 1425 (1982).

\bibitem{CoreCorr} 
S.G. Louie, S. Froyen, and M.L. Cohen, Phys. Rev. B {\bf 26}, 1738 (1982).

\bibitem{Zunger} S.H. Wei and A. Zunger, Phys. Rev. B {\bf 37}, 8958 (1987).

\end{thebibliography}

\end{document}
