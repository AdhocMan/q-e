\documentclass[12pt,a4paper]{article}
\def\version{6.5+}
\def\qe{{\sc Quantum ESPRESSO}}

\usepackage{html}

% BEWARE: don't revert from graphicx for epsfig, because latex2html
% doesn't handle epsfig commands !!!
\usepackage{graphicx}

\textwidth = 17cm
\textheight = 24cm
\topmargin =-1 cm
\oddsidemargin = 0 cm

\def\pwx{\texttt{pw.x}}
\def\cpx{\texttt{cp.x}}
\def\phx{\texttt{ph.x}}
\def\epwx{\texttt{epw.x}}
\def\nebx{\texttt{neb.x}}
\def\configure{\texttt{configure}}
\def\PWscf{\texttt{PWscf}}
\def\PHonon{\texttt{PHonon}}
\def\CP{\texttt{CP}}
\def\PostProc{\texttt{PostProc}}
\def\NEB{\texttt{PWneb}}
\def\make{\texttt{make}}
\def\libxc{\texttt{libxc}}

\begin{document}
\author{}
\date{}

\def\qeImage{quantum_espresso}

\title{
  \includegraphics[width=5cm]{\qeImage} \\
  % title
  \Huge User's Guide for \\ \qe\ (v.\version)
}

\maketitle

\tableofcontents

\section{Introduction}

This guide gives a general overview of the contents and of the installation
of \qe\ (opEn-Source Package for Research in Electronic Structure, Simulation,
and Optimization), version \version.

{\em Important notice: due to the lack of time and of manpower, this
  manual does not cover many important aspects and may contain outdated
  information.}

The \qe\ distribution contains the core packages \PWscf\ (Plane-Wave
Self-Consistent Field) and \CP\ (Car-Parrinello) for the calculation
of electronic-structure properties within
Density-Functional Theory (DFT), using a Plane-Wave (PW) basis set
and pseudopotentials. It also includes other packages for
more specialized calculations:
\begin{itemize}
  \item \NEB:
        energy barriers and reaction pathways through the Nudged Elastic Band
        (NEB) method.
      \item \PHonon:
        vibrational properties  with Density-Functional Perturbation Theory
        (DFPT).
  \item \PostProc:
        codes and utilities for data postprocessing.
  \item \texttt{PWcond}:
        ballistic conductance.
  \item \texttt{XSPECTRA}:
        K-, L$_1$-, L$_{2,3}$-edge X-ray absorption spectra.
  \item \texttt{TD-DFPT}:
        spectra from Time-Dependent
        Density-Functional Perturbation Theory.
  \item \texttt{GWL}: electronic excitations within the GW approximation
        and with the Bethe-Salpeter Equation      
  \item \texttt{EPW}: calculation of the electron-phonon coefficients
        and related quantities;
  \item \texttt{HP}: calculation of Hubbard $U$ parameters using DFPT.
\end{itemize}
The following auxiliary packages are included as well:
\begin{itemize}
\item \texttt{PWgui}:
      a Graphical User Interface, producing input data files for
      \PWscf\ and some \PostProc\ codes.
\item \texttt{atomic}:
      atomic calculations and pseudopotential generation.
%\item \texttt{QHA}:
%      utilities for the calculation of projected density of states (PDOS)
%      and of the free energy in the Quasi-Harmonic Approximation (to be
%      used in conjunction with \PHonon).
%\item \texttt{PlotPhon}:
%      phonon dispersion plotting utility (to be
%      used in conjunction with \PHonon).
\end{itemize}
A copy of required external libraries is also included.
Finally, several additional packages that exploit data produced by \qe\
or patch some \qe\ routines can be automatically installed using
\texttt{make}:
\begin{itemize}
\item \texttt{Wannier90}:
      maximally localized Wannier functions.
\item \texttt{WanT}:
      quantum transport properties with Wannier functions.
\item \texttt{YAMBO}:
      electronic excitations within Many-Body Perturbation Theory,
      GW and Bethe-Salpeter equation.
\item \texttt{PLUMED} (v.1.3 only):
      calculation of free-energy surface through metadynamics.
\item \texttt{GIPAW} (Gauge-Independent Projector Augmented Waves):
      NMR chemical shifts and EPR g-tensor.
%\item \texttt{WEST}: Many-body perturbation corrections for standard DFT.
\end{itemize}
For \qe\ with the self-consistent continuum solvation (SCCS) model,
aka ``Environ'', see \texttt{http://www.quantum-environment.org/}.

Documentation on single packages can be found in the \texttt{Doc/}
directory of each package. A detailed description of input
data is available for most packages in files \texttt{INPUT\_*.txt} and
\texttt{INPUT\_*.html}.

The \qe\ codes work on many different types of Unix machines,
including parallel machines using both OpenMP and MPI
(Message Passing Interface).
\qe\ also runs on Mac OS X and MS-Windows machines
(see section \ref{Sec:Installation}). A GPU-enabled version is
available on \texttt{https://github.com/fspiga/qe-gpu}.

Further documentation, beyond what is provided in this guide, can be found in:
\begin{itemize}
\item the \texttt{Doc/} and \texttt{examples/} directories
  of the \qe\ distribution;
  \item the web site \texttt{www.quantum-espresso.org};
  \item the archives of the  mailing list:
   See section \ref{SubSec:Contacts}, ``Contacts'', for more info.
\end{itemize}
People who want to contribute to \qe\ should read the
Developer Manual: \texttt{Doc/developer\_man.pdf}.

This guide does not explain the basic Unix concepts (shell, execution
path, directories etc.) and utilities needed to run \qe; it does not
explain either solid state physics and its computational methods.
If you want to learn the latter, you should first read a good textbook,
such as e.g. the book by Richard Martin:
{\em Electronic Structure: Basic Theory and Practical Methods},
Cambridge University Press (2004); or:
{\em Density functional theory: a practical introduction},
D. S. Sholl, J. A. Steckel (Wiley, 2009); or
{\em Electronic Structure Calculations for Solids and Molecules:
Theory and Computational Methods},
J. Kohanoff (Cambridge University Press, 2006). Then you should consult
the documentation of the package you want to use for more specific references.

All trademarks mentioned in this guide belong to their respective owners.

\subsection{People}

The maintenance and further development of the \qe\ distribution
is promoted by the \qe\ Foundation under the coordination of
Paolo Giannozzi (Univ.Udine and IOM-CNR, Italy) and Pietro Delugas
(SISSA Trieste) with the strong support
of the CINECA National Supercomputing Center in Bologna under
the responsibility of Carlo Cavazzoni.

Contributors to \qe, beyond the authors of the papers
mentioned in Sec.\ref{SubSec:Terms}, include:
\begin{itemize}
  \item Ye Luo (Argonne) for improved FFT threading and miscellaneous
  contributions and optimizations;
  \item Pietro Bonf\`a (CINECA) for multiple contributions to optimization,
  GPU version, and maintenance;
  \item Fabio Affinito (CINECA) for ELPA support, for contributions
  to the FFT library, and for various parallelization improvements;
  \item Sebastiano Caravati for direct support of GTH pseudopotentials
  in analytical form, Santana Saha and Stefan Goedecker (Basel U.)
  for improved UPF converter of newer GTH pseudopotentials;
  \item Axel Kohlmeyer for libraries and utilities to call \qe\
  from external codes (see the \texttt{COUPLE} sub-directory), made the
  parallelization more modular and usable by external codes;
  \item \`Eric Germaneau for TB09 meta-GGA functional, using \libxc;
  \item Guido Roma (CEA Saclay) for  vdw-df-obk8  e vdw-df-ob86 functionals; 
  \item Yves Ferro (Univ. Provence) for SOGGA and M06L functionals; 
  \item Ikutaro Hamada (NIMS, Japan) for OPTB86B-vdW, REV-vdW-DF2
    functionals, fixes to pw2xsf utility;
  \item Daniel Forrer (Padua Univ.) and Michele Pavone
  (Naples Univ. Federico II) for dispersions interaction in the
  framework of DFT-D;
  \item Filippo Spiga (University of Cambridge, UK) for mixed MPI-OpenMP
  parallelization and for the GPU-enabled version;
  \item Costas Bekas and Alessandro Curioni (IBM Zurich) for the initial
  BlueGene porting.
\end{itemize}

Contributors to specific \qe\ packages are acknowledged in the
documentation of each package.

An alphabetic list of further
contributors who answered questions on the mailing list, found
bugs, helped in porting to new architectures, wrote some code,
contributed in some way or another at some stage, follows:
\begin{quote}
  {\AA}ke Sandgren, Audrius Alkauskas, Alain Allouche, Francesco Antoniella,
  Uli Aschauer, Francesca Baletto, Gerardo Ballabio, Mauro Boero,
  Scott Brozell, Claudia Bungaro, Paolo Cazzato, Gabriele Cipriani,
  Jiayu Dai, Stefano Dal Forno, Cesar Da Silva, Alberto Debernardi,
  Gernot Deinzer, Alin Marin Elena, Francesco Filipponi, Prasenjit Ghosh,
  Marco Govoni, Thomas Gruber, Martin Hilgeman, Yosuke Kanai, Konstantin Kudin,
  Nicolas Lacorne, Hyungjun Lee, Stephane Lefranc, Sergey Lisenkov, Kurt Maeder,
  Andrea Marini, Giuseppe Mattioli, Nicolas Mounet, William Parker,
  Pasquale Pavone, Samuel Ponc\'e, Mickael Profeta, Chung-Yuan Ren,
  Kurt Stokbro, David Strubbe, Sylvie Stucki, Paul Tangney, Pascal Thibaudeau,
  Davide Tiana, Antonio Tilocca, Jaro Tobik, Malgorzata Wierzbowska,
  Vittorio Zecca, Silviu Zilberman, Federico Zipoli,
\end{quote}
and let us apologize to everybody we have forgotten.

\subsection{Contacts}
\label{SubSec:Contacts}

The web site for \qe\ is \texttt{http://www.quantum-espresso.org/}.
Releases and patches can be downloaded from this
site or following the links contained in it. The main entry point for
developers is the GitLab web site:
\texttt{https://gitlab.com/QEF/q-e}.

The recommended place where to ask questions about installation
and usage of \qe, and to report problems, is the 
mailing list \texttt{users@lists.quantum-espresso.org}.
Here you can obtain help from the developers and from
knowledgeable users. You have to be subscribed (see the ``Contacts''
section of the web site) in order to post to the  users' list.
Please check your spam folder if you do not get a confirmation
message when subscribing.

Please read the guidelines for posting, section \ref{SubSec:Guidelines}!
PLEASE NOTE: only messages that appear to come from the
registered user's e-mail address, in its {\em exact form}, will be
accepted. In case of trouble, carefully check that your return
e-mail is the correct one (i.e. the one you used to subscribe).
Please also note that since some time the correct functioning of mailing
list is frequently disrupted by policy changes at large commercial
providers. There is not much we can do about that.

If you need to contact the developers for {\em specific} questions
about coding, proposals, offers of help, etc., you may either post
an ``Issue'' to GitLab, or send a message
to the developers' mailing list \texttt{developers@lists.quantum-espresso.org}.
Please do not post general questions there: they will be ignored.

\subsection{Guidelines for posting to the mailing list}
\label{SubSec:Guidelines}
Life for mailing list subscribers will be easier if everybody
complies with the following guidelines:
\begin{itemize}
\item Before posting, {\em please}: browse or search the archives --
  links are available in the ``Contacts'' section  of the   web site.
  Most questions are asked over and over again. Also: make an attempt
  to search the
  available documentation, notably the FAQs and the User Guide(s).
  The answer to most questions is already there.
\item Reply to both the mailing list and the author or the post, using
  ``Reply to all''.
\item Sign your post with your name and affiliation.
\item Choose a meaningful subject. Do not use "reply" to start a new
  thread:
  it will confuse the ordering of messages into threads that most mailers
  can do. In particular, do not use "Reply" to a Digest!!!
\item Be short: no need to send 128 copies of the same error message just
  because this is what came out of your 128-processor run. No need to
  send the entire compilation log for a single error appearing at the end.
\item Do not post large attachments: point a linker to a place where the
  attachment(s) can be downloaded from, such as e.g. DropBox, GoogleDocs, 
  or one of the various web temporary storage spaces.
\item Avoid excessive or irrelevant quoting of previous messages. Your
  message must be immediately visible and easily readable, not hidden
  into a sea of quoted text.
\item Remember that even experts cannot guess where a problem lies in
  the absence of sufficient information. One piece of information that
  must {\em always} be provided is the version number of \qe.
\item Remember that the mailing list is a voluntary endeavor: nobody is
  entitled to an answer, even less to an immediate answer.
\item Finally, please note that the mailing list is not a replacement
  for your own work, nor is it a replacement for your thesis director's
  work.
\end{itemize}

\subsection{Terms of use}
\label{SubSec:Terms}

\qe\ is free software, released under the
GNU General Public License. See
\texttt{http://www.gnu.org/licenses/old-licenses/gpl-2.0.txt},
or the file License in the distribution).

%%%%%%%%%%%%%%%%%%%%%%%%%%%%%%%%%%%%%%%%%%%%%%%%%%%%%%%%%%%%%%%%%
\input quote.tex
%%%%%%%%%%%%%%%%%%%%%%%%%%%%%%%%%%%%%%%%%%%%%%%%%%%%%%%%%%%%%%%%%
\section{Installation}

\subsection{Download}
\label{SubSec:Download}

\qe\ is distributed in source form, but selected binary packages
and virtual machines are also available.
Stable and development releases of the \qe\ source package
(current version is \version), as well as available binary
packages, can be downloaded from the links listed in the
``Download'' section of \texttt{www.quantum-espresso.org}.

The Quantum Mobile virtual machine for Windows/Mac/Linux/Solaris
provides a complete Ubuntu Linux environment, containing \qe\ and
more. Link and description in
\texttt{https://www.materialscloud.org/work/quantum-mobile}.

For source compilation, uncompress and unpack compressed archives
in the typical .tar.gz format using the command:
\begin{verbatim}
     tar zxvf qe-X.Y.Z.tar.gz
\end{verbatim}
(a hyphen before "zxvf" is optional) where \texttt{X.Y.Z} stands for the
version number. If your version of \texttt{tar}
doesn't recognize the "z" flag:
\begin{verbatim}
     gunzip -c qe-X.Y.Z.tar.gz | tar xvf -
\end{verbatim}
A directory \texttt{qe-X.Y.Z/} will be created.

A few additional packages that are not included in the base distribution
will be downloaded on demand at compile time, using \texttt{make}
(see Sec.\ref{SubSec:Compilation}).
Note however that this will work only if the computer you are
installing on is directly connected to the internet and has
either \texttt{wget} or \texttt{curl} installed and working.
If you run into trouble, manually download each required package
into subdirectory \texttt{archive/}, {\em not unpacking or
uncompressing it}:
command \texttt{make} will take care of this during installation.

% Occasionally, patches for the current version, fixing some errors and bugs,
% may be distributed as a "diff" file. In order to install a patch (for
% instance):
% \begin{verbatim}
%    cd espresso-X.Y.Z/
%    patch -p1 < /path/to/the/diff/file/patch-file.diff
% \end{verbatim}
%If more than one patch is present, they should be applied in the correct order.

% Daily snapshots of the development version can be downloaded from the
%developers' site \texttt{qe-forge.org}: follow the link ''Quantum ESPRESSO'',
%then ''SCM''.

%The bravest may access the development version via anonymous access to the
%Subversion (SVN) repository: \texttt{qe-forge.org/gf/project/q-e/scmsvn},
%link ''Access Info'' on the left. See also the Developer Manual
%(\texttt{Doc/developer\_man.pdf}), section ''Using SVN''.
%Beware: the development version
%is, well, under development: use at your own risk!

The \qe\ distribution contains several directories. Some of them are
common to all packages:

\begin{tabular}{ll}
\texttt{Modules/} &  Fortran modules and utilities used by all programs\\
\texttt{include/} &  files *.h included by fortran and C source files\\
\texttt{clib/}    &  libraries and utilities written in C\\
\texttt{FFTXlib/} &  FFT libraries\\
\texttt{LAXlib/}  &  Linear Algebra (parallel) libraries\\
\texttt{KS\_Solvers/}  &  Iterative diagonalization routines\\
\texttt{UtilXlib/}&  Miscellaneous timing, error handling, MPI utilites\\
\texttt{install/} &  installation scripts and utilities\\
\texttt{pseudo}/  &  pseudopotential files used by examples\\
\texttt{upftools/}&  converters to unified pseudopotential format (UPF)\\
\texttt{Doc/}     &  general documentation\\
\texttt{archive/} &  external libraries in .tar.gz form\\
\texttt{test-suite/} &  automated tests\\
\end{tabular}
\\
while others are specific to a single package:

\begin{tabular}{ll}
\texttt{PW/}      & \PWscf\ package\\
\texttt{EPW/}     & \texttt{EPW} package\\
\texttt{NEB/}     & \NEB\ package\\
\texttt{PP/}      & \PostProc\ package\\
\texttt{PHonon/}  & \PHonon\ package\\
\texttt{PWCOND/}  & \texttt{PWcond}\ package\\
\texttt{CPV/}     & \CP\ package\\
\texttt{atomic/}  & \texttt{atomic} package\\
\texttt{GUI/}     & \texttt{PWGui} package\\
\texttt{HP/}      & \texttt{HP} package
\end{tabular}

Finally, directory \texttt{COUPLE/} contains code and documentation
that is useful to call \qe\ programs from external codes; directory
\texttt{LR\_Modules/} contains source files for modules that are common
to all linear-response codes.
\subsection{Prerequisites}
\label{Sec:Installation}

To install \qe\ from source, you need first of all a minimal Unix
environment, that is: a command shell (e.g., bash, sh) and utilities \make,
\texttt{awk}, \texttt{sed}. For MS-Windows, see Sec.\ref{SubSec:Windows}.

Note that the scripts contained
in the distribution assume that the local  language is set to the
standard, i.e. "C"; other settings
may break them. Use \texttt{export LC\_ALL=C} (sh/bash) or
\texttt{setenv LC\_ALL C} (csh/tcsh) to prevent any problem
when running scripts (including installation scripts).

Second, you need C and Fortran compilers, compliant with C89 and
F2003 standards\footnote{since v.6.4 a standard 2008 feature is
used: if unallocated pointers are passed as optional arguments,
they are interpreted as not present}. For parallel
execution, you will also need MPI libraries and a parallel
(i.e. MPI-aware) compiler. For massively parallel machines, or
for simple multicore parallelization, an OpenMP-aware compiler
and libraries are also required.

As a rule, \qe\ tries to keep compatibility with older compilers,
avoiding nonstandard extensions and newer features that
are not widespread or stabilized. If however your compiler is older 
than $\sim 5$ years or so, it is unlikely to work. The same applies
to mathematical and MPI libraries.

Big machines with
specialized hardware (e.g. IBM SP, CRAY, etc) typically have a
Fortran compiler with MPI and OpenMP libraries bundled with
the software. Workstations or ``commodity'' machines, using PC
hardware, may or may not have the needed software. If not, you need
either to buy a commercial product (e.g Intel, NAG, PGI) or to
use an open-source compiler like gfortran from the gcc distribution.
Some commercial compilers (e.g. PGI) may be available free of charge 
under some conditions (e.g. academic or personal usage).

\subsection{\configure}

To install the \qe\ source package, run the \configure\
script. This is actually a wrapper to the true \configure,
located in the \texttt{install/} subdirectory (\configure\ -h for help).
\configure\ will (try to) detect compilers and libraries available on
your machine, and set up things accordingly. Presently it is expected
to work on most Linux 32- and 64-bit PCs (all Intel and AMD CPUs) and
PC clusters, IBM BlueGene machines, NEC SX, Cray XT
machines, Mac OS X, MS-Windows PCs. Detailed but sometimes outdated
installation instructions for specific HPC machines may be found in
files \texttt{install/README.}{\em sys},
where {\em sys} is the machine name.

Instructions for the impatient:
\begin{verbatim}
    cd qe-X.Y.Z/
    ./configure
     make all
\end{verbatim}
This will (try to) produce parallel (MPI) executable if a proper parallel
environment is detected, serial executables otherwise. For OpenMP executables,
specify \texttt{./configure --enable-openmp}. Symlinks to executable programs
will be placed in the \texttt{bin/}
subdirectory. Note that both C and Fortran compilers must be in your execution
path, as specified in the PATH environment variable.
Additional instructions for special machines:

\begin{tabular}{ll}
    \texttt{./configure ARCH=crayxt4}& for CRAY XT machines \\
    \texttt{./configure ARCH=necsx}   & for NEC SX machines \\
    \texttt{./configure ARCH=ppc64-mn}& PowerPC Linux + xlf (Marenostrum) \\
    \texttt{./configure ARCH=ppc64-bg}& IBM BG/P (BlueGene)
\end{tabular}

\noindent \configure\ generates the following files:

\begin{tabular}{ll}
\texttt{make.inc} &      compilation rules and flags (used by \texttt{Makefile})\\
\texttt{install/configure.msg} & a report of the configuration run (not needed for compilation)\\
\texttt{install/config.log} & detailed log of the configuration run (useful for debugging)\\
\texttt{include/c\_defs.h} &    a few definitions used by C files\\
\texttt{include/configure.h} &  info on compilation flags (not used: in Modules/environment.f90\\
&  uncomment \verb|#define __HAVE_CONFIG_INFO| to enable its usage)\\ \end{tabular}\\
NOTA BENE: \configure\ no longer updates files \texttt{make.depend},
containing dependencies upon modules, in the various subdirectories.
If you modify the sources, run \texttt{./install/makedeps.sh} or type
\texttt{make depend} to update files \texttt{make.depend}.\\
NOTA BENE 2: \texttt{make.inc} used to be called \texttt{make.sys}
until v.6.0. The change of name is due to frequent problems with mailers
assuming that whatever ends in \texttt{.sys} is a suspect virus.
NOTA BENE 3: if you interrupt \make, it may fail when you start it again 
later (this will happen for instance if \make\ is interrupted while unpacking
and compiling the FoX library). If so, run \texttt{make clean} before
running \make\ again.

You should always be able to compile the \qe\ suite
of programs without having to edit any of the generated files. However you
may have to tune \configure\ by specifying appropriate environment variables
and/or command-line options. Usually the tricky part is to get external
libraries recognized and used: see Sec.\ref{Sec:Libraries}
for details and hints.

Environment variables may be set in any of these ways:
\begin{verbatim}
     export VARIABLE=value; ./configure             # sh, bash, ksh
     setenv VARIABLE value; ./configure             # csh, tcsh
     env VARIABLE=value ./configure                 # any shell
     ./configure VARIABLE=value                     # any shell
\end{verbatim}
Some environment variables that are relevant to \configure\ are:

\begin{tabular}{ll}
\texttt{ARCH}& label identifying the machine type (see below)\\
\texttt{F90, F77, CC} &names of Fortran, Fortran-77, and C compilers\\
\texttt{MPIF90} &       name of parallel Fortran 90 compiler (using MPI)\\
\texttt{CPP} &          source file preprocessor (defaults to \$CC -E)\\
\texttt{LD} &           linker (defaults to \$MPIF90)\\
\texttt{(C,F,F90,CPP,LD)FLAGS}& compilation/preprocessor/loader flags\\
\texttt{LIBDIRS}&     extra directories where to search for libraries\\
\end{tabular}\\
(note that \texttt{F90} is an ``historical'' name -- we actually use
Fortran 2003 -- and that it should be used only together with option
\texttt{--disable-parallel}. In fact, the value of F90 must be
consistent with the parallel Fortran compiler which is determined by 
\configure\ and stored in the \texttt{MPIF90} variable).

For example, the following command line:
\begin{verbatim}
     ./configure MPIF90=mpif90 FFLAGS="-O2 -assume byterecl" \
                  CC=gcc CFLAGS=-O3 LDFLAGS=-static
\end{verbatim}
instructs \configure\ to use \texttt{mpif90} as Fortran compiler
with flags \texttt{-O2 -assume byterecl}, \texttt{gcc} as C compiler with
flags \texttt{-O3}, and to link with flag \texttt{-static}.
Note that the value of \texttt{FFLAGS} must be quoted, because it contains
spaces. NOTA BENE: passing the complete path to compilers (e.g.,
\texttt{F90=/path/to/f90xyz}) may lead to obscure errors during
compilation. As a rule: do not define environment variables for
\configure\ unless you need it. Always try
\configure\ with no options as a first step.

If your machine type is unknown to \configure, you may use the
\texttt{ARCH}
variable to suggest an architecture among supported ones. Some large
parallel machines using a front-end (e.g. Cray XT) will actually
need it, or else \configure\ will correctly recognize the front-end
but not the specialized compilation environment of those machines.
In some cases, cross-compilation requires to specify the target machine with the
\texttt{--host} option. This feature has not been extensively
tested, but we had at least one successful report (compilation
for NEC SX6 on a PC). Currently supported architectures are:

\begin{tabular}{ll}
\texttt{ia32}&    Intel 32-bit machines (x86) running Linux\\
\texttt{ia64}&    Intel 64-bit (Itanium) running Linux\\
\texttt{x86\_64}&  Intel and AMD 64-bit running Linux - see note below\\
\texttt{crayxt4}& Cray XT4/XT5/XE machines\\
\texttt{mac686}&  Apple Intel machines running Mac OS X\\
\texttt{cygwin}&  MS-Windows PCs with Cygwin\\
\texttt{mingw32}& Cross-compilation for MS-Windows, using mingw, 32 bits\\
\texttt{mingw64}& As above, 64 bits\\
\texttt{necsx}&   NEC SX-6 and SX-8 machines\\
\texttt{ppc64}&   Linux PowerPC machines, 64 bits\\
\texttt{ppc64-mn}&as above, with IBM xlf compiler\\
\texttt{ppc64-bg}&IBM BlueGene\\
\texttt{arm}     &ARM machines (with gfortran or armflang)
\end{tabular}\\
{\em Note}: \texttt{x86\_64} replaces \texttt{amd64} since v.4.1.
Cray Unicos machines, SGI
machines with MIPS architecture, HP-Compaq Alphas are no longer supported
since v.4.2; PowerPC Macs are no longer
supported since v.5.0. IBM machines with AIX are no longer supported
since v.6.0.
Finally, \configure\ recognizes the following command-line options:\\
\begin{tabular}{ll}
\texttt{--enable-parallel}&     compile for parallel (MPI) execution if possible (default: yes)\\
\texttt{--enable-openmp}&       compile for OpenMP execution if possible (default: no)\\
\texttt{--enable-shared}&       use shared libraries if available (default: yes;\\
                        &       "no" is implemented, untested, in only a few cases)\\
\texttt{--enable-debug}&        compile with debug flags (only for selected cases; default: no)\\
\texttt{--enable-signals}&      enable signal trapping (default: disabled)\\
\end{tabular}\\
\\
and the following optional packages:\\
\begin{tabular}{ll}
\texttt{--with-internal-blas}&    compile with internal BLAS (default: no)\\
\texttt{--with-internal-lapack}&  compile with internal LAPACK (default: no)\\
\texttt{--with-scalapack}&        (yes$|$no$|$intel) Use scalapack if available. \\
      &Set to \texttt{intel} to use Intel MPI and blacs (default: use OpenMPI)\\
\texttt{--with-elpa-include}&   Specify full path of ELPA include and modules
  headers (default: no)\\
\texttt{--with-elpa-lib}& Specify full path of the ELPA library
	                          (default: no)\\
\texttt{--with-elpa-version}& Specify ELPA version, only year (2015 or 2016,
	                          default: 2016)\\
	\texttt{--with-hdf5}&  (no $|$ $|$ yes $|$ \texttt{<path>}) \\
	                    & Use HDF5, if yes configure assumes    \\
                            & that a valid installation with version >= 1.8.16 is \\
                            & available, and h5cc and h5fc are in the default   \\ 
                            & executable search path; <path> must be the root   \\ 
                            & folder of a standalone hdf5 installation. (default: no). \\
\texttt{--with-hdf5-libs=}  & Specify the linker options needed by HDF5 when  \\
                            & configure fails to detect them by itself. As value \\ 
                            & to specify here is usually composed by many \\ 
                            & substrings it should be enclosed by quotes so to \\
                            & prevent configure failures. (default: no)   \\
\texttt{--with-hdf5-include}& Specify full path the HDF5 include folder containing \\
                            & module and headers files. Use it if configure fails \\ 
                            & to detect the path by itself. (default: no)\\ 
\texttt{--with-libxc}        & Link the \libxc\ library (default:no) \\
\texttt{--with-libxc-prefix} & directory where \libxc\ is installed \\
\texttt{--with-libxc-include}& directory where \libxc\ Fortran headers reside\\
\end{tabular}\\
\\
The following options are available for the CUDA Fortran accelerated version
(currently in a separate package, not yet included in the main distribution):\\
\begin{tabular}{ll}
\texttt{--with-cuda=value}&         enables compilation of the CUDA Fortran\\
                          &         accelerated subroutines. \\
                          &         \texttt{value} should point the path where the CUDA toolkit \\
                          &         is installed, e.g. \texttt{/opt/cuda} (default: no)\\
\texttt{--with-cuda-cc=value}&      sets the compute capabilities for the compilation\\
                             &      of the accelerated subroutines. \\
                             &      \texttt{value} must be consistent with the hardware and the\\
                             &      NVidia driver installed on the workstation or on the\\
                             &      compute nodes of the HPC facility (default: 35)\\
\texttt{--with-cuda-runtime=value}& sets the version of the CUDA toolkit used \\
                                  & for the compilation of the accelerated code.\\
                                  & \texttt{value} must be consistent with the\\
                                  & CUDA Toolkit installed on the workstation \\
                                  & or available on the compute nodes of the HPC facility.\\
                                  & PGI compilers currently accept 7.5, 8.0 or 9.0 (default: 8.0)\\
\end{tabular}\\

Please note that in order to compile the CUDA Fortran code you need ...
the CUDA Fortran code (it is still separated from the main distribution)! 
you also need a recent version of the PGI Fortran compiler.
OpenMP must be enabled,
and you may want to use a CUDA-aware MPI distribution to optimize the data
transfer between the processes.

If you want to modify \configure\ (advanced users only!),
see the Developer Manual.

\subsubsection{Manual configuration}
\label{SubSec:manconf}
If \configure\ stops before the end, and you don't find a way to fix
it, you have to write a working \texttt{make.inc} file (optionally,
\texttt{include/c\_defs.h}). The template used by \configure\ is
\texttt{install/make.inc.in} and contains explanations of the meaning
of the various variables. Note that you may need
to select appropriate preprocessing flags
in conjunction with the desired or available
libraries (e.g. you need to add \texttt{-D\_\_FFTW} to \texttt{DFLAGS}
if you want to link internal FFTW). For a correct choice of preprocessing
flags, refer to the documentation in \texttt{include/defs.h.README}.

NOTA BENE: If you change any settings (e.g. preprocessing,
compilation flags)
after a previous (successful or failed) compilation, you must run
\texttt{make clean} before recompiling, unless you know exactly which
routines are affected by the changed settings and how to force their
recompilation. \configure\ will clean object and executables, unless
you use option \texttt{--save}.

\subsection{Libraries}
\label{Sec:Libraries}

\qe\ contains a copy of some needed external libraries:
\begin{itemize}
 \item iotk and FoX for reading and writing xml files;
 \item BLAS (\texttt{http://www.netlib.org/blas/}) and LAPACK
  (\texttt{http://www.netlib.org/lapack/}) for linear algebra;
\item FFTW (\texttt{http://www.fftw.org/}) for Fast Fourier Transforms.
\end{itemize}
Optimized vendor-specific libraries often yield huge performance gains
with respect to compiled libraries and should be used whenever
possible. \configure\ always try to locate the best mathematical
libraries.
\\

\paragraph{BLAS and LAPACK}
\qe\ can use any architecture-optimized BLAS and LAPACK replacements,
like those contained e.g. in the following libraries:
\begin{quote}
MKL for Intel CPUs\\
ACML for AMD CPUs\\
ESSL for IBM machines\\
\end{quote}

If none of these is available, we suggest that you use the optimized ATLAS
library: see \\
\texttt{http://math-atlas.sourceforge.net/}. Note that ATLAS is not
a complete replacement for LAPACK: it contains all of the BLAS, plus the
LU code, plus the full storage Cholesky code. Follow the instructions in the
ATLAS distributions to produce a full LAPACK replacement.

Sergei Lisenkov reported success and good performances with optimized
BLAS by Kazushige Goto. The library is now available under an
open-source license: see the GotoBLAS2 page at \\
\texttt{http://www.tacc.utexas.edu/tacc-software/gotoblas2/}.

\paragraph{FFT}
\qe\ has an internal copy of an old FFTW library. It also supports
the newer FFTW3 library and some vendor-specific FFT libraries.
\configure\ will first search for vendor-specific FFT libraries;
if none is found, it will search for an external FFTW v.3 library;
if none is found, it will fall back to the internal  copy of FFTW.
\configure\ will add the appropriate preprocessing options:
\begin{itemize}
\item \texttt{-D\_\_LINUX\_ESSL} for ESSL on IBM Linux machines,
\item \texttt{-DASL} for NEC ASL library on NEC machines,
\item \texttt{-D\_\_ARM\_LIB} for ARM Performance library,
\item \texttt{-D\_\_DFTI}  for DFTI (Intel MKL library),
\item \texttt{-D\_\_FFTW3} for FFTW3 (external),
\item \texttt{-D\_\_FFTW}  for FFTW (internal library),
\end{itemize}
to \texttt{DFLAGS} in the \texttt{make.inc} file.
If you edit \texttt{make.inc} manually, please note that one and
only one among the mentioned preprocessing option must be set.

If you have MKL libraries, you may either use the provided FFTW3
interface (v.10 and later), or directly link FFTW3 from MKL (v.12
and later) or use DFTI (recommended).

\paragraph{MPI libraries}
MPI libraries are usually needed for parallel execution, unless you are 
happy with OpenMP-only multicore parallelization.
In well-configured machines, \configure\ should find the appropriate
parallel compiler for you, and this should find the appropriate
libraries. Since often this doesn't
happen, especially on PC clusters, see Sec.\ref{SubSec:LinuxPCMPI}.

{\em Note:} since v.6.1, MPI libraries implementing v.3 of the standard
(notably, non-blocking broadcast and gather operations) are required.

\paragraph{Libraries for accelerators}
The accelerated version of the code uses standard CUDA libraries such as
\texttt{cublas, cufft, cusolver} and the eigensolver library explicitly 
developed for \qe{} by NVidia and distributed at \texttt{https://github.com/NVIDIA/Eigensolver\_gpu}.

\paragraph {HDF5}
The HDF5 library (\texttt{https://www.hdfgroup.org/downloads/hdf5/})
can be used to perform binary I/O using the  HDF5 format.  

The user may need to install \texttt{HDF5}  library, in this case have care to compile it
with the options \texttt{--enable-fortran}, \texttt{--enable-fortran2003},
and \texttt{--enable-parallel} (see below). These options must be passed
to the \configure\ script of the library, not of \qe. 

One can use either the \texttt{1.10} or \texttt{1.8} version of the library. 
For the latter the user has to download a version at least as new as \texttt{1.8.16}.  

To use the \texttt{HDF5} is usually sufficient to specify the path to the fortran compiler wrapper for
 \texttt{HDF5} (\texttt{h5fc} of \texttt{h5pfc} with the \texttt{--with-hdf5=} option of configure. If the wrapper is in the default path just use \texttt{--with-hdf5=yes}.  
 The configure script is usually able to extract the 
 linker options and the include directory path from the output of the wrapper. If it fails the user can
 use configure options \texttt{--with-hdf5-libs=<options>} and \texttt{--with-hdf5-include=<path>} for the linker options and include path respectively. 
These two latter options are often  needed when using the HDF5 packages 
provided by many LINUX distributions. In this case you may first try  first the \texttt{--with-hdf5=yes} 
option. If if it fails just type the command  \texttt{h5fc --show} ( or \texttt{h5pfc} if you are using paralel HDF5), the command will printout the linker and include options to be passed manually to the configure script.

The configure script is able to determine whether one is linking to a serial or parallel HDF5 library, and will  set the flag \texttt{-D\_\_HDF5\_SERIAL} in the \texttt{make.inc} accordingly. 



\paragraph{LIBXC}
\qe\ can use the \libxc\ library. You need to install \libxc\ first, then:
\verb|configure --with-libxc --with-libxc-prefix=... --with-libxc-include=...|.
You may look for "libxc" in \texttt{make.inc}  in case of trouble. Note that
currently only a (small) subset of functionals implemented in \libxc\ can be
used and that the \libxc-enabled version cannot use functionals from \qe.
This will change in the future.

\paragraph{Other libraries}
\qe\ can use the MASS vector math
library from IBM, if available (only on machines with XLF compiler:
likely obsolete).

\paragraph{If optimized libraries are not found}
The \configure\ script attempts to find optimized libraries, but may fail
if they have been installed in non-standard places. You should examine
the final value of \texttt{BLAS\_LIBS, LAPACK\_LIBS, FFT\_LIBS, MPI\_LIBS} (if needed),
\texttt{MASS\_LIBS} (IBM only), either in the output of \configure\ or in the generated
\texttt{make.inc}, to check whether it found all the libraries that you intend to use.

If some library was not found, you can specify a list of directories to search
in the environment variable \texttt{LIBDIRS},
and rerun \configure; directories in the
list must be separated by spaces. For example:
\begin{verbatim}
   ./configure LIBDIRS="/opt/intel/mkl70/lib/32 /usr/lib/math"
\end{verbatim}
If this still fails, you may set some or all of the \texttt{*\_LIBS} variables manually
and retry. For example:
\begin{verbatim}
   ./configure BLAS_LIBS="-L/usr/lib/math -lf77blas -latlas_sse"
\end{verbatim}
Beware that in this case, \configure\ will blindly accept the specified value,
and won't do any extra search.

\subsection{Compilation}
\label{SubSec:Compilation}

The compiled codes can run with any input: almost all variables are
dinamically allocated at run-time. Only a few variables have fixed
dimensions, set in file \texttt{Modules/parameters.f90}:
\begin{verbatim}
      ntypx  = 10,     &! max number of different types of atom
      npsx   = ntypx,  &! max number of different PPs (obsolete)
      nsx    = ntypx,  &! max number of atomic species (CP)
      npk    = 40000,  &! max number of k-points
      lmaxx  = 3,      &! max non local angular momentum (l=0 to lmaxx)
      lqmax= 2*lmaxx+1  ! max number of angular momenta of Q
\end{verbatim}
These values should work for the vast majority of cases. In case you need
more atomic types or more k-points, edit this file and recompile.

At your choice, you may compile the complete \qe\ suite of programs
(with \texttt{make all}), or only some specific programs.
\texttt{make} with no arguments yields a list of valid compilation targets:
\begin{itemize}
\item \texttt{make pw}  compiles the self-consistent-field package \PWscf
\item \texttt{make cp}  compiles the Car-Parrinello package \CP
\item \texttt{make neb} compiles the \NEB\ package. All executables are linked
			in main \texttt{bin} directory
\item \texttt{make ph}  compiles the \PHonon\ package. All executables are linked
			in main \texttt{bin} directory
\item \texttt{make pp}  compiles the postprocessing package \PostProc
\item \texttt{make pwcond} compiles the ballistic conductance package 
	                \texttt{PWcond}. All executables are linked
			in main \texttt{bin} directory
\item \texttt{make pwall} produces all of the above.
\item \texttt{make ld1}  compiles the pseudopotential generator package 
	                 \texttt{atomic}. All executables are linked
			in main \texttt{bin} directory
\item \texttt{make xspectra} compiles the package \texttt{XSpectra}.
			All executables are linked
			in main \texttt{bin} directory
\item \texttt{make upf} produces utilities for pseudopotential conversion in
                        directory \texttt{upftools/}
\item \texttt{make all} produces all of the above
\item \texttt{make epw} compiles package \texttt{EPW}
\item \texttt{make plumed} unpacks \texttt{PLUMED}, patches several routines
                           in \texttt{PW/}, \texttt{CPV/} and \texttt{clib/},
                           recompiles \PWscf\ and \CP\ with \texttt{PLUMED}
                           support
\item \texttt{make w90} downloads \texttt{wannier90}, unpacks it, copies an appropriate
                       \texttt{make.inc} file,  produces all executables
                       in \texttt{W90/wannier90.x} and in \texttt{bin/}
\item \texttt{make want} downloads \texttt{WanT}, unpacks it, runs its 
	                 \configure,
                         produces all executables for \texttt{WanT} in
                         \texttt{WANT/bin}.
\item \texttt{make yambo} downloads \texttt{yambo},
			  unpacks it, runs its \configure,
                          produces all \texttt{yambo} executables in
                          \texttt{YAMBO/bin}
\item \texttt{make gipaw} downloads \texttt{GIPAW},
                          unpacks it, runs its \configure,
                          produces all \texttt{GIPAW} executables in
                          \texttt{GIPAW/bin} and in main \texttt{bin} directory.
%\item \texttt{make west} downloads \texttt{WEST} from \texttt{www.west-code.org},
%                          unpacks it, produces all the executables
%                          in \texttt{West/Wfreq} and \texttt{West/Wstat}.
\end{itemize}
For the setup of the GUI, refer to the \texttt{PWgui-X.Y.Z /INSTALL} file, where
X.Y.Z stands for the version number of the GUI (should be the same as the
general version number). If you are using sources from the git repository, see
the \texttt{GUI/README} file instead.

If \texttt{make} refuses for some reason to download additional
packages, manually download them into subdirectory
\texttt{archive/}, {\em not unpacking or uncompressing them},
and try \texttt{make} again. Also see Sec.(\ref{SubSec:Download}).

\subsection{Running tests and examples}
\label{SubSec:Examples}

As a final check that compilation was successful, you may want to run some or
all of the tests and examples. 
Notice that most tests and examples are devised to be run serially
or on a small number of processors; do not use tests and examples
to benchmark parallelism, do not try to run on too many processors.

\subsubsection{Test-suite}
Automated tests give a "pass/fail" answer. All tests run quickly 
(less than a minute at most), but they are not meant to be realistic, 
just to test a specific case. Many features are tested but only for
the following codes: \pwx, \cpx, \phx, \epwx.
Instructions for the impatient:
\begin{verbatim}
    cd test-suite
    make run-tests
\end{verbatim}
Instructions for all others: go to the \texttt{test-suite/} directory,
read the \texttt{README} file, or at least, type \make. You may need
to edit the \texttt{run-XX.sh} shells, defining variables
\texttt{PARA\_PREFIX} and \texttt{PARA\_POSTFIX} (see below for their
meaning).

\subsubsection{Examples}
There are many examples and reference data for almost every piece of \qe,
but you have to manually inspect the results.

In order to use examples, you should edit file \texttt{environment\_variables},
setting the following variables as needed.
\begin{quote}
   BIN\_DIR: directory where executables reside\\
   PSEUDO\_DIR: directory where pseudopotential files reside\\
   TMP\_DIR: directory to be used as temporary storage area
\end{quote}
The default values of BIN\_DIR and PSEUDO\_DIR should be fine,
unless you have installed things in nonstandard places. TMP\_DIR
must be a directory where you have read and write access to, with
enough available space to host the temporary files produced by the
example runs, and possibly offering high I/O performance (i.e., don't
use an NFS-mounted directory). NOTA BENE: do not use a
directory containing other data: the examples will clean it!

If you have compiled the parallel version of \qe\ (this
is the default if parallel libraries are detected), you will usually
have to specify a launcher program (such as \texttt{mpirun} or
\texttt{mpiexec}) and the number of processors: see Sec.\ref{Sec:para} for
details. In order to do that, edit again the \texttt{environment\_variables}
file and set the \texttt{PARA\_PREFIX} and \texttt{PARA\_POSTFIX} variables
as needed. Parallel executables will be run by a command like this:
\begin{verbatim}
      $PARA_PREFIX pw.x $PARA_POSTFIX -i file.in > file.out
\end{verbatim}
For example, if the command line is like this (as for an IBM SP):
\begin{verbatim}
      poe pw.x -procs 4 -i file.in > file.out
\end{verbatim}
you should set \texttt{PARA\_PREFIX="poe"}, \texttt{PARA\_POSTFIX="-procs 4"}.
Furthermore, if your machine does not support interactive use, you
must run the commands specified above through the batch queuing
system installed on that machine. Ask your system administrator for
instructions. For execution using OpenMP on N threads,
use \texttt{PARA\_PREFIX="env OMP\_NUM\_THREADS=N ... "}.

To run an example, go to the corresponding directory (e.g.
 \texttt{PW/examples/example01}) and execute:
\begin{verbatim}
      ./run_example
\end{verbatim}
This will create a subdirectory \texttt{results/}, containing the input and
output files generated by the calculation. Some examples take only a
few seconds to run, while others may require up to several minutes.

In each example's directory, the \texttt{reference/} subdirectory contains
verified output files, that you can check your results against. They
were generated on a Linux PC using the Intel compiler. On different
architectures the precise numbers could be slightly different, in
particular if different FFT dimensions are automatically selected. For
this reason, a plain diff of your results against the reference data
doesn't work, or at least, it requires human inspection of the results.

The example scripts stop if an error is detected. You should look {\em inside}
the last written output file to understand why.

\subsection{Installation tricks and problems}

\subsubsection{All architectures}
\begin{itemize}
\item
Working Fortran and C compilers, compliant with F2003 and C89 standards
(see Sec.\ref{Sec:Installation})
respectively, are needed in order to compile \qe. Most recent Fortran
compilers will do the job.

C and Fortran compilers must be in your PATH.
If \configure\ says that you have no working compiler, well,
you have no working compiler, at least not in your PATH, and
not among those recognized by \configure.
\item
If you get {\em Compiler Internal Error} or similar messages: your
compiler version is buggy. Try to lower the optimization level, or to
remove optimization just for the routine that has problems. If it
doesn't work, or if you experience weird problems at run time, try to
install patches for your version of the compiler (most vendors release
at least a few patches for free), or to upgrade to a more recent
compiler version.
\item
If you get error messages at the loading phase that look like
{\em file XYZ.o: unknown / not recognized/ invalid / wrong
file type / file format / module version},
one of the following things have happened:
\begin{enumerate}
\item you have leftover object files from a compilation with another
  compiler: run \texttt{make clean} and recompile.
\item \make\ did not stop at the first compilation error (it may
happen in some software configurations). Remove the file *.o
that triggers the error message, recompile, look for a
compilation error.
\end{enumerate}
If many symbols are missing in the loading phase: you did not specify the
location of all needed libraries (LAPACK, BLAS, FFTW, machine-specific
optimized libraries), in the needed order.
Note that \qe\ is self-contained (with the exception of MPI libraries for
parallel compilation): if system libraries are missing, the problem is in
your compiler/library combination or in their usage, not in \qe.
\item
If you get {\em Segmentation fault} or similar errors
in the provided tests and examples: your compiler, or
your mathematical libraries, or MPI libraries,
or a combination thereof, is buggy, or there is some
software incompatibility. Although one can never rule out
the presence of subtle bugs in \qe\ that are not revealed during
the testing phase, it is very unlikely
that this happens on the provided tests and examples.
\item
If all test fails, look into the output and error files:
there is some dumb reason for failure.
\item
If most test pass but some fail, again: look into the output
and error files. A frequent source of trouble is complex function 
\texttt{zdotc}. See the "Linux PCs with gfortran compilers" paragraph, 
or replace \texttt{zdotc} with fortran intrinsic \texttt{dot\_product}.
\end{itemize}


\subsubsection{Linux PC}

Both AMD and Intel CPUs, 32-bit and 64-bit, are supported and work,
either in 32-bit emulation and in 64-bit mode. 64-bit executables
can address a much larger memory space than 32-bit executable, but
there is no gain in speed.
Beware: the default integer type for 64-bit machine is typically
32-bit long. You should be able to use 64-bit integers as well,
but it is not guaranteed to work and will not give
any advantage anyway.

Currently, \configure\ supports Intel (ifort), NAG (nagfor), PGI (pgf90)
and gfortran compilers. Pathscale, Sun Studio, AMD Open64, are no
longer supported after v.6.2: g95, since v.6.1.

Both Intel MKL and AMD acml mathematical libraries are supported, the
former much better than the latter.

It is usually convenient to create semi-statically linked executables (with only
libc, libm, libpthread dynamically linked). If you want to produce a binary
that runs on different machines, compile it on the oldest machine you have
(i.e. the one with the oldest version of the operating system).

\paragraph{Linux PCs with gfortran}

You need at least gfortran v.4.4 or later to properly compile \qe.

"There is a known incompatibility problem between different calling
convention for Fortran functions that return complex values [...]
If your code crashes during a call to \texttt{zdotc},
recompile \qe\ using the internal BLAS and LAPACK routines 
(using \configure\ options \texttt{--with-internal-blas} and
\texttt{--with-internal-lapack})
to see if the problem disappears; or, add the \texttt{-ff2c} flag"
(info by Giovanni Pizzi, Jan. 2013).

If you want to use MKL libraries together with gfortran, 
link \texttt{-lmkl\_gf\_lp64}, not \texttt{-lmkl\_intel\_lp64}
(and the like for other architectures).

\paragraph{Linux PCs with Intel compiler (ifort)}

IMPORTANT NOTE: ifort versions earlier than v.15 miscompile the new
XML code in QE v.6.4 and later. Please install this patch:\\
\texttt{https://gitlab.com/QEF/q-e/wikis/Support/Patch-for-old-Intel-compilers}.

The Intel compiler ifort \texttt{http://software.intel.com/}
produces fast executables, at least on Intel CPUs, but not all versions
work as expected. In case of trouble, update your version
with the most recent patches. Since each major release of ifort
differs a lot from the previous one, compiled objects from different
releases may be incompatible and should not be mixed.

The Intel compiler is no longer free for personal usage, but it is still
for students and open-source contributors
(\texttt{https://software.intel.com/en-us/qualify-for-free-software}).

If \configure\ doesn't find the compiler, or if you get
{\em Error loading shared libraries} at run time, you may have
forgotten to execute the script that
sets up the correct PATH and library path. Unless your system manager has
done this for you, you should execute the appropriate script -- located in
the directory containing the compiler executable -- in your
initialization files. Consult the documentation provided by Intel.

The warning: {\em feupdateenv is not implemented and will always fail},
can be safely ignored. Warnings on ``bad preprocessing option'' when compiling
iotk and complains about ``recommended formats'' may also be ignored.

\paragraph{Linux PCs with MKL libraries}
On Intel CPUs it is very convenient to use Intel MKL libraries
(freely available at
\texttt{https://software.intel.com/en-us/performance-libraries}).
MKL libraries can be used also with non-Intel compilers.
They work also for AMD CPU, selecting the appropriate machine-optimized
libraries, but with reduced performances.

\configure\ properly detects only recent (v.12 or later) MKL libraries,
as long as the \$MKLROOT environment variable is set in the current shell.
Normally this environment variable is set by sourcing the Intel MKL or Intel 
Parallel Studio environment script.
By default the non-threaded version of MKL is linked, unless option
\texttt{configure --with-openmp} is specified. In case of trouble,
refer to the following web page to find the correct way to link MKL:\\
\texttt{http://software.intel.com/en-us/articles/intel-mkl-link-line-advisor/}.

For parallel (MPI) execution on multiprocessor (SMP) machines, set the
environment variable OMP\_NUM\_THREADS to 1 unless you know what you
are doing. See Sec.\ref{Sec:para} for more info on this
and on the difference between MPI and OpenMP parallelization.

If you get a mysterious "too many communicators" error and a 
subsequent crash: there is a bug in Intel MPI and MKL 2016 update 3.
See this thread and the links quoted therein:
\verb|http://www.mail-archive.com/pw_forum@pwscf.org/msg29684.html|.

\paragraph{Linux PCs with ACML libraries}
For AMD CPUs, especially recent ones, you may find convenient to
link AMD acml libraries (can be freely downloaded from AMD web site).
\configure\ should recognize properly installed acml libraries.

\subsubsection{Linux PC clusters with MPI}
\label{SubSec:LinuxPCMPI}
PC clusters running some version of MPI are a very popular
computational platform nowadays. \qe\ is known to work
with at least two of the major MPI implementations (MPICH, LAM-MPI),
plus with the newer MPICH2 and OpenMPI implementation.
\configure\ should automatically recognize a properly installed
parallel environment and prepare for parallel compilation.
Unfortunately this not always happens. In fact:
\begin{itemize}
\item \configure\ tries to locate a parallel compiler in a logical
  place with a logical name,  but if it has a strange names or it is
  located  in a strange location, you will have to instruct \configure\
  to find it. Note that in many PC clusters (Beowulf), there is no
  parallel Fortran compiler in default installations:  you have to
  configure an appropriate script, such as mpif90.
\item \configure\ tries to locate libraries (both mathematical and
  parallel libraries) in the usual places with usual names, but if
  they have strange names or strange locations, you will have to
  rename/move them, or to instruct \configure\ to find them. If MPI
  libraries are not found,
  parallel compilation is disabled.
\item \configure\ tests that the compiler and the libraries are
  compatible (i.e. the compiler may link the libraries without
  conflicts and without missing symbols). If they aren't and the
  compilation fails, \configure\ will revert to serial compilation.
\end{itemize}

Apart from such problems, \qe\ compiles and works on all non-buggy, properly
configured hardware and software combinations. In some cases you may have to
recompile MPI libraries: not all MPI installations contain support for
the Fortran compiler of your choice (or for any Fortran compiler
at all!).

If \qe\ does not work for some reason on a PC cluster,
try first if it works in serial execution. A frequent problem with parallel
execution is that \qe\ does not read from standard input,
due to the configuration of MPI libraries: see Sec.\ref{SubSec:badpara}.
If you are dissatisfied with the performances in parallel execution,
see Sec.\ref{Sec:para} and in particular Sec.\ref{SubSec:badpara}.

\subsubsection{Microsoft Windows}
\label{SubSec:Windows}
Since February 2020 \qe\ can be compiled on MS-Windows 10 using PGI 19.10 
Community Edition (freely downloadable). \configure\ works with the bash
script provided by PGI but the \configure\ of FoX fails: use script 
\texttt{install/build\_fox\_with\_pgi.sh} to manually compile FoX.

Another option: use MinGW/MSYS. Download the installer from
\texttt{https://osdn.net/projects/mingw/}, install MinGW, MSYS, gcc and
gfortran. Start a shell window; run "./configure"; edit \texttt{make.inc}; 
uncommenting the second definition of TOPDIR (the first one introduces a 
final "/" that Windows doesn't like); run "make". Note that on some Windows 
the code fails when checking that \texttt{tmp\_dir} is writable, for unclear 
reasons.

Another option is Cygwin, a UNIX environment which runs under Windows: see\\
\texttt{http://www.cygwin.com/}. 

Windows-10 users may also enable the Windows Subsystem for Linux (see here:\\
\texttt{https://docs.microsoft.com/en-us/windows/wsl/install-win10}),
install a Linux distribution, compile \qe\ as on Linux. It works very well.

As a final option, one can use Quantum Mobile:\\
\texttt{https://www.materialscloud.org/work/quantum-mobile}.

\subsubsection{Mac OS}

Mac OS-X machines with gfortran or with the Intel compiler ifort
and MKL libraries should work, but "your mileage may vary", depending
upon the specific software stack you are using. Parallel compilation 
with OpenMPI should also work. 

Gfortran information and binaries for Mac OS-X here:
\texttt{http://hpc.sourceforge.net/} and
\texttt{https://wiki.helsinki.fi/display/HUGG/GNU+compiler+install+on+Mac+OS+X}.

Mysterious crashes in \texttt{zdotc} are due to a known incompatibility of 
complex functions with some optimized BLAS. See the "Linux PCs with gfortran" 
paragraph.

\subsection{Cray machines}

For Cray XE machines:
\begin{verbatim}
$ module swap PrgEnv-cray PrgEnv-pgi
$ ./configure --enable-openmp --enable-parallel --with-scalapack
$ vim make.inc
\end{verbatim}
then manually add \texttt{-D\_\_IOTK\_WORKAROUND1} at the end of \texttt{DFLAGS} line.

''Now, despite what people can imagine, every CRAY machine deployed can
have different environment. For example on the machine I usually use
for tests [...] I do have to unload some modules to make QE running
properly. On another CRAY [...] there is also Intel compiler as option
and the system is slightly different compared to the other.
So my recipe should work, 99\% of the cases.'' (info by Filippo Spiga)

For Cray XT machines, use \texttt{./configure ARCH=crayxt4} or else
\configure\ will not recognize the Cray-specific software environment.

Older Cray machines: T3D, T3E, X1, are no longer supported.

\subsubsection{Obsolescent architectures}

\paragraph{Intel Xeon Phi}

For Intel Xeon CPUs with Phi coprocessor, see this link:\\
\texttt{https://software.intel.com/en-us/articles/explicit-offload-for-quantum-espresso}.

There are three ways of compiling:
\begin{itemize}
\item {\em offload} mode, executed on main CPU and offloaded onto coprocessor
"automagically";
\item {\em native} mode, executed completely on coprocessor;
\item {\em symmetric} mode, requiring creation of both binaries.
\end{itemize}
"You can take advantage of the offload mode using the \texttt{libxphi}
library. This library offloads the BLAS/MKL functions on the Xeon Phi
platform hiding the latency times due to the communication. You just
need to compile this library and then to link it dynamically. The
library works with any version of QE. Libxphi is available from
\texttt{https://github.com/cdahnken/libxphi}. Some documentation is
available therein.

Instead, if you want to compile a native version of QE, you just need
to add the \texttt{-mmic} flag and cross compile. If you want to use
the symmetric mode, you need to compile twice: with and without the
\texttt{-mmic} flag". "[...] everything, i.e. code+libraries, must be
cross-compiled with the \texttt{-mmic} flag. In my opinion, it's pretty
unlikely that native mode can outperform the execution on the standard
Xeon cpu. I strongly suggest to use the Xeon Phi in offload mode, for now"
(info by Fabio Affinito, March 2015).

\paragraph{IBM BlueGene}

The current \configure\ was working on the machines at CINECA and at J\"ulich.
For other machines, you may need something like
\begin{verbatim}
  ./configure ARCH=ppc64-bg BLAS_LIBS=...  LAPACK_LIBS=... \
              SCALAPACK_DIR=... BLACS_DIR=..."
\end{verbatim}
where the various *\_LIBS and *\_DIR "suggest" where the various libraries
are located.

\newpage

\section{Parallelism}
\label{Sec:para}

\subsection{Understanding Parallelism}

Two different parallelization paradigms are currently implemented
in \qe:
\begin{enumerate}
\item {\em Message-Passing (MPI)}. A copy of the executable runs
on each CPU; each copy lives in a different world, with its own
private set of data, and communicates with other executables only
via calls to MPI libraries. MPI parallelization requires compilation
for parallel execution, linking with MPI libraries, execution using
a launcher program (depending upon the specific machine). The number of CPUs used
is specified at run-time either as an option to the launcher or
by the batch queue system.
\item {\em OpenMP}.  A single executable spawn subprocesses
(threads) that perform in parallel specific tasks.
OpenMP can be implemented via compiler directives ({\em explicit}
OpenMP) or via {\em multithreading} libraries  ({\em library} OpenMP).
Explicit OpenMP require compilation for OpenMP execution;
library OpenMP requires only linking to a multithreading
version of mathematical libraries, e.g.:
ESSLSMP, ACML\_MP, MKL (the latter is natively multithreading).
The number of threads is specified at run-time in the environment
variable OMP\_NUM\_THREADS.
\end{enumerate}

MPI is the well-established, general-purpose parallelization.
In \qe\ several parallelization levels, specified at run-time
via command-line options to the executable, are implemented
with MPI. This is your first choice for execution on a parallel
machine.

The support for explicit OpenMP is steadily improving.
Explicit OpenMP can be used together with MPI and also 
together with library OpenMP. Beware
conflicts between the various kinds of parallelization!
If you don't know how to run MPI processes
and OpenMP threads in a controlled manner, forget about mixed
OpenMP-MPI parallelization.

\subsection{Running on parallel machines}

Parallel execution is strongly system- and installation-dependent.
Typically one has to specify:
\begin{enumerate}
\item a launcher program such as \texttt{mpirun} or \texttt{mpiexec},
  with the  appropriate options (if any);
\item the number of processors, typically as an option to the launcher
  program;
\item the program to be executed, with the proper path if needed;
\item other \qe-specific parallelization options, to be
  read and interpreted by the running code.
\end{enumerate}
Items 1) and 2) are machine- and installation-dependent, and may be
different for interactive and batch execution. Note that large
parallel machines are  often configured so as to disallow interactive
execution: if in doubt, ask your system administrator.
Item 3) also depend on your specific configuration (shell, execution path, etc).
Item 4) is optional but it is very important
for good performances. We refer to the next
section for a description of the various
possibilities.

\subsection{Parallelization levels}

In \qe\ several MPI parallelization levels are
implemented, in which both calculations
and data structures are distributed across processors.
Processors are organized in a hierarchy of groups,
which are identified by different MPI communicators level.
The groups hierarchy is as follow:
\begin{itemize}
\item {\bf world}: is the group of all processors (MPI\_COMM\_WORLD).
\item
{\bf images}: Processors can then be divided into different "images", each corresponding to a
different self-consistent or linear-response
calculation, loosely coupled to others.
\item
{\bf pools}: each image can be subpartitioned into
"pools", each taking care of a group of k-points.
\item
{\bf bands}: each pool is subpartitioned into
"band groups", each taking care of a group
of Kohn-Sham orbitals (also called bands, or
wavefunctions). Especially useful for calculations
with hybrid functionals.
\item
{\bf PW}: orbitals in the PW basis set,
as well as charges and density in either
reciprocal or real space, are distributed
across processors.
This is usually referred to as "PW parallelization".
All linear-algebra operations on array of  PW /
real-space grids are automatically and effectively parallelized.
3D FFT is used to transform electronic wave functions from
reciprocal to real space and vice versa. The 3D FFT is
parallelized by distributing planes of the 3D grid in real
space to processors (in reciprocal space, it is columns of
G-vectors that are distributed to processors).
\item
{\bf tasks}:
In order to allow good parallelization of the 3D FFT when
the number of processors exceeds the number of FFT planes,
FFTs on Kohn-Sham states are redistributed to
``task'' groups so that each group
can process several wavefunctions at the same time.
Alternatively, when this is not possible, a further
subdivision of FFT planes is performed.
\item
{\bf linear-algebra group}:
A further level of parallelization, independent on
PW or k-point parallelization, is the parallelization of
subspace diagonalization / iterative orthonormalization.
 Both operations required the diagonalization of
arrays whose dimension is the number of Kohn-Sham states
(or a small multiple of it). All such arrays are distributed block-like
across the ``linear-algebra group'', a subgroup of the pool of processors,
organized in a square 2D grid. As a consequence the number of processors
in the linear-algebra group is given by $n^2$, where $n$ is an integer;
$n^2$ must be smaller than the number of processors in the PW group.
The diagonalization is then performed
in parallel using standard linear algebra operations.
(This diagonalization is used by, but should not be confused with,
the iterative Davidson algorithm). The preferred option is to use
ELPA and ScaLAPACK; alternative built-in algorithms are anyway available.
\end{itemize}
Note however that not all parallelization levels
are implemented in all codes.

\paragraph{About communications}
Images and pools are loosely coupled: inter-processors communication
between different images and pools is modest. Processors within each 
pool are instead tightly coupled and communications are significant. 
This means that fast communication hardware is needed if
your pool extends over more than a few processors on different nodes.

\paragraph{Choosing parameters}:
To control the number of processors in each group,
command line switches:
\texttt{-nimage}, \texttt{-npools}, \texttt{-nband},
\texttt{-ntg}, \texttt{-ndiag} or \texttt{-northo}
(shorthands, respectively: \texttt{-ni}, \texttt{-nk}, \texttt{-nb},
\texttt{-nt}, \texttt{-nd})
are used.
As an example consider the following command line:
\begin{verbatim}
mpirun -np 4096 ./neb.x -ni 8 -nk 2 -nt 4 -nd 144 -i my.input
\end{verbatim}
This executes a NEB calculation on 4096 processors, 8 images (points in the configuration
space in this case) at the same time, each of
which is distributed across 512 processors.
k-points are distributed across 2 pools of 256 processors each,
3D FFT is performed using 4 task groups (64 processors each, so
the 3D real-space grid is cut into 64 slices), and the diagonalization
of the subspace Hamiltonian is distributed to a square grid of 144
processors (12x12).

Default values are: \texttt{-ni 1 -nk 1 -nt 1} ;
\texttt{nd} is set to 1 if ScaLAPACK is not compiled,
it is set to the square integer smaller than or equal to  half the number
of processors of each pool.

\paragraph{Massively parallel calculations}
For very large jobs (i.e. O(1000) atoms or more) or for very long jobs,
to be run on massively parallel  machines (e.g. IBM BlueGene) it is
crucial to use in an effective way all available parallelization levels:
on linear algebra (requires compilation with ELPA and/or ScaLAPACK), 
on "task groups" (requires run-time option "-nt N"), and mixed
MPI-OpenMP (requires OpenMP compilation: \configure --enable-openmp).
Without a judicious choice of parameters, large jobs will find a
stumbling block in either memory or CPU requirements. Note that I/O
may also become a limiting factor.

\subsubsection{Understanding parallel I/O}
In parallel execution, each processor has its own slice of data
(Kohn-Sham orbitals, charge density, etc), that have to be written
to temporary files during the calculation,
or to data files at the end of the calculation.
This can be done in two different ways:
\begin{itemize}
\item ``collected'': all slices are
collected by the code to a single processor
that writes them to disk, in a single file,
using a format that doesn't depend upon
the number of processors or their distribution.
This is the default since v.6.2 for final data.
\item ``portable'': as above, but data can be 
copied to and read from a different machines
(this is not guaranteed with Fortran binary files).
Requires compilation with \verb|-D__HDF5|
preprocessing option and HDF% libraries.
\end{itemize}
There is a third format, no longer used for final
data but used for scratch and restart files:
\begin{itemize}
\item ``distributed'': each processor
writes its own slice to disk in its internal
format to a different file.
The ``distributed'' format is fast and simple,
but the data so produced is readable only by
a job running on the same number of processors,
with the same type of parallelization, as the
job who wrote the data, and if all
files are on a file system that is visible to all
processors (i.e., you cannot use local scratch
directories: there is presently no way to ensure
that the distribution of processes across
processors will follow the same pattern
for different jobs).
\end{itemize}

The directory for data is specified in input variables
\texttt{outdir} and \texttt{prefix} (the former can be specified
as well in environment variable ESPRESSO\_TMPDIR):
\texttt{outdir/prefix.save}. A copy of pseudopotential files
is also written there. If some processor cannot access the
data directory, the pseudopotential files are read instead
from the pseudopotential directory specified in input data.
Unpredictable results may follow if those files
are not the same as those in the data directory!

{\em IMPORTANT:}
Avoid I/O to network-mounted disks (via NFS) as much as you can!
Ideally the scratch directory \texttt{outdir} should be a modern
Parallel File System. If you do not have any, you can use local
scratch disks (i.e. each node is physically connected to a disk
and writes to it) but you may run into trouble anyway if you
need to access your files that are scattered in an unpredictable
way across disks residing on different nodes.

You can use input variable \texttt{disk\_io} to vary the
amount of I/O done by \pwx. Since v.5.1, the dafault value is
\texttt{disk\_io='low'}, so the code will store wavefunctions
into RAM and not on disk during the calculation. Specify
\texttt{disk\_io='medium'} only if you have too many k-points
and you run into trouble with memory; choose \texttt{disk\_io='none'}
if you do not need to keep final data files.

% For very large \cpx\ runs, you may consider using
% \texttt{wf\_collect=.false.}, \texttt{memory='small'} and
% \texttt{saverho=.false.} to reduce I/O to the strict minimum.

\subsection{Tricks and problems}
\label{SubSec:badpara}

Many problems in parallel execution derive from the mixup of different
MPI libraries and runtime environments. There are two major MPI
implementations, OpenMPI and MPICH, coming in various versions,
not necessarily compatible; plus vendor-specific implementations
(e.g. Intel MPI). A parallel machine may have multiple parallel
compilers (typically, \texttt{mpif90} scripts calling different
serial compilers), multiple MPI libraries, multiple launchers
for parallel codes (different versions of \texttt{mpirun} and/or
\texttt{mpiexec}). You have to figure out the proper combination
of all of the above, which may require using command \texttt{module}
or manually setting environment variables and execution paths.
What exactly has to be done depends upon the configuration of your
machine. You should inquire with your system administrator or user
support (if available; if not, YOU are the system administrator
and user support and YOU have to solve your problems).

Always verify if your executable is actually compiled for
parallel execution or not: it is declared in the first lines
of output. Running several instances of a serial code with
\texttt{mpirun} or \texttt{mpiexec} produces strange crashes.

\paragraph{Trouble with input files}
Some implementations of the MPI library have problems with input
redirection in parallel. This typically shows up under the form of
mysterious errors when reading data. If this happens, use the option
\texttt{-i} (or \texttt{-in}, \texttt{-inp}, \texttt{-input}),
followed by the input file name.
Example:
\begin{verbatim}
   pw.x -i inputfile -nk 4 > outputfile
\end{verbatim}
Of course the
input file must be accessible by the processor that must read it
(only one processor reads the input file and subsequently broadcasts
its contents to all other processors).

Apparently the LSF implementation of MPI libraries manages to ignore or to
confuse even the \texttt{-i/in/inp/input} mechanism that is present in all
\qe\ codes. In this case, use the \texttt{-i} option of \texttt{mpirun.lsf}
to provide an input file.

\paragraph{Trouble with MKL and MPI parallelization}
If you notice very bad parallel performances with MPI and MKL libraries,
it is very likely that the OpenMP parallelization performed by the latter
is colliding with MPI. Recent versions of MKL enable autoparallelization
by default on multicore machines.  You must set the environment variable
OMP\_NUM\_THREADS to 1 to disable it.
Note that if for some reason the correct setting  of variable
OMP\_NUM\_THREADS
does not propagate to all processors, you may equally run into trouble.
Lorenzo Paulatto (Nov. 2008) suggests to use the \texttt{-x} option to \texttt{mpirun} to
propagate OMP\_NUM\_THREADS to all processors.
Axel Kohlmeyer suggests the following (April 2008):
"(I've) found that Intel is now turning on multithreading without any
warning and that is for example why their FFT seems faster than
FFTW. For serial and OpenMP based runs this makes no difference (in
fact the multi-threaded FFT helps), but if you run MPI locally, you
actually lose performance. Also if you use the 'numactl' tool on linux
to bind a job to a specific cpu core, MKL will still try to use all
available cores (and slow down badly). The cleanest way of avoiding
this mess is to either link with
\begin{quote}
\texttt{-lmkl\_intel\_lp64 -lmkl\_sequential -lmkl\_core} (on 64-bit:
x86\_64, ia64)\\
\texttt{-lmkl\_intel -lmkl\_sequential -lmkl\_core} (on 32-bit, i.e. ia32 )
\end{quote}
or edit the \texttt{libmkl\_'platform'.a} file. I'm using now a file
\texttt{libmkl10.a} with:
\begin{verbatim}
  GROUP (libmkl_intel_lp64.a libmkl_sequential.a libmkl_core.a)
\end{verbatim}
It works like a charm". UPDATE: Since v.4.2, \configure\ links by
default MKL without multithreaded support.

\paragraph{Trouble with compilers and MPI libraries}
Many users of \qe, in particular those working on PC clusters,
have to rely on themselves (or on less-than-adequate system managers) for
the correct configuration of software for parallel execution. Mysterious and
irreproducible crashes in parallel execution are sometimes due to bugs
in \qe, but more often than not are a consequence of buggy
compilers or of buggy or miscompiled MPI libraries.

\end{document}
