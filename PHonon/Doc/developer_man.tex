\documentclass[12pt,a4paper]{article}
\def\version{6.6}
\def\qe{{\sc Quantum ESPRESSO}}
\textwidth = 17cm
\textheight = 24cm
\topmargin =-1 cm
\oddsidemargin = 0 cm

\usepackage{html}

% BEWARE: don't revert from graphicx for epsfig, because latex2html
% doesn't handle epsfig commands !!!
\usepackage{graphicx}


% \def\htmladdnormallink#1#2{#1}

\def\configure{\texttt{configure}}
\def\configurac{\texttt{configure.ac}}
\def\autoconf{\texttt{autoconf}}

\def\qeImage{../../Doc/quantum_espresso}

\def\pwx{\texttt{pw.x}}
\def\phx{\texttt{ph.x}}
\def\configure{\texttt{configure}}
\def\PWscf{\texttt{PWscf}}
\def\PHonon{\texttt{PHonon}}
\def\make{\texttt{make}}


\begin{document} 
\author{}
\date{}
\title{
  \includegraphics[width=5cm]{\qeImage} \\
  % title
  \Huge Developers' Manual for \PHonon\ (\version)
    \\ \Large (only partially updated)
}
\maketitle

\tableofcontents

\newpage

\section{Introduction}

{\em Important notice: due to the lack of time and of manpower, this
manual is only partially updated and may contain outdated information.}

\subsection{Who should read (and who should {\em write}) this guide}

The intended audience of this guide is everybody who wants to:
\begin{itemize}
\item know how the \PHonon\ package works, including its internals;
\item modify/customize/add/extend/improve/clean up the \PHonon\ package;
\item know how to read data produced by the \PHonon\ package.
\end{itemize}
The same category of people should also {\em write} this guide, of course.

\subsection{Who may read this guide but will not necessarily profit from it}

People who want to know about the capabilities of the \PHonon\ package.

People who want to know about the methods or the physics
behind \PHonon\ should read first the relevant  
literature (some pointers in the User Guide and in the Bibliography).


\section{General structure of \phx}

The behavior of the \phx\ code is controlled by a set of flags.
In a general run when all control flags are \texttt{.true.} the phonon 
code computes the following quantities in the given order:

\begin{verbatim}
                              frequency             q        perturbations

polarizability                   iu                 gamma       x,y,z 
dielectric constant               0                 gamma       x,y,z
zeu                               0                 gamma       x,y,z  
electro optic coefficient         0                 gamma       x,y,x 
raman tensor                      0                 gamma       3 x 3
dynamical matrix                  0                 all q      all irreps
zue                               0                 gamma      all irreps
electron phonon interactions      0                 all q      all irreps

zeu = Born effective charges as derivative of the forces,
zue = Born effective charges as derivative of the polarization 
\end{verbatim}

Two control flags associated to every calculated quantity 
allow to set/unset the calculation of that quantity independently from 
the others. One of these flags is an input variable:

\begin{verbatim}
fpol,             if .TRUE. computes the frequency dependent polarizability
epsil,            if .TRUE. computes the dielectric constant
zeu,              if .TRUE. computes eff. charges as induced forces
lraman,           if .TRUE. computes the raman tensor
elop,             if .TRUE. computes the el-optical coefficient
trans,            if .TRUE. computes the dynamical matrix
zue,              if .TRUE. computes eff. charges as induced polarization
elph              if .TRUE. computes the electron phonon coupling
\end{verbatim}

By default, only the \texttt{trans} flag is \texttt{.true.}.
The second flag is described in the following Section.

The phonon code contains three loops.
The outer loop is over {\bf q} points. The other two loops are inside the
{\bf q}-point loop, but they are separate and carried out sequentially. 
There is a loop over the frequencies that calculates the frequency 
dependent polarizabilities and a loop over the irreducible 
representations (\texttt{irreps}). 
In addition to this there is the calculation of the response to the electric
field. The loop over the frequencies and the response to an electric field are 
calculated only if {\bf q} is the $\Gamma$ point. The size of the loops over
the frequencies and over {\bf q} points is controlled by input variables.  

\begin{verbatim}
nfs               ! number of frequencies
fiu(nfs)          ! frequencies in Ry

nq1, nq2, nq3     ! the mesh of q points
or
xq                ! the coordinates of a q point

start_q           ! initial q to calculate
last_q            ! last q to calculate
start_irr         ! initial representation to calculate
last_irr          ! last representation to calculate
\end{verbatim}

The run can be controlled also in other two ways by the following input
variables:

\begin{verbatim}
nat_todo          ! the number of atoms to move
atomo(nat_todo)   ! which atoms to move

or

modenum           ! the response to a single mode
\end{verbatim}
The first two options limit the calculation to the representations in which
at least one of a set of atoms (specified by \texttt{atomo}) moves.
The second option calculates only the motion with respect to one 
vibrational mode.

The flow of the code can be summarized as follows:

\begin{verbatim}
1) Read input and set the flags of the quantities to compute
   1.1) Read all the quantities written by pw.x
   1.2) Read the pseudopotential data

2) Decide what must be calculated.
   2.1) If not already on disk, compute the grid of q points and 
        all the modes for all q points and save on disk (SD)
   2.2) If image parallelization is requested divide the work among images

3) In a recover run check what is already available on the .xml files and
   sets the appropriate done flags to .TRUE.

4) Start a main loop over the q points:

   4.1) Compute all quantities that do not depend on the response of the system
   4.2) Check if a band calculation is needed and do it.
   NB: the following points are executed only when q is Gamma.
     4.3) Start a loop on the frequencies
          4.3.1) Compute the polarizability as a function of iu SD
     4.4) Compute the response to an electric field 
     4.5) Compute epsilon and SD
     4.6) Compute zeu and SD
     4.7) Compute the electro-optic coefficient and SD
     4.8) Compute the second order response to E
     4.9) Compute Raman tensor and SD
   END NB

5) Start a loop over the irreducible representation 
     5.1) Compute the response to an irreducible representation
     5.1.1) Accumulate the contribution to electron-phonon SD
     5.1.2) Accumulate the contribution to the dynamical matrix 
     5.1.3) Accumulate the contribution to zue 
     5.1.4) SD this contribution to the dynamical matrix and to zue
continue the loop 5) until all representations of the current q point
have been computed

6) diagonalize the dynamical matrix and SD (only if all representations of 
   this q have been computed)

7) Sum over k and bands the electron-phonon couplings to calculate gamma_mat
   SD (only if all representations of this q have been computed)

8) continue the loop at point 4 until all q points have been computed

\end{verbatim}

In more detail the quantities calculated by the phonon code and
the routines where these quantities are calculated are:

\begin{itemize}

\item
4.2.1) The polarization as a function of the complex frequency is a
\texttt{3x3} real tensor for each frequency: \texttt{polar(3,3,nfs)} 
(calculated in \texttt{polariz}). These quantities are presently written 
on output.

\item
4.5) The dielectric constant is a real \texttt{3x3} tensor: 
\texttt{epsilon} (calculated in \texttt{dielec}). 

\item
4.6) Zeu is a real array: \texttt{zstareu(3,3,nat)}. The first index is 
the electric field, while the other two indices give the atom that moves and the
direction. 

\item
The electro-optic tensor is a three indices tensor \texttt{eloptns(3,3,3)}
that is calculated by the routine \texttt{el\_opt}. It requires the response
to the electric field perturbation.

\item
The raman tensor is a real array \texttt{ramtns(3,3,3,nat)} that
gives the derivatives of the dielectric constant when the atom nat moves.
The third index give the direction of the displacement.
It requires the first and the second order response of the wavefunctions
with respect to the electric field perturbation. It is calculated
by the routine \texttt{raman\_mat}.

\item The dynamical matrix is a complex matrix of dimensions
\texttt{(3 * nat, 3 * nat)}. It is calculated by three routines:
\texttt{dynmat0} computes the part that does not require the linear
response of the system. It has an ion-ion term, a term common to NC, US, and
PAW scheme and the nonlinear core correction term. 
The US and PAW schemes have additional parts, 
one of them calculated inside \texttt{dynmat0} with a call to
\texttt{addusdynmat}, and another part calculated in \texttt{drho}.
There is then a contribution that requires the response of the
wavefunctions calculated in \texttt{drhodv} and \texttt{drhodvloc}
which is common to the NC, US, and PAW schemes. The latter two schemes
have other contributions calculated in \texttt{drhodvus}. This
routine contains also the additional PAW term.

\item
5.1.3 Zue is a real array: \texttt{zstarue(3,nat,3)}. The first two indices 
give the atom that moves and the direction, the third gives the electric 
field. 

\item
The electron phonon coefficients are explained in the \PHonon\ user guide. 
\phx\ saves on \texttt{.xml} files $g_{{\bf q},\nu} ({\bf k},i,j)$ 
for all the modes of an irreducible representation. The coefficients are
saved for each {\bf k} and for all the perturbations. Each irreducible
representation is contained in a different file (see below). Note that 
these quantities are gauge dependent, so if you calculate them on 
different machines with the GRID parallelization, you can use them only 
for gauge invariant quantities. Be very careful with it. (still at an 
experimental stage).

\end{itemize}

All the quantities calculated by the phonon code are saved in the
\texttt{fildyn} files with the exception of the
polarization as a function of the complex frequency that is written 
on output, and of the electron phonon coefficients. The output of the
code in the latter case is given by the files {\tt a2Fq2r.\#.\#iq}.

The charge density response to the electric field perturbations and
to the atomic displacements, or the change of the Kohn and Sham
potential can be saved on disk giving appropriate input variables.
These quantities are saved on disk by \texttt{solve\_e} and 
\texttt{solve\_linter}.


\section{GRID parallelization and recover}

The \phx\ code might start from scratch or recover an interrupted run. 
In a recover run the input control flags are assumed to coincide with 
those used in the interrupted run. The required quantities might be
found in \texttt{.xml} recover files and do not need to be recalculated. 
If the quantities are found on file the following flags become 
\texttt{.TRUE.}.

\begin{verbatim}
done_fpol,          if .TRUE. all frequency dependent polarizabilities are known
done_epsil,         if .TRUE. the dielectric constant is known
done_start_zstar,   if .TRUE. zstareu0 is known
done_zeu,           if .TRUE. zeu is known
done_lraman,        if .TRUE. the raman tensor is known
done_elop,          if .TRUE. the electron-optical coefficient is known
done_zue,           if .TRUE. zue is known
done_elph           if .TRUE. the electron-phonon coupling coefficient is known
\end{verbatim}
%done_trans,         if .TRUE. the dynamical matrix is known

The variables that control the grid are:

\begin{verbatim}
comp_iq(nqs)=.TRUE.              ! .FALSE. when this q is not computed in 
                                 ! this run (controlled by start_q, last_q,
                                 ! or by the image controller)

comp_irr_iq(0:3*nat,nqs)=.TRUE.  ! .FALSE. for the representations that are
                                 !  not calculated in this run.
                                 ! (controlled by start_q, last_q, 
                                 !  start_irr, last_irr, 
                                 !  or by the image controller)

comp_iu(nfs)=.TRUE.              ! .FALSE. for the frequencies not calculated 
                                 ! in this run. 

\end{verbatim}

These variables are set at the beginning of the run on the basis of
the input and of the number of images requested by the calculation.
If this is a recover run some of these quantities might be already
available on file. The code checks what is already saved on files and
sets the corresponding flags:

\begin{verbatim}
done_iu(nfs)=.FALSE.        ! .TRUE. when the polarization(iu) is available.

done_iq(nqs)=.FALSE.        ! .TRUE. when the dyn. mat. and, if required, the
                            ! electron-phonon coefficients at the q point 
                            ! have been calculated

done_bands(nqs)=.FALSE.     ! .TRUE. when the bands for that q are already 
                            !  on disk

done_irr_iq(0:3*nat,nqs)=.FALSE. ! The representations that have been already 
                                 ! calculated for each q are set .TRUE.. 
                                 ! The representation 0 is the part of the
                                 ! dynamical matrix computed by drho and
                                 ! dynmat0.

done_elph_iq(3*nat,nqs)=.FALSE.  ! .TRUE. when the electron phonon coefficient 
                                 ! for this irreducible representation and
                                 ! this q is available.

\end{verbatim}

The phonon code might stop in the middle of a self-consistent linear response 
run, or while it is computing the bands. This case is controlled  
by a single code that is read from the files written on disk.
This is an integer that tells where the code stopped. This code
is used in several points to avoid too many flags checks. Saved
on disk in \texttt{.xml} file there is also a string.  
The codes are the following:
\newpage

\begin{verbatim}
!  rec_code   where_rec     status description
!
!    -1000              Nothing has been read. There is no recover file.
!    -50     init_rep.. All displacement have been written on file.
!    -40     phq_setup  Only the displacements u have been read from file
!    -30     phq_init   u and dyn(0) read from file
!    -25     solve_e_fp all previous. Stopped in solve_e_fpol. There
!                       should be a recover file.
!    -20     solve_e    all previous. Stopped within solve_e. There 
!                       should be a recover file.
!    -10     solve_e2   epsilon and zstareu are available if requested. 
!                       Within solve_e2. There should be a recover file.
!     2      phescf     all previous, raman tensor and elop tensor are
!                       available if required.
!     10     solve_linter all previous. Stopped within solve linter. 
!                       Recover file  should be present.
!     20     phqscf     all previous dyn_rec(irr) and zstarue0_rec(irr) are
!                       available.
!     30     dynmatrix  all previous, dyn and zstarue are available.
! 
\end{verbatim}


\section{Parallelization}

The \PHonon\ package uses the same parallelization mechanisms of the \qe\
package. See the Developer manual in the \texttt{Doc} directory
two levels above this one for more information.
It is parallelized on plane-waves, pools, bands, and images.
The \texttt{-ortho} flag is not used. \texttt{Scalapack} routines are not 
used in the \phx\ code.

Each tensor should be collected as soon as it is calculated
and all processors must have the same tensors. Please avoid to collect
tensors in routines distant from where they are calculated. There might be
exception to this rule for efficiency, but please try not to abuse for
small arrays. Only collected quantities are saved on the \texttt{.xml} file,
so they should not depend on the parallelization level. Note that only ionode
writes the \texttt{.xml} files, so you have different \texttt{xml} files only 
for different images. The variables are then broadcasted to all processors 
in an image.

\section{Files produced by ph.x}

The output files of the \pwx\ code are not modified by the \phx\ code. 
Each image of \phx\ creates a new directory called \texttt{outdir/\_ph\#} 
where it writes its files. \texttt{\#} is an integer equal to \texttt{0} 
in a calculation with one image or to the image number when the 
\texttt{-nimage} flag is used. 
There are two sets of files written 
by \phx\ in the \texttt{outdir/\_ph\#} directories: unformatted files 
containing internal arrays, and \texttt{.xml} 
files containing partial results or tensors. The former are in 
\texttt{outdir/\_ph\#} if the input flag \texttt{lqdir=.false.}, or in 
separate subdirectories \texttt{outdir/\_ph\#/prefix.q\_\#iq},
where \texttt{\#iq} is the number of the {\bf q} point. Note that if 
\texttt{lqdir=.false.} (default is \texttt{lqdir=elph})
the disk occupation is reduced but the files of each {\bf q} point are
rewritten by the following {\bf q} so it is not possible to run an 
electron-phonon calculation with \texttt{trans=.false.} and 
\texttt{ldisp=.true.} after generating the induced potentials for a mesh of 
{\bf q} points. 
The \texttt{.xml} files calculated by each image are in the 
\texttt{outdir/\_ph\#/{prefix}.phsave} directory for all {\bf q}-vectors and 
irreps calculated by that images. Before closing the image calculation 
the content of all the \texttt{outdir/\_ph\#/{prefix}.phsave}
directories are copied into \texttt{outdir/\_ph0/{prefix}.phsave} directory, so
it is possible to recover the calculation without using images.
The \phx\ code reads the output of \pwx\ from the \texttt{outdir} directory. 
The wavefunctions are in \texttt{outdir/{prefix}.wfc} files 
while information on the structure of the solid and on the \pwx\ 
run are in the \texttt{outdir/{prefix}.save} directory. The wavefunctions are 
also in this directory if \pwx\ was run with the \texttt{wf\_collect=.true.} 
flag. These files are not modified by \phx. 
At a finite {\bf q} vector, \phx\ runs its own instance of \pwx\
to compute the bands and saves the results into 
the \texttt{outdir/\_ph\#/prefix.q\_\#iq} directory (\texttt{lqdir=.true.}) or
in \texttt{outdir/\_ph\#}. The charge density is copied inside 
these directories before calculating the bands. The output of \pwx\ is 
in files called \texttt{outdir/\_ph\#/prefix.q\_\#iq/{prefix}.wfc} and in 
the directory 
{\tt outdir/\_ph\#/prefix.q\_\#iq/prefix.save} (\texttt{lqdir=.true.}),
or in \texttt{outdir/\_ph\#/{prefix}.wfc} and in 
\texttt{outdir/\_ph\#/{prefix}.save} (\texttt{lqdir=.false.}). With
\texttt{lqdir=.false.} \phx\ saves in 
\texttt{outdir/\_ph\#/{prefix}.bar} the
non self-consistent part of the right hand side of the linear system, 
in \texttt{outdir/\_ph\#/{prefix}.dwf} the change of the wavefunctions. 
The files \texttt{outdir/\_ph\#/} \texttt{{prefix}.igk} contain the ${\bf k}+{\bf G}$ 
lists as in the \pwx\ run.
With US or PAW, files called \texttt{outdir/\_ph\#/{prefix}.prd} 
contain the induced charge density, for all modes. 
Only the part that does not depend on the perturbed wavefunctions 
is contained in these files. With electric field perturbations 
there are also files called \texttt{outdir/\_ph\#/{prefix}.com} that 
contain $P_c x |\psi\rangle$ and are needed for the calculation 
of the Born effective charges.
The mixing routine saves its data in files called 
\texttt{outdir/\_ph\#/{prefix}.mixd}. 
The status of \phx\ is saved at each iteration in files called
\texttt{outdir/\_ph\#/{prefix}.recover}. These files can be used 
to recover the run. All these unformatted files are saved in 
\texttt{outdir/\_ph\#/prefix.q\_\#iq} directory when \texttt{lqdir=.true.}.
Using the input flag \texttt{reduce\_io=.true.} these files can be
kept in memory and saved only at the end of the run if necessary.

In parallel calculations, previous files are split into several files 
that have a final number. Each number labels the processor that wrote the 
file. There are as many files as processors per image. 

The files with the dynamical matrices are written in the directory in
which \phx\ is started and are called \texttt{{fildyn}\#iq} where 
\texttt{\#iq} is the {\bf q}-vector number in a dispersion calculation, 
or is not added in a single-{\bf q} calculation. Only one copy of this 
file is written in a parallel run. When the \texttt{-nimage} option
is used some of these files might be empty (if the corresponding {\bf q}
point has been divided between two or more images). The results are
collected running \phx\ another time (with \texttt{recover=.true.})
without images.

Moreover \phx\ opens a directory called \texttt{outdir/\_ph\#/{prefix}.phsave}.
This directory contains the partial information on the calculation.
These files can be used to recover a run also when the recover file 
is corrupted. In the directory {\tt outdir/\_ph\#/{prefix}.phsave} the 
files are in \texttt{.xml} format. Note that this directory is
always in \texttt{outdir/\_ph\#/} also when \texttt{lqdir=.true.}.
There are several files:

\texttt{control\_ph.xml} contains information on the flags that control
what \phx\ calculates. The content of this file is used mainly for 
checking purposes. The code reads these flags in input and does not need
to reread them from file, but a recover run in which these flags change 
is not allowed. \texttt{control\_ph.xml} contains also the mesh of 
{\bf q}-vectors and their coordinates. This file is written only in a non 
recovered calculation from the routine \texttt{check\_initial\_status} after 
the creation of the {\bf q}-vector mesh. It is read, if 
\texttt{recover=.true.}, at the beginning of 
the run by \texttt{phq\_readin}.

\texttt{status\_run.xml} contains information that tell \phx\ 
at which point the code stopped. It has information on the current 
{\bf q} vector, the current frequency, and a recover code that tells 
\phx\ if it has to expect a recover file and which routine produced this 
recover file. 
\texttt{status\_run.xml} file is rewritten each time the phonon code 
reaches a point from which a new recover is possible.
It is read, if \texttt{recover=.true.},
at the beginning of the run by \texttt{phq\_readin}.

If some routine wrote it, \texttt{tensors.xml} contains the tensors that
have been calculated. Possible tensors are: dielectric constant, 
Born effective charges calculated as derivative of the forces (EU), 
Born effective charges calculated as derivative of the polarization (UE), 
raman tensor, electro-optic coefficient. This file is written by the 
routines that calculate the tensors. 
It is read by the routine \texttt{phq\_recover}, if \texttt{recover=.true.} 
and the {\bf q} vector is $\Gamma$.

If \texttt{polariz} wrote it, \texttt{polariz.xml} contains the frequency
dependent polarizabilities for the frequencies calculated so far. 
It is read by the routine 
\texttt{phq\_recover}, if \texttt{recover=.true.} and the {\bf q} vector
is $\Gamma$.

\texttt{patterns.\#iq.xml} are files written for each {\bf q} vector 
(\texttt{\#iq} is its number).
They contain the information on the displacement patterns that 
transform according to irreducible representations of the small 
group of {\bf q}: number of irreducible representations, their dimensions, 
the displacement patterns and the name of the irreducible representation 
to which each mode belongs. It is written in nonrecover runs by the routine 
\texttt{init\_representations}.
It is read for each {\bf q} vector by \texttt{phq\_setup}. The routine
reads the data of the file with \texttt{iq=current\_iq}.

\texttt{dynmat.\#iq.0.xml} contains the part of the dynamical matrix 
calculated by \texttt{dynmat0} that does not depend on the perturbed 
wavefunctions. It is written by \texttt{dynmat0} and read only in recover runs
by \texttt{phq\_recover}.

\texttt{dynmat.\#iq.\#irr.xml}
contains the contribution to the dynamical matrix at the 
{\bf q} vector \texttt{\#iq} of
the representation \texttt{\#irr}. 
Note that these files can be calculated independently even on 
different machines and collected in a single directory (see the GRID example),
but it is necessary to calculate the patterns file in a single machine and
send it to all the machine where the calculation is run to be sure that
all machines use the same displacement patterns.
When the files \texttt{dynmat.\#iq.\#irr.xml} are present for all 
\texttt{\#irr} of a given \texttt{\#iq} the dynamical matrix for that 
${\bf q}$ can be calculated. If all the \texttt{\#irr} of a given symmetry 
for a given \texttt{\#iq} are present, 
the partial dynamical matrix that can be constructed with this information 
can be diagonalized and the frequencies of the modes of that symmetry can 
be calculated (using the \texttt{ldiag=.true.} flag).
These files are written by \texttt{phqscf} after calculating the 
contribution of the representation to the dynamical matrix by 
\texttt{drhodv}. They are read only in recover runs by the routine 
\texttt{phq\_recover}.

\texttt{elph.\#iq.\#irr.xml} contains the contribution to the electron 
phonon coefficients 
at the {\bf q} vector \texttt{\#iq} of the representation \texttt{\#irr}. 
These files are written by \texttt{elphel} and contain the quantities 
$g_{{\bf q}\nu} ({\bf k}, i, j)$ 
(see User Manual). They are read in recover runs by the routine 
\texttt{phq\_recover}.


\section{The routines of the PHonon package}

The routines of the \PHonon\ package can be divided in groups of  
related task. There are high level drivers that call the
routines that do the actual work and low level routines that make
a single task. Note that the phonon code is tightly integrated in the QE
package, so it uses the routines provided by the 
\texttt{Modules} or by the \texttt{PW/src} directories. Only a brief comment
on the purpose and the use of the routines can be found here. More details
might be written inside the routines themselves. We report here the name of the
file that contains the routines. Each file might contain more than one
routine. Unfortunately sometimes there is no correspondence between the name of
the file and the name of the routine. This is mainly for historical reasons.
We adopt the following convention: if the file and the routine contained inside
have the same name we report only the filename; if the file contains a single
routine with a different name or more than one routine, we report in 
parenthesis the routine name.

Modules that contain the variables used by \phx:

\begin{verbatim}
phcom.f90  Almost all global variables are here.
elph.f90   Variables needed for the electron-phonon part.
ramanm.f90 Variables for Raman calculation.

\end{verbatim}

Global variables allocation and deallocation. Note that some
variables are allocated by \texttt{phq\_readin} and by \texttt{ph\_restart}.

\begin{verbatim}
allocate_phq.f90    This is the main allocation routine in which almost 
                    all global variables are allocated. It needs only the 
                    dimensions defined in pw.x.
allocate_part.f90   Allocate quantities for the partial computation of
                    the dynamical matrix. It is called in phq_readin.          
allocate_pert.f90   Allocate the symmetry matrices in the basis of the 
                    modes. It needs the maximun number of perturbations.              
deallocate_part.f90 Deallocate the variables allocated by allocate part.
deallocate_phq.f90  Deallocate all the ph.x variables allocated in 
                    allocate_phq. The variables allocated in phq_readin
                    or ph_restart should be deallocated by destroy_status_run,
                    contained in ph_restart.       
clean_pw_ph.f90     Clean all variables of pw.x and of ph.x. Used to 
                    reinitialize the calculation at each q.               
\end{verbatim}

Starting point and main programs. The directory \texttt{PHonon/PH} contains 
seven executables whose main programs are:

\begin{verbatim}
phonon.f90     This is the main program of ph.x
q2r.f90        This is the main program of q2r.x
matdyn.f90     This is the main program of matdyn.x
dynmat.f90     This is the main program of dynmat.x
fqha.f90       This is the main program for fqha.x
dvscf_q2r.f90  This is the main program for dvscf_q2r.x
postahc.f90    This is the main program for postahc.x
\end{verbatim}

Reading input, pseudopotentials, and files written by \texttt{pw.x}:

\begin{verbatim}
phq_readin.f90      This is the routine that reads the input, the PP and
                    the punch file of pw.x.
bcast_ph_input.f90  This routine broadcasts the input variables to all
                    processors.
save_ph_input.f90 (save_ph_input_variables) A few input variables are 
                  changed by the ph.x code and are saved by this routine.
                  (restore_ph_input_variables) this routine restores the
                  saved variables.
                  (clean_input_variables) deallocate the saved variables.
\end{verbatim}

Check the initial status of the calculation and decide what has to be
computed:

\begin{verbatim}
check_initial_status.f90  Tests the initial status of the calculation,
                          prepare or reads the mesh of q points and the
                          irreps, divide the work among images and creates
                          the necessary directories in outdir.    
           (image_q_irr) Divide the work among several images.
           (collect_grid_files) Copy the files produced by images in
                         the .phsave directory of the image0.
check_if_partial_dyn.f90  Control partial calculations in phonon. 
check_restart_recover.f90 Check if a restart or recover file is present
                          in the outdir directory       
\end{verbatim}

Note: in the following some of the listed routines are contained in folder
\texttt{LR\_Modules}).
Routines that select the small group of {\bf q} and other symmetry related 
quantities used by the \phx\ code:

\begin{verbatim}
set_small_group_of_q.f90  This is a driver that selects among the s matrices 
               those of the small group of q. Check if q-> -q+G symmetry 
               exists. If modenum > 0 removes also the symmetries that do not
               send the mode in itself.
      (smallg_q) do the actual work of selecting the s matrices.
mode_group.f90 Find the small group of q and of the mode (used with modenum)

smallgq.f90 (set_giq)  Find the G vectors associated to each rotation: Sq=q+G.
\end{verbatim}

%.....sgam_ph is present in the old versions only....
%sgam_ph.f90    Finds the rtau vectors. These are Bravais lattice vectors that
%               link an atom na to its rotated atom nb if these two atoms are
%               not in the same cell. These quantities are needed to rotate 
%               the modes and to symmetrize the potentials.
%\end{verbatim}

Routines that manipulate or generate the irreducible representations, 
the {\bf q}-point mesh and all the preparatory stuff that is needed by the
\phx\ code:

\begin{verbatim}
q_points.f90   Generate the mesh of q vectors.
check_q_points_sym.f90  Check if the q point mesh is compatible with the fft
               mesh used by q2r.x.

init_representations.f90 This is a driver that initialize all the irreps 
               for all q vectors. First it finds the small group of q 
               and then calls find_irrep for each q.
    (initialize_grid_variables) This routine reads the irreps from file and
               sets the variables that define the grid of q and irreps.

find_irrep.f90 Find the irreps of a given q calling set_irr or set_irr_nosym.
   (find_irrep_sym) is a driver that allocate the symmetry matrices in
                the basis of the modes and calls set_irr_sym to calculate
                them.
  
random_matrix.f90 Generate the random matrix to calculate the irreps.

set_irr.f90    Call random_matrix to generate a random matrix and
               symmetrize it. The eigenvectors are the irreps. Count their
               degeneracy and if search_sym is true find their symmetry.

set_irr_nosym.f90 As set_irr in the case in which the system has no
               symmetry or symmetry is not used.

set_irr_sym.f90 Calculate the rotation matrices on the irreps basis.

\end{verbatim}

High level drivers that make the actual calculation:

\begin{verbatim}
prepare_q.f90  Decides if a given q has to be calculated and if it needs
            the band calculation or just to open the k-point list.

initialize_ph.f90 Initialization driver. It calls the other initialization
             routines one after the other: allocate_phq, phq_setup,
             phq_recover, phq_summary, openfilq, and phq_init.               

phq_setup.f90 Setup many quantities needed by the phonon. The
             most significant are: the local+SCF potential, derivatives
             of xc potential, using dmxc or similar functions and setup_dgc,
             alpha_pv and occupated bands, rotation matrices on the 
             basis of the mode (calling find_irrep_sym), setup the gamma_gamma
             tricks.

phq_init.f90 Setup more complex quantities that require the implementation
             of more complex formula.
             It is a driver that uses auxiliary routines:
             set_drhoc, setlocq, dvanqq, drho, dynmat0. Moreover it computes
             becp1, alphap, eprec. 

phescf.f90  This is the main driver for the electric field perturbations.
            It decides what to compute on the basis of the input flags.
            It can compute polarization, epsilon, raman, and elop.

phqscf.f90  This is the main driver for the phonon perturbation. It has 
            a loop over the irreps at a given q. It calls solve_linter 
            to calculate the perturbed wavefunctions and potentials, drhodv 
            to update the dynamical matrix and add_zstar_ue to update the 
            zue effective charges.
\end{verbatim}

Opening and closing files:

\begin{verbatim}
openfilq.f90 Open almost all files of the ph.x code.
close_phq.f90 Close the above files if opened.                  
\end{verbatim}

Drivers that compute the band structure using the \pwx\ routines:

\begin{verbatim}
run_nscf.f90 This routine runs pw.x to calculate the bands. It calls
             init_run, electrons, and punch. However the functionalities
             of setup are provided by setup_nscf.
set_defaults_pw.f90 (setup_nscf)
            This routine sets the input of pw.x with default values. 
            It sets the k point list.
\end{verbatim}

Routines that compute quantities independent from the perturbed wavefunctions
that are used in the rest of the code (mainly US/PAW part). These
routines are called by \texttt{phq\_init}:

\begin{verbatim}
dvanqq.f90 This routine computes four of the five integrals  
           of the augmentation functions and its derivatives with 
           derivatives of the local potential. Needed only in the US/PAW case.

drho.f90   This is a driver that computes the parts of the
           induced charge density and of the dynamical matrix that
           do not depend on the change of the wavefunctions. These
           terms are present only in the US/PAW case.            
           It calls many of the following routines.

compute_becsum_ph.f90  This routine computes becsum.        
compute_alphasum.f90   This routine computes alphasum.           
compute_becalp.f90     Compute the product of vkb and psi_{k+q} or of the
                       derivative of vkb and psi_{k+q}

compute_drhous.f90  This is a driver that makes a loop over the k points
           to accumulate, using incdrhous, the part of the induced
           charge density due to the change of the orthogonality
           constraint. All the modes are computed here. (US/PAW case only).
                     
compute_drhous_nc.f90  As compute_drhous in the noncollinear/so case.

incdrhous.f90  Accumulate for a given k point and a given mode
               the contribution to the induced charge density due to the
               change of the orthogonality constraint.
incdrhous_nc.f90  As incdrhous in the noncollinear/so case.

compute_nldyn.f90  Computes the orthogonality term in the dynamical matrix.
                   Used only in the US/PAW case.         

compute_weight.f90 Compute the composite weights for metals.         

qdipol_cryst.f90 This routine computes the dipole moment of the augmentation
           functions.

setlocq.f90 This routine computes the local potential at q+G.
compute_dvloc.f90 Computes the change of the local potential due to
            a phonon perturbation.     

setqmod.f90 Computes (q+G)**2
hdiag.f90   Computes the kinetic energy.                      
\end{verbatim}

Lower level drivers that set up and solve the linear system to calculate
the response of the system to a perturbation:

\begin{verbatim}
solve_linter.f90 Driver to calculate the phonon perturbation.
solve_e.f90      Driver to calculate the static electric field perturbation.
solve_e_fpol.f90 Driver to calculate the electric field perturbation at 
                 imaginary frequency.
solve_e2.f90     Driver for the electric field perturbation at second order.
solve_e_nscf.f90 A simplified version of solve_e in which the induced
                 self consistent potential is already known. This 
                 routine is used in dhdrhopsi.f90.
\end{verbatim}

Routines used by the above drivers to do their job. Some of these routines
are used by all drivers, others are specific for a given perturbation:

\begin{verbatim}
dvpsi_e.f90      Compute the right hand side of the linear system in
                 the electric field case (only non SCF part). It
                 uses commutator_Hx_psi.
commutator_Hx_psi.f90  Compute the commutator of the Hamiltonian with r.       

dvpsi_e2.f90   Compute the right hand side of the linear system for
               the second order perturbation in the electric field case.                   
dvqpsi_us.f90    Compute the right-hand side of the linear system in the
                 phonon case (Only non SCF part). It uses dvqpsi_us_only.
dvqpsi_us_only.f90  The part of dvqpsi due to the nonlocal potential.          

cft_wave.f90    Wavefunction from real to reciprocal space and return.
apply_dpot.f90  Add the contribution of the change of the SCF potential  
                to the right-hand side of the linear system.               
adddvscf.f90    Add the additional US/PAW contributions to the right-hand
                side of the linear system (phonon case).
adddvepsi_us.f90 As adddvscf for the electric field case.

orthogonalize.f90 Apply the projector on the valence bands to the right-hand
                side of the linear system. Deal with both insulators and metals.

cgsolve_all.f90  Solve the linear system with an iterative conjugate gradient
                 method.                

pcgreen.f90      Orthogonalize and solve the linear system. Used by 
                 solve_e2 and solve_e_nscf instead of the more standard method.
                 Call cgsolve_all for doing the actual calculation.

gmressolve_all.f90 Solve the linear system in the case of 
                   imaginary frequency polarizability calculation.           

ch_psi_all.f90     Apply H+Q-eS to the wavefunctions. Used by the routine that 
                   solves the linear system.

cch_psi_all.f90    As ch_psi_all for complex e. Used by gmresolve_all.

h_psiq.f90         Calculate h psi for k+q. Compute also S psi.              
cg_psi.f90         Apply the preconditioning.               
ccg_psi.f90        A complex preconditioning for gmresolve_all.

incdrhoscf.f90     Add the contribution of the computed set of perturbed
                   wavefunction at a given k and for a given perturbation
                   to the perturbed change density.              
incdrhoscf_nc.f90  As incdrhoscf for the noncollinear/so case.         
addusdbec.f90      Add the contribution of the computed set of perturbed
                   wavefunctions at a given k and for a given perturbation
                   to the change of the becsum.               
addusdbec_nc.f90   As addusdbec for the noncollinear/spin-orbit case.

addusddens.f90     Add the US/PAW augmentation contribution to the change 
                   of the charge density. (Phonon case)
addusddense.f90    Add the US/PAW augmentation contribution to the change 
                   of the charge density. (Electric field case)              

dv_of_drho.f90     Compute the change of the SCF potential given the change
                   of the SCF charge density.

mix_pot.f90        Mix input and output induced SCF potentials. In the
                   PAW case mixes also dbecsum.

newdq.f90          Integrate the augmentation function with the change of
                   the SCF potential (US/PAW case only). In the PAW case
                   add the PAW contribution to the change of the coefficients
                   of the nonlocal potential. The coefficients calculated
                   here are used by adddvscf (phonon case) and adddvepsi_us
                   (electric field case).

PW/src/paw_onecenter.f90: 
                   (PAW_dpotential) Computes the change of the coefficients 
                   on the nonlocal potential due to the perturbation
                   (Only PAW case).

ef_shift.f90       Accounts for the change of the Fermi level in metals at
                   the gamma point.
  (ef_shift_paw)   Account also for the change of dbecsum. 

localdos.f90       Computes the local DOS.
addusldos.f90      US contribution to the local DOS.               
\end{verbatim}

Routines that calculate the derivative of the xc potential.
Note that some of them are also in \texttt{Module/funct.f90}:

\begin{verbatim}
setup_dgc.f90   Sets the derivative of the xc functionals needed to
                calculate the change of the potential. It is called by
                phq_setup.
d2mxc.f90       LDA second derivatives of the xc functional            
dgradcorr.f90   Change of the GGA part of the xc potential.          
compute_vsgga.f90  Additional GGA term present in the noncollinear/spin-orbit
                 case.
\end{verbatim}

Routines that deal with the nonlinear core correction (NLCC):

\begin{verbatim}
set_drhoc.f90  Fourier transform of the core charge at q+G. Called by
               phq_setup.
addcore.f90    Change of the core charge for a phonon perturbation.            
               Used by dv_of_drho and addnlcc.
dynmatcc.f90   NLCC contribution to the dynamical matrix independent from 
               the perturbed wavefunctions. Called by dynmat0.
addnlcc.f90    The nlcc part of the dynamical matrix that depends on the
               perturbed potential. Called by solve_linter.
\end{verbatim}

Frequency dependent polarizability:

\begin{verbatim}
polariz.f90   Computes the frequency dependent polarizability, given dpsi.
\end{verbatim}

Dielectric tensor:

\begin{verbatim}
dielec.f90    Computes the dielectric tensor, given dpsi.          
\end{verbatim}

Born effective charges:

\begin{verbatim}
add_zstar_ue.f90     Add the contribution to zue due to dpsi induced by 
                     a phonon      
add_zstar_ue_us.f90  Add the US contribution to zue             
zstar_eu.f90         Compute zeu from the dpsi induced by an electric field 
zstar_eu_us.f90      Add the US/PAW contribution to zeu.
add_dkmds.f90        Additional terms for the US/PAW Born effective charges    
psidspsi.f90         Calculate <psi_v'|ds/du|psi_v>
add_for_charges.f90  Calculate dS/du P_c [x, H-eS] |psi>            
addnlcc_zstar_eu_us.f90  Add nlcc contribution to zeu         
dvkb3.f90     Derivative of beta functions with respect to q and tau.  
\end{verbatim}

Raman tensor:

\begin{verbatim}
raman.f90       This is the main driver for the raman calculation. It 
                computes the second order response calling solve_e2 and
                the right hand side calling dvpsi_e2.
raman_mat.f90   Computes and writes the raman tensor.
dhdrhopsi.f90   Computes Pc [DH,Drho] |psi>.            
dielec_test.f90 Compute the dielectric constant with the quantities 
                calculated inside dhdrhopsi. 
\end{verbatim}

Electro-optic tensor:

\begin{verbatim}
el_opt.f90     Computes the electro-optic tensor.
\end{verbatim}

Dynamical matrix:

\begin{verbatim}
dynmat0.f90      Driver for the part of the dynamical matrix independent
                 from the perturbation. It calls dynmatcc, d2ionq, and 
                 dynmat_us. This routine is called by init_phq. 

dynmat_us.f90    Expectation value of the second derivative of the 
                 local and nonlocal potentials.                 
addusdynmat.f90  US/PAW contribution to the second derivative of 
                 the potential. There are terms due to the change of the
                 augmentation function.               
d2ionq.f90       Ewald contribution.            

drhodv.f90       Contribution to the dynamical matrix due to the change
                 of the wavefunctions.
drhodvnl.f90     Accumulate the contribution to the dynamical matrix due 
                 to the change of the wavefunctions (Only the contribution
                 of the nonlocal PP). Called at each k point.        
drhodvloc.f90    As drhodvnl for the local potential. It can be calculated
                 as an integral of the potential and the induced charge 
                 density.
drhodvus.f90     A term present only in the US/PAW case. Integral of the
                 induced SCF potential and the change of the charge at
                 fixed wavefunctions. It is called in solve_linter because
                 the induced potential is not available outside.      

dynmatrix.f90    Is a driver that collects the dynamical matrix, checks if
                 all representations have been calculated, symmetrize the
                 dynamical matrix, computes the matrices rotated in all 
                 equivalent q and diagonalizes the matrix. The same is
                 done for zue.                 
\end{verbatim}

Electron-phonon coupling coefficients:

\begin{verbatim}
elphon.f90   This is a driver that in the case trans=.false. reads the
             induced self-consistent potential and calculates the
             electron-phonon matrix elements. It reads also the 
             dynamical matrix and diagonalizes it.
    (readmat) read the dynamical matrix.
    (elphel) compute the electron-phonon matrix elements.                 
    (elphsum) make a sum over the BZ of the square moduli of the 
              el-ph matrix elements and compute phonon linewidths. This
              routine makes a linear interpolation on k points 
              (still unsettled). Require compatibility between q and k 
              meshes.
    (elphsum_simple) As elphsum but without the interpolation. It can be
              used at arbitrary q.
el_ph_collect.f90  Collect the electron-phonon matrix elements among pools.
clinear.f90
ahc.f90       Calculate first- and second-order electron-phonon quantities
              for the calculation of phonon-induced electron self-energy.
\end{verbatim}

Routines that write the output quantities:

\begin{verbatim}
phq_summary.f90  Summarize what has been read from the pw output and
                 what has been calculated by phq_setup.

summarize.f90    Write the tensors on output. 
     (summarize_epsilon) write the dielectric tensor.
     (summarize_zeu) write zeu.
     (summarize_zue) write zue.
     (summarize_elopt) write the electro-optic tensor.
     (summarize_fpol) write the frequency dependent polarizability.


write_epsilon_and_zeu.f90 Use the routines of summarize, but contain also old 
                 instructions to write the dielectric constant and 
                 the Born effective charges in the dynamical matrix file.

write_modes.f90  
      (write_modes_out) This routine writes the modes on output. It is called
                 by set_irr and by phq_summarize.

write_qplot_data.f90  Write a file that can be read by plotband with
                  q vectors and phonon frequencies.

write_ramtns.f90  Write the raman tensor.

write_eigenvectors.f90 Used by matdyn to write the eigenvectors on output.
                 Writes the displacements in several format suited to some
                 molecular graphics programs.
\end{verbatim}

Routines that write on file the induced charge densities:

\begin{verbatim}
punch_plot_e.f90 Write the change of the charge due to an electric field.
davcio_drho.f90  Write the change of the charge due to a phonon perturbation.  
\end{verbatim}

Routines that read or write the \texttt{.xml} files with the partial results:

\begin{verbatim}
ph_restart.f90  This file contains many routines to write and read the .xml
                files that contain the partial results of ph.x. See the section 
                "file produced by ph.x". 
       (ph_writefile) This routine can be called from external routines to
                write the tensors on file.
       (ph_readfile)  This routine can be called from external routines to
                read the tensors from file.
       (check_directory_phsave) This routine tries to read the files in the
                phsave directory to check what has been already calculated.
       (check_available_bands) This routine search on the outdir directory
                for the bands files to see if they have been already 
                calculated.
       (allocate_grid_variables) This routine allocates space for the variables
                that control the grid calculation.
       (destroy_status_run) This routine deallocates the variables that 
                control the grid and the variables allocated by phq_readin
                or ph_restart.

io_dyn_mat.f90  This file contains the routines that read and write the
                dynamical matrix in .xml format.

io_dyn_mat_old.f90 These are the routines that read and write the dynamical
                matrix in the old format (not .xml).
\end{verbatim}

Routines that read or write the recover file:

\begin{verbatim}
phq_recover.f90 This routine reads the recover files and reconstruct the
                status of the calculation so far.
write_rec.f90   This file contains the routine that writes the 
                recover file (in unformatted form).
     (read_rec) read the recover file.
\end{verbatim}

Symmetrization of induced potentials:

\begin{verbatim}
symdvscf.f90   Symmetrize the change of the potentials due to a set of
               perturbations that form an irreducible representation.
syme.f90       Symmetrize the change of potentials due to electric field
               perturbations.
sym_dmag.f90   Symmetrize the change of B_xc due to a set
               of phonon perturbations.
sym_dmage.f90  Symmetrize the change of B_xc due to a set of electric field
               perturbations
syme2.f90      Symmetrize the potential of the second order response.
\end{verbatim}

and parallel routines that collect on a single processor 
the quantity to symmetrize and call the previous routines:

\begin{verbatim}
psymdvscf.f90   Parallel version of symdvscf.
psyme.f90       Parallel version of syme.
psym_dmag.f90   Parallel version of sym_dmag.
psym_dmage.f90  Parallel version of sym_dmage.
psyme2.f90      Parallel version of syme2.
\end{verbatim}

Symmetrization of tensors or other quantities:

\begin{verbatim}
symdyn_munu.f90 Symmetrize a dynamical matrix on the basis of the modes,
                transforming it in the cartesian basis and applying
                symdynph_gq.
symdynph_gq.f90 Symmetrize a dynamical matrix written in cartesian coordinates.
star_q.f90      Given a q point finds all the q in its star. 
q2qstar_ph.f90  Generate the dynamical matrix in all the q of the star.
rotate_and_add_dyn.f90 Rotate a dynamical matrix with a given symmetry
                operation.
tra_write_matrix.f90 Symmetrize the dynamical matrix written in the basis 
              of the modes, brings it in cartesian form and write it.
trntnsc.f90   Transform a complex 2D tensor from the crystal basis to the 
              cartesian basis or vice-versa.
sym_def.f90   Symmetrize the change of the Fermi level due to the phonon
              perturbations.
sym_and_write_zue.f90  Symmetrize zue.
symm.f90      Symmetrize the electron-phonon coefficients.
rotate_pattern_add.f90  These are a set of auxiliary routines that manipulate
              the dynamical matrix in different forms. See the heading
              of this matrix to see its capabilities.
\end{verbatim}

Routines that perform the symmetry analysis of the eigenvectors to find
to which irreducible representation they belong:

\begin{verbatim}
prepare_sym_analysis.f90 Prepare the quantities for the symmetry analysis.
symmorphic_or_nzb.f90  A function that checks if symmetry analysis can be
                       carried out. It returns true if q is not at zone border
                       or if the group is symmorphic.
find_mode_sym.f90    Symmetry analysis of the modes.              
\end{verbatim}

Routines that apply the Clebsch Gordan coefficients for the spin-orbit
part of the code:

\begin{verbatim}
transform_alphasum_nc.f90  Apply the coefficients to alphasum (no-so case) 
transform_alphasum_so.f90  Apply the coefficients to alphasum (so case)
transform_dbecsum_nc.f90   Apply the coefficients to dbecsum (no-so case)
transform_dbecsum_so.f90   Apply the coefficients to dbecsum (so case)
transform_int_nc.f90       Apply the coefficients to the integrals (no-so case)
transform_int_so.f90       Apply the coefficients to the integrals (so case)
set_int12_nc.f90        This is a driver that call the previous routines
                        according to the type of PP.
\end{verbatim}

Routines that apply the \texttt{gamma\_gamma} trick:

\begin{verbatim}
find_equiv_sites.f90              
generate_dynamical_matrix_c.f90   
generate_effective_charges_c.f90 
set_asr_c.f90
\end{verbatim}

Routine for the Fourier interpolation of the phonon potential:

\begin{verbatim}
dvscf_interpolate.f90
\end{verbatim}

Miscellaneous routines:

\begin{verbatim}
print_clock_ph.f90   Print timings info.
stop_ph.f90          Stops the phonon code closing all the files.
rigid.f90            Used by matdyn and dynmat to compute the long range
                     electrostatic part of the dynamical matrix.
dyndia.f90           Diagonalizes the dynamical matrix.                        
\end{verbatim}

Obsolete routines that are here for compatibility with other codes that
might use them:

\begin{verbatim}
obsolete.f90
\end{verbatim}

Development routines provided by some developers but still incomplete,
or used in proprietary codes not yet in the QE distribution, or added and 
forgotten:

\begin{verbatim}
acfdtest.f90              
read_wfc_rspace_and_fwfft.f90
dfile_autoname.f90          
dfile_star.f90             
rotate_dvscf_star.f90
q_points_wannier.f90
set_dvscf.f90
ep_matrix_element_wannier.f90     
io_pattern.f90                    
cgsolve_all_imfreq.f90            
q2qstar.f90
write_matrix.f90
chi_test.f90                      
\end{verbatim}

\section{Suggestion for developers}

If you plan to add something to the \texttt{PHonon} package follow these
simple rules:

\begin{itemize}

\item
All quantities that do not require the perturbed wavefunctions, are
calculated in setup or by calling a separate routine in \texttt{phq\_init}. 

\item
The quantities that require the perturbed wavefunctions due to an
electric field are calculated by a separate routine after 
\texttt{solve\_e} in the routine \texttt{phescf}.

\item
The quantities that require the perturbed wavefunctions due to an
atomic displacement are accumulated by calling a separate routine
in phqscf after \texttt{solve\_linter}. 
NB: the perturbed wavefunctions are saved in a file that is rewritten at
each new \texttt{irrep}.

\item
After calculating a quantity, it has to be saved in the directory
\texttt{outdir} in an \texttt{.xml} file, by adding it to the list 
of variables in the routine \texttt{write\_tensors}  
(preferable), or by writing a routine similar to \texttt{write\_tensors} 
that writes a separate file. The same quantity must be read by 
\texttt{read\_tensors} or by writing a separate routine.

\item
If you introduce the calculation of a new quantity in the phonon code 
and save it in the \texttt{.xml} file, please add also the associated flags 
that control the calculation: 
\texttt{lquantity} is read in input and tells \phx\ that that quantity must be 
calculated, \texttt{done\_quantity} tells \phx\ that that quantity 
was available in the \texttt{.xml} files and should not be recalculated, 
\texttt{comp\_quantity} can be introduced if the quantity depends on 
{\bf q} or on the frequency and tells \phx\ that that quantity must be 
calculated in this run. The image controller can divide the work among images
by setting the array \texttt{comp\_quantity}. At each {\bf q} point and 
at each frequency the quantity must be saved in the \texttt{.xml} file.
Please update the image controller to add the additional work that the 
calculation of your quantity involves and make a single image calculate it 
or divide the work among different images.

\item
Please, try to avoid opening files inside routines.
Files must be opened in \texttt{openfilq} and closed in \texttt{close\_phq}.

\item
Global variables must be allocated in \texttt{allocate\_phq}, directly in the 
routine, or by calling a separate routine that allocates all
your new variables. The same variables must be deallocated in 
\texttt{deallocate\_phq}, by a separate routine or by adding them to the
list of variables. Note that at each new {\bf q} point these variables are
deallocated and reallocated. 

\item
Variables that control the grid should not be deallocated at
each new {\bf q} point must be allocated in \texttt{allocate\_grid\_variables} 
and deallocated in \texttt{destroy\_status\_run}.
A few arrays that must be read from input are allocated in 
\texttt{phq\_readin} after reading their size and deallocated in
\texttt{destroy\_status\_run}.

\item
Preferably global variables are calculated by in a single routine 
and used by the other routines. In particular routines are not allowed to 
modify:
\begin{itemize}
\item
  The variables calculated by \pwx.
\item
  The modes.
\item
  The variables that describe the symmetry of the small group of {\bf q}.
\item
  The variables that describe the response of the ultrasoft quantities
        (e.g. \texttt{int1}, \texttt{int2}, ..., \texttt{alphasum}, 
\texttt{becsum}, \texttt{dpqq}, etc.).
\end{itemize}
If you need to modify these quantities, please allocate new variables 
and copy the variables of the phonon on them.

\item
If you want to establish a new recover point, add the appropriate
\texttt{rec\_code} in the list above. The point in which the code stopped 
is saved in \texttt{prefix.phsave/status\_run.xml}. 

\end{itemize}

If you are searching for some interesting project to contribute to the
\texttt{PHonon} package, please read the header of \texttt{phonon.f90}
and implement some feature that is not yet ready. Ideally all quantities
should be at level [10], presently level [5] is still experimental and
some quantities are at level [1].


\section{File Formats}
\PHonon\ recover file specifications:

Format name: QEXML \\
Format version: 1.4.0 \\

The structure of the file \texttt{status\_run.xml} is:

\begin{verbatim}
<Root>
  <STATUS_PH>
    <STOPPED_IN>
     <where_rec>
    </STOPPED_IN>
    <RECOVER_CODE>
     <rec_code>
    </RECOVER_CODE>
    <CURRENT_Q>
     <current_iq>
    </CURRENT_Q>
    <CURRENT_IU>
     <current_iu>
    </CURRENT_IU>
  </STATUS_PH>
</Root>
\end{verbatim}

The structure of the file \texttt{control\_run.xml} is:

\begin{verbatim}
<Root>
  <HEADER>
    <FORMAT>
    <CREATOR>
  </HEADER>
  <CONTROL>
    <DISPERSION_RUN>
     <ldisp>
    </DISPERSION_RUN>
    <ELECTRIC_FIELD>
     <epsil>
    </ELECTRIC_FIELD>
    <PHONON_RUN>
     <trans>
    </PHONON_RUN>
    <ELECTRON_PHONON>
     <elph>
    </ELECTRON_PHONON>
    <EFFECTIVE_CHARGE_EU>
     <zeu>
    </EFFECTIVE_CHARGE_EU>
    <EFFECTIVE_CHARGE_PH>
    <zue>
    </EFFECTIVE_CHARGE_PH>
    <RAMAN_TENSOR>
    <lraman>
    </RAMAN_TENSOR>
    <ELECTRO_OPTIC>
    <elop>
    </ELECTRO_OPTIC>
    <FREQUENCY_DEP_POL>
    <fpol>
    </FREQUENCY_DEP_POL>
  </CONTROL>
  <Q_POINTS>
    <NUMBER_OF_Q_POINTS>
     <nqs>
    </NUMBER_OF_Q_POINTS>
    <UNITS_FOR_Q-POINT>
    <Q-POINT_COORDINATES>
    <x_q(3,nqs)>
    </Q-POINT_COORDINATES>
  </Q_POINTS>
</Root>
\end{verbatim}

The structure of the file \texttt{tensors.xml} is:

\begin{verbatim}
<Root>
  <EF_TENSORS>
    <DONE_ELECTRIC_FIELD>
    <done_epsil>
    </DONE_ELECTRIC_FIELD>
    <DONE_START_EFFECTIVE_CHARGE>
    <done_start_zstar>
    </DONE_START_EFFECTIVE_CHARGE>
    <DONE_EFFECTIVE_CHARGE_EU>
    <done_zeu>
    </DONE_EFFECTIVE_CHARGE_EU>
    <DONE_EFFECTIVE_CHARGE_PH>
    <done_zue>
    </DONE_EFFECTIVE_CHARGE_PH>
    <DONE_RAMAN_TENSOR>
    <done_raman>
    </DONE_RAMAN_TENSOR>
    <DONE_ELECTRO_OPTIC>
    <done_elop>
    </DONE_ELECTRO_OPTIC>
    <DIELECTRIC_CONSTANT>
    <epsil>
    </DIELECTRIC_CONSTANT>
    <START_EFFECTIVE_CHARGES>
    <zstareu0>
    </START_EFFECTIVE_CHARGES>
    <EFFECTIVE_CHARGES_EU>
    <zstareu>
    </EFFECTIVE_CHARGES_EU>
    <RAMAN_TNS>
    <ramantns>
    </RAMAN_TNS>
    <ELOP_TNS>
    <eloptns>
    </ELOP_TNS>
    <EFFECTIVE_CHARGES_UE>
    <zstarue>
    </EFFECTIVE_CHARGES_UE>
  </EF_TENSORS>
</Root>
\end{verbatim}


The structure of the file \texttt{patterns.\#iq.xml} is:

\begin{verbatim}
<Root>
  <IRREPS_INFO>
    <QPOINT_NUMBER>
      <iq>
    </QPOINT_NUMBER>
    <QPOINT_GROUP_RANK>
       <nsymq>
    </QPOINT_GROUP_RANK>
    <MINUS_Q_SYM>
     <minus_q>
    </MINUS_Q_SYM>
    <NUMBER_IRR_REP>
     <nirr> 
    </NUMBER_IRR_REP>
#for each irr    
    <REPRESENTION.irr>
      <NUMBER_OF_PERTURBATIONS>
        <npert(irr)> 
      </NUMBER_OF_PERTURBATIONS>
#for each ipert
      <PERTURBATION.ipert>
        <SYMMETRY_TYPE_CODE>
         <num_rap_mode>
        </SYMMETRY_TYPE_CODE>
        <SYMMETRY_TYPE>
         <name_rap_mode>
        </SYMMETRY_TYPE>
        <DISPLACEMENT_PATTERN>
         <u>
        </DISPLACEMENT_PATTERN>
      </PERTURBATION.ipert>
    </REPRESENTION.irr>
  </IRREPS_INFO>
</Root>
\end{verbatim}

The structure of the file \texttt{dynmat.\#iq.\#irr.xml} is:

\begin{verbatim}
<Root>
  <PM_HEADER>
    <DONE_IRR>
     done_irr(irr)
    </DONE_IRR>
  </PM_HEADER>
  <PARTIAL_MATRIX>
    <PARTIAL_DYN>
     <dynmat_rec>
    </PARTIAL_DYN>
  </PARTIAL_MATRIX>
</Root>
\end{verbatim}

The structure of the file \texttt{elph.\#iq.\#irr.xml} is:

\begin{verbatim}
<Root>
  <EL_PHON_HEADER>
    <DONE_ELPH type="logical" size="1">
     <done_elph_iq(irr,iq)>
    </DONE_ELPH>
  </EL_PHON_HEADER>
  <PARTIAL_EL_PHON>
    <NUMBER_OF_K>
      <nksqtot> 
    </NUMBER_OF_K>
    <NUMBER_OF_BANDS>
        <nbnd> 
    </NUMBER_OF_BANDS>
#for each ik
    <K_POINT.ik>
      <COORDINATES_XK>
       xk(ik)
      </COORDINATES_XK>
      <PARTIAL_ELPH>
       el_ph_mat_rec_col 
      </PARTIAL_ELPH>
    </K_POINT.ik>
#enddo
  </PARTIAL_EL_PHON>
</Root>


</Root>
\end{verbatim}


\section{Bibliography}

\begin{itemize}
\item
Original idea and first implementation: 
S. Baroni, P. Giannozzi, and A. Testa, 
``Green’s-function approach to linear response in solids''
Phys. Rev. Lett.  {\bf 58}, 1861 (1987).
P. Giannozzi, S. de Gironcoli, P. Pavone, and S. Baroni,
``Ab initio calculation of phonon dispersions in semiconductors''
Phys. Rev. B {\bf 43}, 7231 (1991).

\item
NC, phonon in metals: 
S. de Gironcoli, 
``Lattice dynamics of metals from density-functional perturbation theory''
Phys. Rev. B {\bf 51}, 6773 (1995).

\item
General overview of DFPT: 
S. Baroni, S. de Gironcoli, A. Dal Corso, and P. Giannozzi
``Phonons and related properties of extended systems from density
functional perturbation theory'', Rev. Mod. Phys. {\bf 73}, 515 (2001).

\item
NC, Raman tensor:
Michele Lazzeri and Francesco Mauri,
``High-order density-matrix perturbation theory''
Phys. Rev. B {\bf 68}, 161101 (2003).

\item
GGA dynamical matrix:
F. Favot and A. Dal Corso,
``Phonon dispersions: Performance of the GGA approximation'',
Phys. Rev. B. {\bf 60}, 11427 (1999).

\item
LSDA, spin-GGA dynamical matrix:
A. Dal Corso and S. de Gironcoli,
 ``{\it Ab-initio} phonon dispersions of Fe and Ni'',
Phys. Rev. B {\bf 62}, 273 (2000).

\item
US-PPs dynamical matrix:
A. Dal Corso
``Density functional perturbation theory with ultrasoft pseudopotentials'',
Phys. Rev. B {\bf 64}, 235118 (2001).

\item 
US-PPs dielectric constant:
J. T\'obik and A. Dal Corso,
``Electric fields with ultrasoft pseudo-potentials: applications to
benzene and anthracene'', Jour. of Chem. Phys. {\bf 120}, 9934 (2004).

\item 
US-PPs + spin-orbit dynamical matrix:
A. Dal Corso, ``Density functional perturbation theory for lattice
dynamics with fully relativistic ultrasoft pseudopotentials: application
to fcc-Pt and fcc-Au'', Phys. Rev. B {\bf 76}, 054308 (2007).

\item 
PAW dynamical matrix:
A. Dal Corso,
``Density functional perturbation theory within the projector augmented wave
method'', Phys. Rev. B {\bf 81}, 075123 (2010).

\item
NC, Electron-phonon interaction:
F. Mauri, O. Zakharov, S. de Gironcoli, S. G. Louie, and M. L. Cohen,
``Phonon Softening and Superconductivity in Tellurium under Pressure''
Phys. Rev. Lett. {\bf 77}, 1151 (1996).

\item
US-PPs, Electron-phonon interaction:
M. Wierzbowska, S. de Gironcoli, P. Giannozzi,
``Origins of low- and high-pressure discontinuities of $T_{c}$ in niobium''
arXiv:cond-mat/0504077.

\end{itemize}
\end{document}
